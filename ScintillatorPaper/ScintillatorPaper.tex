\documentclass{JINST}

\usepackage{graphicx,subfigure}
\let\ifpdf\relax 

%\title{Optimizing Liquid Scintillators for Low Energy Events via Wavelength-Shifter and Quantum-Dot Doping} 
%From Lindley
\title{Optimizing the Solvent and Fluor Concentration for Quantum-Dot-Doped Liquid Scintillator}

\author{R.~Schofield\setcounter{footnote}{0}\thanks{corresponding author}~$^a$,Y.~Ungar$^a$, C.~Aberle$^a$, C.~Coy$^a$, T.~De Guillebon$^a$, A.~Elagin, A.~Ierokomos$^a$, B.~Naranjo~$^a$ L.~Winslow$^b$\\
\llap{$^a$}University of California, Los Angeles, Los Angeles, CA 90095, USA\\
%%\llap{$^b$}Ecole Normale Superieure de Cachan, 94235 Cachan cedex, France\\
%%\llap{$^c$}University of California, Berkeley, Berkeley, CA 94704, USA\\
\llap{$^b$}Massachusetts Institute of Technology, USA\\
E-mail: \email{rschofield@physics.ucla.edu}}

%Original.
%\abstract{Liquid scintillator-based detectors can be made more sensitive to low energy events by increasing the scintillator's light yield and enabling direction reconstruction of particles. This study presents the optimal concentration of wavelength-shifter 2,5-Diphenyloxazole (PPO) in various solvents.   The light yield of seven different scintillators (Toluene, Pseudocumene (PC), Linear Alkyl Benzene (LAB), Phenylcyclohexane (PCH), Phenyl-o-Xylyl Ethane (PXE) and two grades of Di-Isopropyl Napthalene (DIN)) are measured and modeled as a function of the concentration of PPO. In addition, the optical properties of 2~g/L PPO in toluene doped with quantum-dots (QDs) are measured. QDs' ability to fine-tune the absorption and emission spectra of liquid scintillators demonstrate their viability to conserve the directional Cherenkov signal produced by interactions in the liquid scintillator.}

%From Lindley.
\abstract{Quantum dots are semiconducting nanocrystals. Their unique optical properties make them attractive secondary wavelength shifters for liquid scintillator and their composition makes them attractive dopants for loading metals into liquid scintillator. The light yield of seven different liquid scintillator solvents (Toluene, Pseudocumene (PC), Linear Alkyl Benzene (LAB), Phenylcyclohexane (PCH), Phenyl-o-Xylyl Ethane (PXE) and two grades of Di-Isopropyl Napthalene (DIN)) are measured and modeled as a function of the concentration of the fluor and primary wavelength shifter 2,5-Diphenyloxazole (PPO). These results are compared to the light yield of several candidate quantum-dots samples in a standard toluene-based liquid scintillator. The next steps}

\begin{document}

\section{Introduction}\label{intro}
Liquid scintillator-based detectors are one of the primary technologies used in radiation detection and measurement. They are a popular choice for neutrino experiments because they scale well to large volumes. The addition of heavy elements to the scintillator increases the detectors sensitivity to neutrons with Cd or other capture reactions like solar neutrinos on In\cite{raghavan76}. With the measurement of neutrino mass and the increased interest in the search for neutrinoless double-beta decay to determine the Majorana nature of the neutrino,  interest in the addition of these candidate isotopes to scintillator has increased accordingly\cite{SNO+, KZ}. Unfortunately, the chemistry of introducing heavy elements into organic scintillators can be difficult and degrade the optical properties of the scintillator.

Quantum dots are nanocrystals. The size of a QD is inversely proportional to its band gap and therefore proportional to the wavelength of light it absorbs and emits. Because the size is well controlled during synthesis, the emission happens in a narrow peak around the characteristic wavelength set by the band gap. They are made of interesting elements like Cd, and In and many of the double-beta decay candidates like Cd, Se, Zn and Te. They can be suspended in organic liquids or water by the proper choice of organic coatings or ligands. This chemistry could be more robust than traditional methods. The combination of their optical, chemical and nuclear properties motivates the development of quantum-dot-doped liquid scintillator.

Liquid scintillators are composed of a organic solvent with secondary wavelength shifters. When ionizing particles traverse the scintillator, they mainly excite the solvent molecules. The molecules can de-excite through fluorescent light emitting processes or the non-radiative transfer of energy to another molecule. The solvent is not transparent to its own fluorescence light, so primary and secondary wavelength shifters are added to the solvent. When QDs are added to the scintillator cocktail, they act as a secondary wavelength shifter.

In previous work, we performed preliminary studies on the light yield of a quantum-dot-doped liquid scintillator\cite{qd1}. This work was followed up by a more detailed study of the chemistry and energy transfer in quantum-dot-doped scintillator including preliminary purification studies\cite{qd2}. This is part of a greater effort to develop a directional liquid scintillator detector. By narrowing the scintillation emission spectrum, the scintillation light is more efficiently separated from the directional Cherenkov improving the detectors ability to reconstruct a particle's direction\cite{direction}.

In this work, we measure the light yields of simple scintillator cocktails made from the six most common solvents to determine the best solvent for a quantum-dot-doped liquid scintillator. We use 2,5-Diphenyloxazole (PPO) as the primary wavelength shifter.  The details of these solvents and the PPO used are listed in Table~ \ref{solvents_table}. We study the light yield as a function of PPO concentration. These results are then compared to the light yields of a selection of currently available QDs, see Table~\ref{quantum_dots_table}, disolved in a scintillator cocktail of toluene with 2~g/L of PPO. Additional measurements of the absorption and emission spectra of the QD samples are made to understand the observed reduced light yields.

\begin{table}
\caption{The solvents used in this paper, their names, abbreviations, corresponding IUPAC names, CAS numbers and information on the source of the chemicals. The primary wavelength shifter PPO is included for completeness. \label{solvents_table}}
 \begin{center}
    \begin{tabular}{llllll}
       Name & IUPAC Name & CAS Number & Source Information \\
       \hline\hline\\[-5px]
       Toluene & Methylbenzene & 108-88-03 & Sigma Aldrich \cite{sigmaaldrich} \\
               &               &           & (Chromasolv Plus) \\
       \hline
       Pseudocumene & 1,2,4-Trimethylbenzene & 95-63-6 & Aldrich (98~$\%$) \cite{sigmaaldrich} \\
       (PC)         &                        &         &                   \\
       \hline
       Linear Alkyl Benzene & various chain lengths & 67774-74-7 & Cepsa/Petresa \cite{cepsa} \\
       (LAB)                           &                       &            & PETRELAB\textsuperscript{\textregistered} 550-Q \\
       \hline
       Phenyl-o-Xylylethane & 1,2-Dimethyl-4- & 6196-95-8 & Dixie Chemical \\
       (PXE)                & (1-Phenyl-Ethyl)-Benzene &  & Company \cite{dixiechemical} \\
       \hline
       Di-Isopropylnapthalene & Isomer mixture & 38640-62-9 & courtesy of \\
       (DIN)                  &                &            & PerkinElmer \cite{perkinelmer} \\
       \hline
       Di-Isopropylnapthalene, & Isomer mixture & 38640-62-9 & courtesy of \\ 
       high purity (DIN HP) & & & PerkinElmer \cite{perkinelmer} \\
       \hline
       Phenylcyclohexane & Cyclohexylbenzene & 827-52-1 & Aldrich ($\geq$ 97~$\%$) \cite{sigmaaldrich} \\
       (PCH)             &                   &          &                          \\
       \hline\hline
       Diphenyloxazole & 2,5-Diphenyloxazole & 92-71-7 & Aldrich (99~$\%$) \cite{sigmaaldrich} \\
       (PPO)               &     &         & suitable for scint. \\[5px] 
      \hline\hline
    \end{tabular}
  \end{center}
\end{table}

\begin{table}
\caption{Quantum-Dot composition and source information. \label{quantum_dots_table}}
 \begin{center}
    \begin{tabular}{llllll}
       Composition  & Source Information \\
       (core/shell)  & & \\
       \hline\hline\\[-5px]
         CdS &   NN-Labs-CS360\cite{nnLabs} \\
       \hline
         CdS &    NN-Labs-CS400\cite{nnLabs}   \\
       \hline
         CdS &  MKN-CdS-T360\cite{mkNano} \\
       \hline
         CdS &  MKN-CdS-T380\cite{mkNano}   \\
       \hline
        CdS &  MKN-CdS-T400\cite{mkNano} \\
             \hline
      CdS/ZnS  &  Ocean NanoTech QZR-400-0010\cite{oceanNanotech} \\
       \hline
      CdS/ZnS & Ocean NanoTech QZR-425-0010 \cite{oceanNanotech} \\
      \hline
        CdSeS/ZnS &   Crystalplex NC-450-A \cite{crystalplex} \\
      \\[5px] 
      \hline
 \hline
    \end{tabular}
  \end{center}
\end{table}

\section{Experimental Methods}
\label{exp_sec}

The light yield of the candidate liquid scintillators was measured using the charge collected by a photomultiplier tube (PMT) when the samples were illuminated with a gamma source. Further measurements of the emission and absorption spectrum of the QD scintillator were performed. The details of the experimental methods including the sample handling are described in the following sections.

\subsection{Light Yield Measurements}
\label{ly_setup_sec}

The setup for the light yield measurements is illustrated in Figure \ref{setup_fig}. A (1~cm$\times$1~cm$\times$3.5~cm) UV-transparent quartz cell (Starna 21-Q-10 \cite{starnacells}) holds the scintillator mixtures. The cell is coupled to the PMT with transparent silicone optical grease (Saint Gobain BC-630) with a similar index of refraction to the cell and PMT glass so that light losses due to reflection are minimized. The quartz cell is then secured in a reflective white Teflon block to further increase the light collection efficiency. The scintillator is excited by a ${^{137}}$Cs source with an activity of $\approx$1~$\mu$Ci (=37~kBq). The source is attached to the outside of the Teflon block. Isotope ${^{137}}$Cs undergoes beta decay with the subsequent emission of a single 662 keV gamma. The emitted gamma rays enter the quartz cell, then typically Compton scatter with electrons, which excite the scintillator. Cherenkov light is also produced from electrons with energies above a given threshold as discussed in Section  \ref{qd_sec} . A fraction of the emitted optical photons hit the photocathode of a Hamamatsu R1828-01 PMT \cite{hamamatsupmt}. A Hamamatsu E2979-500 base \cite{hamamatsubase} is used, and the high voltage of -1675~V is provided by a LeCroy 1454 high voltage system. The charge incident on the photocathode produce pulses which are recorded by an AlazarTech ATS9870 PCI waveform digitizer \cite{alazartech}. 

%Was there supposed to be something here?
%\subsubsection{Photomultiplier Tube Calibration}

In order to ensure reproducible and accurate measurements, a sample handling procedure is defined. In particular, contamination of the samples with dust, residual chemicals and oxygen is avoided. The bottles that store the samples are rinsed with isopropanol, cyclohexane, dried and finally rinsed with the solvent. In addition, the bottles are sealed under N$_{2}$ to reduce the amount of oxygen in contact with the sample. Oxygen interferes with the light production processes (oxygen quenching \cite{birks64}) and leads to a loss of light output. The lids of the bottles are sealed with Teflon tape and electric tape to further limit the amount of oxygen entering the bottle. 

The quartz cell is also cleaned thoroughly when samples are changed. The optical grease on the outside is removed with ethanol and on the inside it is washed with isopropanol then dried with N$_{2}$ twice, washed with cyclohexane, and again dried with N$_{2}$. The amount of liquid scintillator sample in the cell is kept constant to avoid effects due to differences in light collection. Before each measurement, a pipet is inserted into the liquid to slowly purge the samples with nitrogen for about 10 minutes to actively remove remaining oxygen. 

The coupling of the cell to the PMT is done consistently, and the PMT is aligned in the same way in order to avoid effects due to the earth's magnetic field. Section \ref{ly_analysis_sec} presents the studies that determine measurement reproducibility.  The quartz cell containing the liquid scintillator sample is placed in the dark box setup as described in the previous section, and a measurement is performed for a duration of 30 minutes.
 
\begin{figure}[tbh]
        \begin{center}
        \includegraphics[scale=0.6]{graphs/Diagram_Darkbox.png}
\caption[]{Schematic view of the light yield measurement setup. \label{setup_fig}}
\end{center}
\end{figure}

\subsection{Absorption and Emission Spectrum Measurements}

A Perkin Elmer Lambda 25 UV/Vis is used for measuring the emission spectrum and a Shimadzu UV3101 for measuring absorption spectrum of the QD doped liquid scintillators. Both instruments utilize a double-beam setup. White light from deuterium UV lamp and a halogen lamp produce the beams. \cite{|}

To measure the absorption spectrum of each quantum-dot sample, the following procedure is executed. A (1~cm$\times$1~cm$\times$3.5~cm) UV-transparent quartz cell (Starna 21-Q-10 \cite{starnacells}) is rinsed with toluene to dissolve any residue on the wall. Next, the cell is soaked in hydrochloric acid for 5 minutes to clean the cell and destroy any remaining QD. To wash away the hydrochloric acid and the residue QD constituents, the quartz cell is rinsed with distilled water and is then subjected to three iterations of liquid isopropyl alcohol evaporated by nitrogen gas. Since all of the QD samples are dissolved in toluene, a background absorption spectrum analysis is required to isolate the quantum-dot spectrum.  For absorption measurements, a background scan is performed with a vial filled with pure toluene. The toluene is then replaced with the QD sample, and the absorption scan is performed.  The emission spectrum does not require a background toluene scan.


\section{Results}
\label{analysis_results_sec}
This section presents the analysis methodology as well as the main results, namely the light yield measurements as a function of PPO concentrations for each of the solvents listed in Table \ref{solvents_table} and for the different QDs listed in Table \ref{quantum_dots_table} dissolved in toluene with 2~g/L of PPO. The emission and absorption spectrum of the QD scintillators are also measured.
 
\subsection{Light Yield Analysis}\label{ly_analysis_sec}
\begin{figure}[tbh]
        \begin{center}
        \includegraphics[scale=0.6] {graphs/14August2013_toluene_5gL_PPO_processed_waveform_paper.pdf}
\caption[]{Example waveform recorded by the AlazarTech digitizer card. For each waveform, 192 samples are taken with a rate of 1Gs/s. The blue, dashed line is the baseline which is calculated for each waveform. The red, dotted lines show the start and stop time for the main pulse in this waveform. Integration of the baseline subtracted waveform between start and stop yields 520. The trigger threshold is set at voltage channel 116. \label{waveform_fig}}
\end{center}
\end{figure}

The PMT pulse from the collected scintillation light is digitized and written out as a waveform. An example single waveform is shown in Figure \ref{waveform_fig}. Each waveform consists of 512 voltage samples. At a rate of 1~Gs/s one waveform thus spans 512~ns. The voltage range is [2~V,-2~V] with a resolution of 8-bit, 256 channels. The sampling is therefore done with voltage channels of size $\approx$ 0.0156~V. The amount of light collected is proportional to the the charge of a given pulse. Thus, the integral over the voltage pulse has to be calculated. To do this, the following steps are carried out for each waveform: 
\begin{enumerate}
\item Baseline Calculation: At the beginning of each waveform, 100 samples are averaged to get the baseline value. 
\item Pulse Finder: Pulses are defined as a consecutive series of samples below the calculated baseline value.
\item Pulse Times: For each pulse the start time is defined as the time where the waveform falls below the baseline value. The stop time is set once the pulse reaches this value again.  
\item Pulse Charge: The integral of the baseline-subtracted waveform between the start and stop times.  
\end{enumerate}

The duration of each run is around thirty minutes and during this time on the order of $3\cdot10^5$ waveforms are recorded. Each pulse corresponds to one gamma ray interaction within the liquid scintillator. The amount of light collected by the PMT is dependent on the scattering angle of the electron that is produced by Compton scattering of the 0.662 MeV gamma ray. The result of the charge calculation is placed in a histogram, Figure \ref{ChargeSpec_fig}. Since the Compton effect is the dominant mechanism, this histogram is a Compton spectrum. The events where the Compton scattered electron obtained close to the maximal energy, 478~keV for the 662~keV gammas from $^{137}$Cs, is associated with events at the right end of this spectrum. 

Multiple pulses are found for each waveform, including baseline fluctuations. The small fake pulses from baseline fluctuations did not affect analysis. In order to compare different scintillator mixtures, the charge where the number of events dropped to half of the Compton edge maximum is compared, Figure \ref{ChargeSpec_fig}. This quantity is proportional to the scintillator light yield. For practical reasons and because this study only looks at relative light yield, results are presented in arbitrary charge units instead of real charge units. The uncertainty in the extraction of this quantity is negligible compared to the systematic uncertainties due to the handling of the samples especially the oxygen concentration.

The error due to differences in the procedure of coupling the scintillator cell to the PMT glass is studied with a LAB sample with 5~g/L PPO in a sealed cell.  The coupling is renewed and data taken 10 times. The RMS divided by the sample mean for the trials is 0.30~\%. This number is used as a rough estimate for the relative error of a single measurement due to differences in the coupling. 

Increased oxygen concentration is known to decrease scintillator light yield. A 24-day old sample of DIN HP with 5~g/L PPO is purged with nitrogen bubbles for 10 minutes as in the standard measurement procedure and is measured. Then the sample is opened and left in contact with oxygen for 15 minutes. It is then re-measured without nitrogen purging. The charge value dropped by 2.04~\%. It is left open for an additional 3 hours and the measurement repeated without purging. The relative difference between the first and the last measurement is about 6.92~\%. This gives us the range in the uncertainty due to oxygen concentration.

In order to estimate the total error of a single light yield measurement, the same sample with the full sample protocol, as described in Section \ref{ly_setup_sec}, is done repeatedly. For these measurements an older sample, 7 weeks after mixing, of DIN HP is used as well as a standard DIN sample, 4 weeks old. For the older samples, 10 minutes of nitrogen purging is not sufficient to remove oxygen since the light yield improves over the trials.  The RMS for these data sets is 3.70~\% and is used the uncertainty on a single light yield measurement.

\begin{figure}[tbh]
        \begin{center}
        \includegraphics[scale=0.6]{graphs/toluene_5gL_PPO_Floats_final.pdf}
\caption[]{Example charge spectrum. To compare different scintillators, the charge at a characteristic point close to the true Compton edge is determined. \label{ChargeSpec_fig}}
 \end{center}
\end{figure}

Data is taken for each candidate solvent listed in Table \ref{solvents_table} at several concentrations of PPO. Figure~\ref{ChargeResults_fig} and Figure~\ref{Toluene_ChargeResults_fig} shows the solvent's characteristic charge at the Compton edge for the PPO concentrations of 0.5 g/L, 1 g/L, 2 g/L, 5 g/L, 10 g/L and 50 g/L.  The uncertainty on the data points is 3.70\%.

In general, the efficiency of energy transfer from solvent to the wavelength shifter increases with increasing PPO concentration. At the studied PPO concentrations, the energy transfers are predominantly non-radiative F\"orster energy transfers \cite{foerster48,foerster59}, but the emission and re-absorption of real photons or collision energy transfer and others \cite{dexter53} also contribute to the overall effective energy transfer rate. There also is the effect of self-quenching \cite{birks64} which becomes more important at higher PPO concentrations. In this process, two PPO molecules interact with each other and no light is emitted. We model the light yield $I(c_{PPO})$ as a function of PPO concentration $c_{PPO}$,
\begin{equation}
\label{fit_eq}
I(c_{PPO})=p_1 \cdot \frac{1}{1+p_2/c_{PPO}} \cdot \frac{1}{1+p_3\cdot c_{PPO}}
\end{equation} 
where $p_1$, $p_2$ and $p_3$ are three solvent specific parameters. $p_1$ characterizes the maximum light yield without self-quenching. $p_2$ is the effectiveness of energy transfer from solvent molecules to the wavelength shifter molecules. $p3$ models the effect of self-quenching. For details on the derivation of these types of equations see \cite{buck07,aberle11} and references therein. Table \ref{fitresult_table} contains the fit results for the different liquid scintillator solvent candidates.


As clearly stated in Table \ref{Toluene_ChargeResults_fig}, toluene produces the highest light yield around 2.2~g/L.  QD scintillators are doped with 2~g/L PPO for comparison with 2~g/L PPO in toluene. The light yield as a function of QD emission wavelength is plotted in Figure \ref{ly_QD}.

\begin{table}
\caption{Listed below are the fit results for each solvent's data set. The fit parameters and their errors are given and their respective $\chi^{2}$ values, and the probabilities to get a worse $\chi^{2}$ than the observed one. Since we have 6 data points and 3 parameters in the fit the number of degrees of freedom is 3. \label{fitresult_table}}

%Uncertainty on LY.
  \begin{center}
    \begin{tabular}{lllllll}
       Solvent & $p_1$ [a.u.] & $p_2$ [g/L] & $p_3$ [g/L] & $\chi^{2}$ & prob. \% & max. LY\\
       \hline\hline\\[-5px]
       Toluene  & $681 \pm 10$  & $0.09 \pm 0.014$  & $0.019 \pm 0.001$ & 0.55 & 91 & 627 (at 2.2~g/L)\\
       PC & 578 $\pm$ 47  & 2.09 $\pm$ 0.37  & 0.014 $\pm$ 0.004  & 5.30  & 15  & 422 (at 12.2~g/L)\\
       LAB & 525 $\pm$ 24  & 0.36 $\pm$ 0.01  & 0.013 $\pm$ 0.003  & 4.33  & 23 & 462 (at 5.3~g/L)\\
       PXE & 694 $\pm$ 25  & 0.48 $\pm$ 0.96  & 0.013 $\pm$ 0.002 & 2.55  & 47 & 596 (at 6.1~g/L) \\
       DIN & 668 $\pm$ 19  & 0.27 $\pm$ 0.01  & 0.007 $\pm$ 0.001  & 2.34  & 50 & 613 (at 6.2~g/L)  \\
       DIN HP & 673 $\pm$ 40 & 0.19 $\pm$ 0.08 & 0.005 $\pm$ 0.003  & 10.6  & 1 & 631 (at 6.1~g/L) \\
       PCH & 634 $\pm$ 29  & 0.34 $\pm$ 0.07  & 0.012 $\pm$ 0.003 & 6.11  & 11 & 562 (at 5.4 ~g/L) \\
      \hline \hline
    \end{tabular}
  \end{center}
\end{table}

\begin{figure}[tbh]
        \begin{center}
        \subfigure[PC]{\includegraphics[scale=0.39]{graphs/Light_Yield/PC_Light_Yield.png}}
        \subfigure[LAB]{\includegraphics[scale=0.39]{graphs/Light_Yield/LAB_Light_Yield.png}}
        \subfigure[PXE]{\includegraphics[scale=0.39]{graphs/Light_Yield/PXE_Light_Yield.png}}
        \subfigure[DIN]{\includegraphics[scale=0.39]{graphs/Light_Yield/DIN_Light_Yield.png}}
        \subfigure[DIN HP]{\includegraphics[scale=0.39]{graphs/Light_Yield/DIN_HP_Light_Yield.png}}
        \subfigure[PCH]{\includegraphics[scale=0.39]{graphs/Light_Yield/PCH_Light_Yield.png}}
\caption[]{Show the charge value at the Compton edge (proportional to the light yield) for each of the 6 concentrations of PPO for six of the liquid scintillator candidates. A relative error of $3.7~\%$ was used for each measurement. The red lines show the fit of equation \ref{fit_eq}. Note that the x-axis is logarithmic; the PPO concentrations span two orders of magnitude. %The rise of the LY at low concentrations is faster than the drop due to self-quenching at high concentrations.
  \label{ChargeResults_fig}}
        \end{center}
\end{figure}

\begin{figure}[tbh]
        \begin{center}
        \subfigure[Toluene]{\includegraphics[scale=0.60]{graphs/Light_Yield/Toluene2014_Light_Yield.png}}
\caption[]{Show the charge value at the Compton edge (proportional to the light yield) for each of the 6 concentrations of PPO for toluene. A relative error of $3.7~\%$ was used for each measurement. The red lines show the fit of equation \ref{fit_eq}. Note that the x-axis is logarithmic; the PPO concentrations span two orders of magnitude. %The rise of the LY at low concentrations is faster than the drop due to self-quenching at high concentrations.
  \label{Toluene_ChargeResults_fig}}
        \end{center}
\end{figure}

\begin{figure}[tbh]
        \begin{center}
        \includegraphics[scale=0.3] {graphs/Light_Yield/QDots_Light_Yield.png}
\caption[]{The light yield of quantum-dot doped toluene scintillators with 2~g/L of PPO versus emission wavelength of the QDs.\label{ly_QD}}
\end{center}
\end{figure}

\subsection{Emission and Aborption Spectrum of Quantum-Dot Doped Liquid Scintillators}

\subsubsection{Signal Analysis}
Various trials are conducted to test the output signal of both the Perkin Elmer Lambda 25 UV/Vis and the Shimadzu UV3101 spectrophotometers\cite{lindley13}. The error is described by the following equation:
 \ref{Absorption_equ}.

\begin{equation}
\label{Absorption_equ}
A(x)=\log_{10}\Big(\frac{I(x)}{I(0)}\Big)
\end{equation}
where  $I(0)$ is the intensity of the beam before transiting the quartz cell and $I(x)$ is the intensity after it passes through the cell. 

The errors for the Perkin and Shimadzu spectrophotometers are $A=\pm0.00075$ and $A=\pm0.002$ respectfully. These errors are in close agreement with the manufactures specifications\cite{?}.

\subsubsection{Absorption and Emission Results}\label{abs_emi_results}
The absorption and emission spectrum of the QDs were extensively probed, and the results are graphed in Figures \ref{abs_spec} and \ref{emi_spec} respectfully. As stated in Section \ref{qd_sec} and proved in \ref{ly_results}, the light yield of the QD doped scintillators is much lower than that of the binary mixture of toluene with PPO. This is due to the overlap in emission and absorption QD spectrum coupled with low quantum efficiency \cite{lindley14}. (I'm creating a figure that shows the overlap for emission and absorption for each quantum dot). The absorption spectrum of the QD doped liquid scintillators are narrower than that of toluene doped with 2 g/L PPO.
(How can we back this? Need to take an absorption and emission scan of toluene + 2g/L PPO.)
%Discuss the shift of emission spectrum (Measure Toluene with PPO emission Spectrum)

\begin{figure}[tbh]
        \begin{center}
        \subfigure[NN-Labs]{\includegraphics[scale=0.29]{graphs/Emission_Absorption/NN_Labs_Abs.png}}
        \subfigure[mkNano]{\includegraphics[scale=0.29]{graphs/Emission_Absorption/mkNano_Abs.png}}
        \subfigure[Ocean NanoTech]{\includegraphics[scale=0.29]{graphs/Emission_Absorption/Ocean_NanoTech_Abs.png}}
        \subfigure[Crystalplex]{\includegraphics[scale=0.29]{graphs/Emission_Absorption/Crystalplex_Abs.png}}
\caption[]{Absorption spectrum of the quantum-dots listed in Table \ref{quantum_dots_table} dissolved in toluene and grouped by manufacturer.}\label{abs_spec}
\end{center}
\end{figure}

\begin{figure}[tbh] 
        \begin{center}
        \subfigure[]{\includegraphics[scale=0.28]{graphs/Emission_Absorption/NNLabs_360nm_Emi.png}}
        \subfigure[]{\includegraphics[scale=0.28]{graphs/Emission_Absorption/NNLabs_400nm_Emi.png}}
        \subfigure[]{\includegraphics[scale=0.28]{graphs/Emission_Absorption/mkNano_360nm_Emi.png}}
        \subfigure[]{\includegraphics[scale=0.28]{graphs/Emission_Absorption/mkNano_380nm_Emi.png}}
        \subfigure[]{\includegraphics[scale=0.28]{graphs/Emission_Absorption/mkNano_400nm_Emi.png}}
        \subfigure[]{\includegraphics[scale=0.28]{graphs/Emission_Absorption/Ocean_NanoTech_400nm_Emi.png}}
        \subfigure[]{\includegraphics[scale=0.28]{graphs/Emission_Absorption/Ocean_NanoTech_425nm_Emi.png}}
        \subfigure[]{\includegraphics[scale=0.28]{graphs/Emission_Absorption/Crystalplex_Emi.png}}
\caption[]{Emission spectrum of the quantum-dots listed in Table \ref{quantum_dots_table} dissolved in toluene with and without 2~g/L of PPO added.\label{emi_spec}}
\end{center}
\end{figure}


\section{Conclusion}
Detection of low energy events in liquid scintillators require high light yield and preservation of Cherenkov radiation. This study  successfully measures and models the light yield of seven scintillator liquids as a function of wavelength-shifter concentration in order to find the maximum light yield of each candidate. Furthermore, an extensive study on the effects of quantum-dot doping of toluene with 2 g/L of PPO in solution are analyzed. QDs reduce the width and control the location of the absorption spectrum peak. This allows the quantum efficiency of the photocathode to be matched and minimized the amount of Cherenkov radiation absorbed by the liquid scintillator.

Regrettably, QDs significantly reduce the light yield of scintillators. With improved technology in quantum dot production, QDs with better characteristics such as shorter emission wavelengths are expected to become available in the near future. These QDs should be able to improve the light yield of quantum dot-doped scintillator. 


%Add references for qdots 1 +2, christian's new paper.
%When charged particles propagate through a scintillator, electrons in the $\pi$-bonds of the aromatic rings are readily excited \cite{birks64}. The absorption and emission spectra of a single-molecule based scintillator overlap. In order to prevent degradation of efficiency due to the reabsorption of scintillation light by the scintillator, the liquid scintillator (a solute) was mixed with a wavelength shifter (a solvent). The wavelength shifter absorbs the energy from the excited electron and then releases light of a longer wavelength to which the scintillator is more transparent. Light yield, the number of photons produced per deposited energy, is a vital property of the scintillator because it relates directly to the energy resolution of a scintillator-based detector. Optimizing the energy transfer between solvent and solute molecules maximizes the light yield.

%Unfortunately, doping liquid scintillators with QDs comes at the cost of lower light yields as shown in Section \ref{ly_results} and discussed Section \ref{abs_emi_results}. Fortunately, this can be partially countered by choosing the emission peak to match the optimal wavelength for the quantum efficiency of the photomultiplier tube (PMT)\cite{hamamatsupmt}. The QDs of various emission and absorption spectra that undergo study for suitability are listed in Table \ref{quantum_dots_table}.
 



\begin{thebibliography}{35}
\bibitem{raghavan76}
R.S.~Raghavan, \emph{Inverse $\beta$ Decay of $^{115}$In$\rightarrow$ $^{115}$Sn$^{*}$: A New Possibility for Detecting Solar Neutrinos from the Proton-Proton Reaction}, \emph{Phys. Rev. Lett.} {\bf37} (1976) 259.

\bibitem{majorana37}
E.~Majorana, \emph{Teoria simmetrica dell'elettrone e del positrone}, \emph{Il Nuovo Cimento} {\bf 14} (1937) 171. 

\bibitem{furry39}
W.H.~Furry, \emph{On Transition Probabilities In Double Beta-Disintegration}, \emph{Phys. Rev.} {\bf 56} (1939) 1184. 

\bibitem{Zuber}
K.~Zuber, \emph{Neutrino Physics}, Taylor and Francis Group, 2004.

\bibitem{tabledb95}
V.I.~Tretyak, Y.G.~Zdesenko, \emph{Tables of double beta decay data}, \emph{Atomic Data and Nuclear Data Tables} {\bf 61} (1995) 43. 

\bibitem{elliott02}
S.R.~Elliott and P.~Vogel, \emph{Double Beta Decay}, \href{http://www.annualreviews.org/doi/abs/10.1146/annurev.nucl.52.050102.090641}{\emph{Annu. Rev. Nucl. Part. S.} {\bf 52} (2002) 115}.

\bibitem{barabash11}
A.~Barabash, \emph{Double beta decay experiments}, Phys. Part. Nucl. {\bf 42} (2011) 613 [\href{http://arxiv.org/abs/1107.5663}{arXiv:1107.5663}].

\bibitem{schwingenheuer13}
B.~Schwingenheuer, \emph{Status and prospects of searches for neutrinoless double beta decay}, \emph{Ann. Phys.} {\bf 525} (2013) 269.

\bibitem{cremonesi13}
O.~Cremonesi and M.~Pavan, \emph{Challenges in Double Beta Decay}, \href{http://arxiv.org/abs/1310.4692}{arXiv:1310.4692}.

\bibitem{schechter82}
J.~Schechter and J.W.F.~Valle, \emph{Neutrinoless double-$\beta$ decay in SU(2)$\times$U(1) theories}, \emph{Phys. Rev. D} {\bf 25} (1982) 2951.  

\bibitem{klapdor04}
H.V.~Klapdor-Kleingrothaus, I.V.~Krivosheina, A.~Dietz and O.~Chkvorets, \emph{Search for neutrinoless double beta decay with enriched 76Ge in Gran Sasso 1990-2003}, \emph{Phys. Lett. B} {\bf 586} (2004) 198.

\bibitem{klapdor06}
H.V.~Klapdor-Kleingrothaus and I.~Krivosheina, \emph{The evidence for the observation of 0$\nu\beta\beta$ decay: the identification of 0$\nu\beta\beta$ events from the full spectra}, \emph{Mod. Phys. Lett. A} {\bf 21} (2006) 1547 (2006).

\bibitem{heidelbergmoscow01}
H.V.~Klapdor-Kleingrothaus et al. (Heidelberg-Moscow Collaboration), \emph{Latest results from the HEIDELBERG-MOSCOW double beta decay experiment}, \emph{Eur. Phys. J. A} {\bf 12} (2001) 147.

\bibitem{gerda13}
M.~Agostini et al. (GERDA collaboration), \emph{Results on Neutrinoless Double-$\beta$ Decay of $^{76}$Ge from Phase I of the GERDA Experiment}, \emph{Phys. Rev. Lett.} {\bf 111} (2013) 122503. 

\bibitem{kamlandzen13}
A.~Gando et al. (KamLAND-Zen Collaboration), \emph{Limit on Neutrinoless $\beta\beta$ Decay of $^{136}$Xe from the First Phase of KamLAND-Zen and Comparison with the Positive Claim in $^{76}$Ge}, \emph{Phys. Rev. Lett.} {\bf 110} (2013) 062502.

\bibitem{snoplus10}
C.~Kraus, S.J.M.~Peeters, \emph{The rich neutrino programme of the SNO+ experiment}, \emph{Prog. Part. Nucl. Phys.} {\bf 64} (2010) 273.

\bibitem{hartnell12}
J.~Hartnell for the SNO+ collaboration, \emph{Neutrinoless Double Beta Decay with SNO+}, \emph{J. Phys. Conf. Ser.} {\bf 375} (2012) 042015.

\bibitem{birks64}
J.B.~Birks, \emph{The Theory and Practice of Scintillation Counting}, Pergamon Press, 1964.

\bibitem{buck15} 
  C.~Buck, B.~Gramlich and S.~Wagner,
  \emph{Light propagation and fluorescence quantum yields in liquid scintillators}, \jinst{10}{2015}{P09007}
  JINST {\bf 10}, no. 09, P09007 (2015)
  %[arXiv:1509.02327 [physics.ins-det]].
  %%CITATION = ARXIV:1509.02327;%%

\bibitem{lindley12} 
  L.~Winslow and R.~Simpson,
  `\emph{Characterizing Quantum-Dot-Doped Liquid Scintillator for Applications to Neutrino Detectors}, \jinst{7}{2012}{P07010}
  %[arXiv:1202.4733 [physics.ins-det]].
  %%CITATION = ARXIV:1202.4733;%%
  %5 citations counted in INSPIRE as of 02 Oct 2015

\bibitem{lindley14}
C.~Aberle, J.J.~Li, S.~Weiss and L.~Winslow, \emph{Measuring directionality in double-beta decay and neutrino interactions with kiloton-scale scintillation detectors }, \jinst{9}{2014}{P06012}.

\bibitem{lindley13}
C.~Aberle, J.J.~Li, S.~Weiss and L.~Winslow, \emph{Optical properties of quantum-dot-doped liquid scintillators}, \jinst{8}{2013}{P10015}.

\bibitem{sigmaaldrich}
http://www.sigmaaldrich.com (2014).

\bibitem{cepsa}
http://www.cepsa.com (2014). 

\bibitem{dixiechemical}
http://www.dixiechemical.com (2014).

\bibitem{perkinelmer}
http://www.perkinelmer.com (2014).

\bibitem{nnLabs}
http://www.nn-labs.com (2014).

\bibitem{mkNano}
http://mknano.com (2014).

 \bibitem{oceanNanotech}
 http://www.oceannanotech.com (2014).
 
 \bibitem{crystalplex}
 http://www.crystalplex.com (2014).

\bibitem{starnacells}
http://www.starnacells.com (2014).  

\bibitem{foerster48}
T.~F\"orster, \emph{Zwischenmolekulare Energiewanderung und Fluoreszenz}, \emph{Ann. Phys.} {\bf 2} (1948) 55.

\bibitem{foerster59}
T.~F\"orster, \emph{Transfer Mechanisms of electronic excitation}, \emph{Discuss. Faraday Soc.} {\bf 27} (1959) 7.

\bibitem{dexter53}
D.L.~Dexter, \emph{A theory of sensitized luminescence in solids}, \emph{J. Chem. Phys.} {\bf 21} (1953) 836.

\bibitem{buck07}
C.~Buck, F.X.~Hartmann, D.~Motta, S.~Sch\"onert, \emph{Energy transfer and light yield properties of a new highly loaded indium(III) $\beta$-diketonate organic scintillator system}, \emph{Chem. Phys. Lett.} {\bf 435} (2007) 252. 

\bibitem{aberle11}
C.~Aberle, C.~Buck, F.X.~Hartmann and S.~Sch\"onert, \emph{Light yield and energy transfer in a new Gd-loaded liquid scintillator}, \emph{Chem. Phys. Lett.} {\bf 516} (2011) 257.  


\bibitem{hamamatsupmt}
Hamamatsu Photonics K.K., \emph{Photomultiplier Tubes R1828-01, R2059} (data sheet), accessed March, 2014: http://www.hamamatsu.com/resources/pdf/etd/R1828-01\_R2059\_TPMH1259E04.pdf.

\bibitem{hamamatsubase}
Hamamatsu Photonics K.K., \emph{D-Type Socket Assemblies} (data sheet), accessed March, 2014: http://www.hamamatsu.com/resources/pdf/etd/PMT\_92-103\_e.pdf.

\bibitem{alazartech}
http://www.alazartech.com (2014).

\end{thebibliography}
\end{document}

%Charge figure 2 and 3, and table 3 with Chi-Square Fit
%Should the section on oxygen's afffect on LY be included
%Should the numbers in the uncertainties be changed?