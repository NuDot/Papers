\section{Event Topology and Spherical Harmonics Analysis}
\label{sec:topology_and_harmonics}

In this section we provide details about the spherical harmonics analysis to separate 0\nbb-decay signal events from backgrounds. 
We show that spherical harmonics decomposition of the PE hits on the detector surface is sensitive to the event topology. In particular
we focus on the differences between two-track (0\nbb-decay events) and single-track (\B~events) topologies.

For the purpose of illustration of the spherical harmonics analysis concept we first consider three distinct topologies: 
two electrons produced back-to-back at an 180$^{\circ}$ angle, two electrons at a 90$^{\circ}$ angle, and a single electron.
Figure~\ref{fig:ThreeTopologies_Display_5MeV} shows an idealized simulation of these three topologies for the total electrons 
energy of 10~MeV. In order to emphasize ring patterns formed by cherenkov photons the electron multiple scattering process is turned off
in this idealized simulation. Here the single-track event represents an idealized \B~event topology and the two-track events represent
two special cases of an idealized 0\nbb-decay topology.


\begin{figure*}[h]
  \centering
%  \begin{tabular}{c c c}
  \includegraphics[width=0.45\textwidth]{hDisplay_topology180_5MeV.pdf}
  \includegraphics[width=0.45\textwidth]{hDisplay_topology90_5MeV.pdf}
  \includegraphics[width=0.45\textwidth]{hDisplay_1el_10MeV.pdf}
%  \end{tabular}
  \includegraphics[width=0.45\textwidth]{hMultipleMoment_CHELight_VtxSmear0cm_VtxShiftX0cm_999p9ns_5p0MeV_center_NoMultScat.pdf}
  \caption{\emph{Top row:} Idealized event displays for the three representative event topologies: two back-to-back
    5~MeV electrons (\emph{left}), two 5~MeV electrons at 90$^{\circ}$ angle (\emph{middle}), and a single 10~MeV electron
    (\emph{center}). Multiple scattering is turned off in the simulation to emphasize the difference in the mutual orientation of
    cherenkov rings for the three topologies. For the illustration purposes 100\% QE is applied to cherenkov photons (triangles)
    and the default QE is applied to scintillation photons (dots). All electrons originate at the center of the
    detector. One typical event is shown for each topology.
    \emph{Bottom pannel:} Normalized power spectrum $S_l$ calculated for distribution of cherenkov photons only. The three
    topologies are compared: two back-to-back 5~MeV electrons (\emph{solid red line}), two 5~MeV electrons at 90$^{\circ}$ angle
    (\emph{dotted magenta line}), and a single 10~MeV electron (\emph{dashed blue line}).}
  \label{fig:ThreeTopologies_Display_5MeV}
\end{figure*}



The central strategy of the spherical harmonics analysis is to construct rotationally invariant variables that can be used to separate 
different event topologies. To this end, let the  function $f(\theta,\phi)$ represent the PE distribution on the detector surface. The function $f(\theta,\phi)$ can be decomposed into a sum of spherical harmonics:

\begin{eqnarray}
\label{eq1}
f(\theta,\phi) = \sum_{l=0}^{\infty} \sum_{m=-l}^{l} f_{lm} Y_{lm}(\theta,\phi),
\end{eqnarray}

where $Y_{lm}$ are Laplace's spherical harmonics defined in a real-value basis using Legendre polynomials $P_l$:

\begin{eqnarray}
\label{eq2}
Y_{lm} = \left\{
  \begin{array}{@{}ll@{}}
    \sqrt{2}N_{lm}P_l^m(cos\theta)cos~m\phi, & \text{if}\ m>0 \\
    N_{lm} = \sqrt{\frac{(2l+1)}{4\pi} \frac{(l-m)!}{(l+m)!}}, & \text{if}\ m=0 \\
    \sqrt{2}N_{l|m|}P_l^|m|(cos\theta)sin~|m|\phi, & \text{if}\ m<0
  \end{array}\right.
\end{eqnarray}

where the coefficients $f_{lm}$ are defined as
 
\begin{eqnarray}
\label{eq3}
f_{lm} = \int_{0}^{2\pi} d\phi \int_0^{\pi} d\theta sin\theta f(\theta,\phi) Y_{lm}(\theta,\phi).
\end{eqnarray}

Equation~\ref{eq4} defines the power spectrum of $f(\theta,\phi)$ in the spherical harmonics representation, $s_l$, where $l$ is a multiple moment. The power spectrum, $s_l$, is invariant under rotation. It is unique to each of the functions $f_i(\theta,\phi)$, $i=$1,2,3..., which can not be transformed into each other by rotation.

\begin{eqnarray}
\label{eq4}
s_l = \sum_{m=-l}^{m=l} |f_{lm}|^2
\end{eqnarray}

The event topology in a spherical detector determines the distribution of the PE's on the detector sphere, and, therefore, a set of $s_l$'s. These values can serve as a quantitative figure of merit for different event topologies. The rotation invariance of $s_l$'s ensures that this figure of merit does not depend on the orientation of the event with respect to the chosen coordinate frame.

Sum of $s_l$'s over all multiple moments equals to the L2 norm of the function $f(\theta,\phi)$:

\begin{eqnarray}
\label{eq5}
\sum_{l=0}^{\infty} s_l = \int_{\Omega} |f(\theta,\phi)|^2 d\Omega.
\end{eqnarray}

Therefore, the normalized power spectrum,

\begin{eqnarray}
\label{eq6}
S_l = \frac{s_l}{\sum_{l=0}^{\infty} s_l} =  \frac{s_l}{\int_{\Omega} |f(\theta,\phi)|^2 d\Omega},
\end{eqnarray}

can be used to compare shapes of various functions $f(\theta,\phi)$ with different normalization. The total number of PEs detected on the detector sphere fluctuates from event to event, therefore, in all of the following we use the normalized power $S_l$.

Figure~\ref{fig:ThreeTopologies_Display_5MeV}(\emph(bottom right)) compares the normalized power spectra for the three representative event topologies.
There is sufficien separation between these three idealized event topologies.

In reality, cherenkov rings become very fuzzy due to elecron multiple scattering at energies relevant to 0\nbb-decay. In most cases $\sim$1~MeV 
electrons produce randomly shaped clusters of Cherenkov photons around the direction of the electron track. Examples of such Cherenkov clusters, 
for the same three representative topologies, but with the multiple scattering process included in the simulation and the total kinetic energy of 
the electrons being equal to the Q-value of the \Te, are shown in Fig.~\ref{fig:ThreeTopologies_Display}.


\begin{figure*}[h]
  \centering
%  \begin{tabular}{c c c}
  \includegraphics[width=0.45\textwidth]{hDisplay_topology180_2p529MeVTot}
  \includegraphics[width=0.45\textwidth]{hDisplay_topology90_2p529MeVTot}
  \includegraphics[width=0.45\textwidth]{hDisplay_1el_2p529MeV}
%  \end{tabular}
  \includegraphics[width=0.45\textwidth]{hMultipleMoment_cheLight_VtxSmear0cm_VtxShiftX0cm_999p9ns_center.pdf}
  \caption{\emph{Top row:} Event displays for the three representative event topologies with the electron energies relevant 
    for the {\Te} 0{\nbb}-decay: two back-to-back 1.26~MeV electrons (\emph{left}), two 1.26~MeV electrons at 90$^{\circ}$ angle 
    (\emph{middle}), and a single 2.53~MeV electron (\emph{center}). Multiple scattering is now properly included in the simulation
    For the illustration purposes 100\% QE is applied to cherenkov photons (triangles) and the default QE is applied to scintillation 
    photons (dots). All electrons originate at the center of the detector. One typical event is shown for each topology.
    \emph{Bottom pannel:} Normalized power spectrum $S_l$ calculated for distribution of cherenkov photons only. The three
    topologies are compared: two back-to-back 1.26~MeV electrons (\emph{solid red line}), two 1.26~MeV electrons at 90$^{\circ}$ angle
    (\emph{dotted magenta line}), and a single 2.53~MeV electron (\emph{dashed blue line}).}
  \label{fig:ThreeTopologies_Display}
\end{figure*}


The difference in the Cherenkov photon distribution for the three representative topologies is now not as dramatic as in the idealized 
simulation of 10~MeV events. However, a guess about the event topology still can be made by comparing photon distributions in different 
segments of the detector sphere. Spherical harmonics decomposotion is a natural tool for making such comparison between the segments. 
As shown in Fig.~\ref{fig:ThreeTopologies_Display_5MeV}(\emph(bottom right)) spherical harmonics power spectrum still provides noticable 
separation between all three event topologies.


More realistic examples of \Te~ 0\nbb~and \B events simulated at the center of the detector are shown in Fig.~\ref{fig:Display_Te130}. Early photoelectrons (PEs), defined as those PEs from Cherenkov and scintillation light within 33.5~ns of the start of the event, are shown. In this example, the default QE is applied.  In this more realistic example, the uniformly distributed scintillation light makes it more difficult to visually distinguish the event topology. Nevertheless, we show that there is still a sufficient difference in the spatial distribution of the early PEs to separate two-track and single-track events using an analysis of spherical harmonics.

\begin{figure*}[h]
  \centering
  \includegraphics[width=0.45\textwidth]{hDisplay_Te130_evt124_e1257_e1270_cos-0908}
  \includegraphics[width=0.45\textwidth]{hDisplay_Te130_evt131_e1264_e1263_cos-0029}
  \includegraphics[width=0.45\textwidth]{hDisplay_Te130_evt352_e1186_e1340_cos0888}
  \includegraphics[width=0.45\textwidth]{hDisplay_1el_2p529MeV_33p5ns}
  \includegraphics[width=0.8\textwidth]{hMultipleMomentSignal_allLight_VtxSmear0cm_VtxShiftX0cm_33p5ns_center.pdf} 
  \caption{ (\emph{Top and middle rows:} Event display examples for {\Te} 0{\nbb}-decay signal and {\B} background events.
    The default QE and the time cut of 33.5~ns are now applied to cherenkov (\emph{triangles} and scintillation (\emph{crosses}) 
    photons. For the {\Te} 0{\nbb}-decay signal three representative events are shown each closely matching on of the three
    topologies. A typical single electron event is shown for the {\B} background.
    \emph{Top left:} $^{130}$Te 0{\nbb}-decay back-to-back electrons: $E_1$=1.257~MeV, $E_2$=1.270~MeV, 
    cos($\theta$)=-0.908. \emph{Top right:} $^{130}$Te 0{\nbb}-decay electrons at $\sim$90$^{\circ}$: $E_1$=1.264~MeV, $E_2$=1.263~MeV,
    cos($\theta$)=-0.029. \emph{Bottom left:} $^{130}$Te 0{\nbb}-decay electrons at $\sim$0$^{\circ}$: $E_1$=1.186~MeV, $E_2$=1.340~MeV,
    cos($\theta$)=0.888. \emph{Bottom right:} 2.529~MeV single electron. In all events electrons originate at the center of the detector.
    \emph{Bottom pannel:} Normalized power spectrum $S_l$ calculated for distribution of all PE after the 33.5~ns time cut. 
    {\Te} 0{\nbb}-decay signal (\emph{solid red line}) and {\B} background (\emph{dashed blue line}) topologies are compared.}
\label{fig:Te130_Display}
\end{figure*}


\subsection{Topology of 0\nbb-decay and \B~Events}
\label{subsec:topology}
%Signature of the 0\nbb-decay is two electrons with total kinetic energy equal to the isotope Q-value (e.g., 2.529~MeV for $\Te$). Therefore all \B-decay searches are designed to look for an excess of events in the energy spectrum around Q-value over the predicted number of background events. Backgrounds suc0\nbbh as \B solar neutrinos have distinct event topology that can be used to improve experimental sensitivity to the the 0\nbb-decay signal.



%For all isotopes considered for searches of 0\nbb-decay, the electrons emitted by the process with energies around the Q-value will be above Cherenkov threshold in liquid scintillators.  Each electron above the threshold will produce a fuzzy ring of Cherenkov light at the detector surface. The fuzziness of the ring depends on electron scattering. In most cases Cherenkov rings from low energy electrons degrade to randomly shaped clusters of Cherenkov photons around the direction of the electron track. 

%A large fraction of 0\nbb-decay events will have two Cherenkov clusters~\footnote{Only one Cherenkov cluster is produced when either the angle between the two 0\nbb-decay electrons is too small or when the energy splits between the electrons in such a way that one electron falls below the Cherenkov threshold.} as opposed to one cluster from \B events. Therefore, accurately distinguishing between a 0\nbb-decay signal from a \B background event depends on the ability to identify the topology of Cherenkov light on the detector sphere on top of uniformly distributed scintillation light. We show that an analysis of spherical harmonics on the spatial distribution of early photons can be used to separate 0\nbb-decay and from \B events.


%In order to illustrate differences between different event topologies, we introduce three example event topologies: two electrons produced back-to-back at an 180$^{\circ}$ angle, two electrons at a 90$^{\circ}$ angle, and a single electron. The first two are representative of topologies for 0\nbb-decay signal events while the latter is representative of \B background events. Figure~\ref{fig:Display_top_5MeV} shows the Cherenkov photon distributions of 5~MeV electrons for each of the three topologies. Each distribution is made by overlaying 100 events in order to make the Cherenkov rings more visible. Note that in this overlay, we assume a quantum efficiency (QE) of one for the photodetectors.


%In practice, Cherenkov rings from low energy electrons are not clearly visible. Examples of such Cherenkov clusters, produced by events with a total kinetic energy of 2.539~MeV (the Q-value of \Te), are shown in Fig.~\ref{fig:Display_top_2p5MeV}. One can try to guess the event topology by comparing different segments of the detector sphere.



%For a quantitative description of the difference in the event topology, we analyze spherical harmonics of the photon distributions on the detector sphere. We construct rotation invariant variables and compare their value between signal and background events. As shown in the bottom part of Fig.~\ref{fig:Display_Te130}, 0\nbb-decay events become indistinguishable from single track topology when the angle between the two electrons is small (i.e. they form two degenerate tracks). Event topologies of 0\nbb-decay and \B events are also very similar when only one electron from 0\nbb-decay is above the Cherenkov threshold. Therefore spherical harmonics analysis is most efficient for events with large angular separation between the two electrons and when both electrons are above Cherenkov threshold. 
%
%In this paper we focus on topological differences between two-tracks and single-track events and do not make any attempt to use absolute directional information to suppress single track events. For example, one might reject events where the direction of the track is consistent with the direction of solar neutrinos. Once a single track topology is established, one can use a centroid method (see Ref.~\cite{Directionality}) to reconstruct directionality of the track (or two degenerate tracks) in order to suppress events that are aligned with the direction of \B solar neutrinos.

\subsection{Description of Spherical Harmonics Analysis}

The central strategy of the analysis is to construct rotationally invariant variables that can be used to separate signal and background events. To this end, let the  function $f(\theta,\phi)$ represent the PE distribution on the detector surface. The function $f(\theta,\phi)$ can be decomposed into a sum of spherical harmonics:

\begin{eqnarray}
\label{eq1}
f(\theta,\phi) = \sum_{l=0}^{\infty} \sum_{m=-l}^{l} f_{lm} Y_{lm}(\theta,\phi),
\end{eqnarray}

where $Y_{lm}$ are Laplace's spherical harmonics defined in a real-value basis using Legendre polynomials $P_l$:

\begin{eqnarray}
\label{eq2}
Y_{lm} = \left\{
  \begin{array}{@{}ll@{}}
    \sqrt{2}N_{lm}P_l^m(cos\theta)cos~m\phi, & \text{if}\ m>0 \\
    N_{lm} = \sqrt{\frac{(2l+1)}{4\pi} \frac{(l-m)!}{(l+m)!}}, & \text{if}\ m=0 \\
    \sqrt{2}N_{l|m|}P_l^|m|(cos\theta)sin~|m|\phi, & \text{if}\ m<0
  \end{array}\right.
\end{eqnarray}

where the coefficients $f_{lm}$ are defined as
 
\begin{eqnarray}
\label{eq3}
f_{lm} = \int_{0}^{2\pi} d\phi \int_0^{\pi} d\theta sin\theta f(\theta,\phi) Y_{lm}(\theta,\phi).
\end{eqnarray}

Equation~\ref{eq4} defines the power spectrum of $f(\theta,\phi)$ in the spherical harmonics representation, $s_l$, where $l$ is a multiple moment. The power spectrum, $s_l$, is invariant under rotation. It is unique to each of the functions $f_i(\theta,\phi)$, $i=$1,2,3..., which can not be transformed into each other by rotation.

\begin{eqnarray}
\label{eq4}
s_l = \sum_{m=-l}^{m=l} |f_{lm}|^2
\end{eqnarray}

%% here
%One can consider PEs distribution for each of 0\nbb-decay signal or background event as a function $f_i(\theta,\phi)$. Events with similar power spectrum would correspond to PE distributions on the detector sphere that can be closely aligned by a rotation. Such PE distributions belong to events with similar topology.

The topology of 0\nbb~signal or background in a spherical detector determines the distribution of the PE's on the detector sphere, and, therefore, a set of $s_l$'s. These values can serve as a quantitative figure of merit for different event topologies. The rotation invariance of $s_l$'s ensures that this figure of merit does not depend on the orientation of the event with respect to the chosen coordinate frame.

Sum of $s_l$'s over all multiple moments equals to the L2 norm of the function $f(\theta,\phi)$:

\begin{eqnarray}
\label{eq5}
\sum_{l=0}^{\infty} s_l = \int_{\Omega} |f(\theta,\phi)|^2 d\Omega.
\end{eqnarray}

Therefore, the normalized power spectrum,

\begin{eqnarray}
\label{eq6}
S_l = \frac{s_l}{\sum_{l=0}^{\infty} s_l} =  \frac{s_l}{\int_{\Omega} |f(\theta,\phi)|^2 d\Omega},
\end{eqnarray}

can be used to compare shapes of various functions $f(\theta,\phi)$ with different normalization. The total number of PEs detected on the detector sphere fluctuates from event to event, therefore, in all of the following we use the normalized power $S_l$.

Figure~\ref{fig:Moments} compares the normalized power spectra for the three representative event topologies that were previously shown in Fig.~\ref{fig:Display_top_5MeV}. We note that most of the information is contained in the power spectrum with $l<$6. In most cases we found that there is no need to calculate $S_l$ for $l>$3 to achieve maximal separation between 0\nbb~and \B~events. This limit is due to fluctuations in the PE distribution that produce a lot of noise in the power spectrum for higher orders of multiple moments.

%For a quantitative description of the difference in the event topology, we analyze spherical harmonics of the photon distributions on the detector sphere. We construct rotation invariant variables and compare their value between signal and background events. 

As shown in the bottom part of Fig.~\ref{fig:Display_Te130}, 0\nbb~events become indistinguishable from single-track events when the angle between the two electrons is small (i.e. they form two overlapping tracks). Event topologies of 0\nbb~and \B~events are also very similar when only one electron from 0\nbb~ is above the Cherenkov threshold. Therefore spherical harmonics analysis is most efficient for events with large angular separation between the two electrons and when both electrons are above Cherenkov threshold. 

Being able to distinguish between two-tracks and single-track events using the spherical analyses can allow further cuts to be made.  For example, one might use absolute directional information to suppress single track events where the direction of the track is consistent with the location of a known background such as the sun. Once a single track topology is established, one can use a centroid method (see Ref.~\cite{Directionality}) to reconstruct directionality of the track (or two degenerate tracks) in order to suppress events that are aligned with the direction of \B~solar neutrinos.

\subsection{Spherical Harmonics Analysis and Off-center Events}

The calculation of the normalized power spectrum, $S_I$, above, assumes that the coordinate system is located at the location of the interaction.  Therefore, in order to compare spherical harmonics for events with vertices not at the center of the detector volume, a coordinate transformation for each photon hit is needed. The necessary transformation applied for each PE within an event is illustrated in Fig.~\ref{fig:SphH_transform}. The solid circle in the Figure has a radius, R, and shows the actual detector boundaries. The dotted circle shows a new sphere with the same radius R, which now has the event vertex in its center. The radius vector of each PE is stretched or shorten to its intersection with this new sphere using the transformation, $\vec{r}^{,}_{PE} = \frac{\vec{a}}{|\vec{a}|} \cdot R$, where $\vec{r}^{,}_{PE}$ is a new radius vector of a PE and $\vec{a}=\vec{r}_{PE} - \vec{r}_{vtx}$ with $\vec{r}_{PE}$ and $\vec{r}_{vtx}$ being radius vectors of the PE and the vertex in the original coordinates, respectively.

\begin{figure*}[h]
  \centering
%  \includegraphics[width=0.95\textwidth]{SphH_transform_sketch.JPG}
  \includegraphics[width=0.95\textwidth]{SphH_transform.pdf}
  \caption{Coordinate transformation applied to events that are
    off-center. Solid circle schematically shows actual detector
    boundaries. Dotted circle shows a new sphere of radius R$=$6.5~m
    with the event vertex position in the center. The radius vector of
    each photon hit is stretched or shorten until intersection with
    this new sphere using transformation $\vec{r}^{,}_{hit} =
    \frac{\vec{a}}{|\vec{a}|} \cdot R$. Where $\vec{r}^{,}_{hit}$ is a
    new radius vector of the photon hit, $R$ is detector sphere radius,
    and $\vec{a}=\vec{r}_{hit} - \vec{r}_{vtx}$ with $\vec{r}_{hit}$
    and $\vec{r}_{vtx}$ being radius vectors of the photon hit and
    vertex position in original coordinates and correspondingly.}
  \label{fig:SphH_transform}
\end{figure*}


\subsection{Implementation of the spherical harmonics analysis}

For each event, we create a 2-D histogram, $\theta$ vs $\phi$, with the distribution of PEs on the detector surface. We then treat this histogram as a function $f(\theta,\phi)$ where the value of the function for any pair of $\theta$ and $\phi$ is equal to the number of PE in the histogram bin corresponding to that pair.

Coefficients $f_{lm}$ from Eq.~\ref{eq3} are calculated using a Monte Carlo integration technique. Variables $S_0$, $S_1$, $S_2$, and $S_3$ are calculated using Eqs.\ref{eq4} - \ref{eq6}.

To illustrate spherical harmonics analysis technique we compare distributions of $S_0$, $S_1$, $S_2$, and $S_3$ for the three representative event topologies described in Sec.~\ref{subsec:topology}. Almost all the information about event topology is carried by Cherenkov light. Therefore we first show spherical harmonics for back-to-back,  90$^{\circ}$ and single track topologies based on Cherenkov PEs only (see Fig.~\ref{fig:SL_topologies_CHE}).

Two top panels of Fig.~\ref{fig:SL_topologies_CHE} show 2-dimensional distributions, S0 vs S1 and S2 vs S3, to demonstrate that all four $S_l$'s provide separation between event topologies. No QE is applied in simulation of these events. We also introduce a 1-dimensional variable, S01 (bottom panel of Fig.~\ref{fig:SL_topologies_CHE}), that has the best separation power for majority of event topologies considered in this paper. $S_{01}$ is defined as a projection of S$_1$ vs S$_2$ distribution onto a linear fit of this 2-D distribution.

\begin{figure*}[h]
  \centering
  \includegraphics[width=0.49\textwidth]{ALL/hS0vsS1_topologies_CHELight_VtxSmear0cm_VtxShiftX0cm_33p5ns_center.pdf}
  \includegraphics[width=0.49\textwidth]{ALL/hS2vsS3_topologies_CHELight_VtxSmear0cm_VtxShiftX0cm_33p5ns_center.pdf}
  \includegraphics[width=0.9\textwidth]{ALL/hS01_topologies_CHELight_VtxSmear0cm_VtxShiftX0cm_33p5ns_center.pdf}
  \caption{Spherical harmonics for three event topologies: two
    back-to-back 1.26~MeV electrons (\emph{black squares and black
      dotted line}), two 1.26~MeV electrons at 90$^{\circ}$ angle
    (\emph{blue triangles and blue dashed line}), and a single
    2.529~MeV electron representing $^{8}$B background (\emph{red
      crosses and red solid line}). Simulation of 1000 events
    originated at the center of the sphere. Perfect separation between
    Cherenkov and scintillation light is implemented in this
    simulation by using only Cherenkov photons. \emph{Top left:} $S_0$
    versus $S_1$ scatter plot. Black dotted line is a linear fit of
    the 90$^{\circ}$ topology and $^{8}$B events. Variable $S_{01}$ is
    defined as a projection of 2D distribution onto this linear
    fit. \emph{Top right:} $S_2$ versus $S_3$ scatter
    plot. \emph{Bottom:} $S_{01}$ distributions for the three
    topologies. These distributions are normalized to unit area for
    shape comparison.}
  \label{fig:SL_topologies_CHE}
\end{figure*}


The effects due to the presence of scintillation light and applying the default QE are shown in Fig.~\ref{fig:SL_topologies_all}. Spherical harmonics of the same three representative event topologies are now calculated using early light (photons with arrival time less than 33.5~ns) that contains both directional Cherenkov light and uniform scintillation light. The of number PE seen by each tube is reduced by the default QE. In this more realistic scenario, the higher order multiple moments, S2 and S3, no longer provide noticeable separation between different event topologies.


\begin{figure*}[h]
  \centering
  \includegraphics[width=0.49\textwidth]{hS0vsS1_topologies_allLight_VtxSmear0cm_VtxShiftX0cm_33p5ns_center.pdf}
  \includegraphics[width=0.49\textwidth]{hS2vsS3_topologies_allLight_VtxSmear0cm_VtxShiftX0cm_33p5ns_center.pdf}
  \includegraphics[width=0.9\textwidth]{hS01_topologies_allLight_VtxSmear0cm_VtxShiftX0cm_33p5ns_center.pdf}
  \caption{Spherical harmonics for three event topologies: two
    back-to-back 1.26~MeV electrons (\emph{black squares and black
      dotted line}), two 1.26~MeV electrons at 90$^{\circ}$ angle
    (\emph{blue triangles and blue dashed line}), and a single
    2.529~MeV electron representing $^{8}$B background (\emph{red
      crosses and red solid line}). Simulation of 1000 events
    originated at the center of the sphere. Separation between
    Cherenkov and scintillation light is implemented 33.5~ns cut on
    the photon arrival time. Perfect vertex reconstruction - true
    vertex position is used. \emph{Top left:} $S_0$ versus $S_1$
    scatter plot. Black dotted line is a linear fit of the
    90$^{\circ}$ topology and $^{8}$B events. Variable $S_{01}$ is
    defined as a projection of 2D distribution onto this linear
    fit. \emph{Top right:} $S_2$ versus $S_3$ scatter
    plot. \emph{Bottom:} $S_{01}$ distributions for the three
    topologies. These distributions are normalized to unit area for
    shape comparison}
\label{fig:SL_topologies_all}
\end{figure*}

