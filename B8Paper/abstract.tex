We present a technique for separating nuclear double beta decay
(\bb-decay) events from background neutrino interactions due to
\B~decays in the sun.  This background becomes dominant in a
kiloton-scale liquid-scintillator detector deep underground and is
usually considered as irreducible due to an overlap in deposited
energy with the signal.  However, electrons from 0\nbb-decay often
exceed the Cherenkov threshold in liquid scintillator, producing
photons that are prompt and correlated in direction with the
initial electron direction. The use of large-area fast photodetectors
allows some separation of these prompt photons from delayed isotropic
scintillation light and, thus, the possibility of reconstructing the
event topology.  Using a simulation of a 6.5~m radius liquid
scintillator detector with 100~ps resolution photodetectors, we show
that a spherical harmonics analysis of early-arrival light can
discriminate between 0\nbb-decay signal and
\B~ solar neutrino background events on a statistical basis. 
Good separation will require the development of a slow scintillator 
with a 5 nsec risetime.



%We propose a technique for separating 0{\nbb}-decay events from
%background due to $^{10}$C decays and $^8$B solar neutrino
%interactions in a liquid scintillator detector. These represent the
%key backgrounds at shallow and deep sites. In particular, we focus
%on \B~background suppression which is traditionally viewed as irreducible
%background to 0\nbb-decay searches. The technique compares
%event topology of the signal and background events using spherical
%harmonics analysis of the early light emitted in signal and
%background events. Selection of early photons using fast photo-detectors
%allows for separation of directional Cherenkov from isotropic
%scintillation light and identification of event topologies based
%on the spatial distribution of the early photons in the detector.
