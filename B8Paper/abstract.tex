We present a technique for separating double beta decay (\bb-decay) events from 
background due to \B~solar neutrino interactions. This background becomes dominant 
in a kilo-ton scale liquid scintillator detector. In searches for the neutrinoless 
double beta decay (0\nbb-decay) \B~background is usually considered as irreducible 
due to an overlap in the energy deposition. We note that in a liquid scintillator 
detector electrons from 0\nbb-decay often exceed Cherenkov threshold. Selection of 
early photons using fast photo-detectors separates prompt directional Cherenkov 
light from delayed isotropic scintillation light. This leads to the possibility of 
reconstructing the event topology of 0\nbb-decay candidate events by analyzing 
spatial distribution of early photons. Using a simulation of a 6.5~m radius liquid 
scintillator detector with 100~ps resolution photo-detectors, we perform a 
spherical harmonics analysis of the early light emitted in each candidate event and 
show the difference between 0\nbb-decay signal and \B~background events. We discuss 
key detector parameters that affect the separation power.


%We propose a technique for separating 0{\nbb}-decay events from
%background due to $^{10}$C decays and $^8$B solar neutrino
%interactions in a liquid scintillator detector. These represent the
%key backgrounds at shallow and deep sites. In particular, we focus
%on \B~background suppression which is traditionally viewed as irreducible
%background to 0\nbb-decay searches. The technique compares
%event topology of the signal and background events using spherical
%harmonics analysis of the early light emitted in signal and
%background events. Selection of early photons using fast photo-detectors
%allows for separation of directional Cherenkov from isotropic
%scintillation light and identification of event topologies based
%on the spatial distribution of the early photons in the detector.
