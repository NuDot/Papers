%%\documentclass[preprint,12pt,authoryear]{elsarticle}

%% Use the option review to obtain double line spacing
%%\documentclass[preprint,12pt]{elsarticle}

%% Use the options 1p,twocolumn; 3p; 3p,twocolumn; 5p; or 5p,twocolumn
%% for a journal layout:
%% \documentclass[final,1p,times]{elsarticle}
%% \documentclass[final,1p,times,twocolumn,longtitle]{elsarticle}
 \documentclass[final,3p,times]{elsarticle}
%% \documentclass[final,3p,times,twocolumn,longtitle]{elsarticle}
%% \documentclass[final,5p,times]{elsarticle}
%% \documentclass[final,5p,times,longtitle,twocolumn]{elsarticle}

%% For including figures, graphicx.sty has been loaded in
%% elsarticle.cls. If you prefer to use the old commands
%% please give \usepackage{epsfig}

%% The amssymb package provides various useful mathematical symbols
\usepackage{amssymb}
\usepackage{amsmath}
%% The amsthm package provides extended theorem environments
\usepackage{amsthm}

\usepackage{lipsum}
\usepackage{graphicx}% Include figure files
\usepackage{dcolumn}% Align table columns on decimal point
\usepackage{bm}% bold math
\usepackage{multirow}% multirow for tables
\usepackage{comment}% commenting blocks
%\usepackage{subfig}
\usepackage{subfigure}
\usepackage{soul}

%\usepackage[square]{natbib}
\usepackage[usenames,dvipsnames]{color}
\usepackage{hyperref}% add hypertext capabilities

% The line numbering messes up the title. Comment out these lines and
% the title page will be fine.
\usepackage[mathlines]{lineno}% Enable numbering of text and display math
\DeclareGraphicsExtensions{.pdf,.png,.jpg}

\bibliographystyle{model1-num-names}
\biboptions{sort&compress}

%\usepackage[font=small,skip=0pt]{caption}
%\usepackage[showframe,%Uncomment any one of the following lines to test 
%scale=0.7, marginratio={1:1, 2:3}, ignoreall,% default settings
%%text={7in,10in},centering,
%%margin=1.5in,
%%total={6.5in,8.75in}, top=1.2in, left=0.9in, includefoot,
%%height=10in,a5paper,hmargin={3cm,0.8in},
%]{geometry}
%

\journal{Nuclear~Instruments~and~Methods~A}


\begin{document}
\newcommand{\nbb}{\ensuremath{\nu\beta\beta}}


\newcommand\JOcom[1]{{\textcolor{red}{#1}}}


%\graphicspath{{plots/}}
\graphicspath{{figs/}}

%\usepackage{lineno}
\linenumbers
%\usepackage{setspace}
%\doublespacing
\setstretch{2.0}
%\bibliography

%%%%%%%%%%55555
%YOUAREHERE
\large
%%%%%%%%55555555555555

\title{Separating Double-Beta Decay Events from Solar Neutrino
Interactions in a Kiloton-Scale Liquid Scintillator Detector By Fast Timing}
\author[UChicago]{Andrea Elagin\corref{cor1}}
\cortext[cor1]{Corresponding Author: \href{mailto:an.email.address.here@somewhere.com}{\tt{an.email.address.here@somewhere.com}}}
\author[UChicago]{Henry Frisch}
\author[MIT]{Lindley Winslow}
%\author[MIT]{Jonathan Ouellet}

\address[UChicago]{ Enrico Fermi Institute, University of Chicago }
\address[MIT]{ Massachusetts Institute of Technology, Cambridge, MA 02139 }


\begin{abstract}
We present a technique for separating nuclear double beta decay
(\bb-decay) events from background neutrino interactions due to
\B~decays in the sun.  This background becomes dominant in a
kiloton-scale liquid-scintillator detector deep underground and is
usually considered as irreducible due to an overlap in deposited
energy with the signal.  However, electrons from 0\nbb-decay often
exceed the Cherenkov threshold in liquid scintillator, producing
photons that are prompt and correlated in direction with the
initial electron direction. The use of large-area fast photodetectors
allows some separation of these prompt photons from delayed isotropic
scintillation light and, thus, the possibility of reconstructing the
event topology.  Using a simulation of a 6.5~m radius liquid
scintillator detector with 100~ps resolution photodetectors, we show
that a spherical harmonics analysis of early-arrival light can
discriminate between 0\nbb-decay signal and
\B~ solar neutrino background events on a statistical basis. 
Good separation will require the development of a slow scintillator 
with a 5 nsec risetime.



%We propose a technique for separating 0{\nbb}-decay events from
%background due to $^{10}$C decays and $^8$B solar neutrino
%interactions in a liquid scintillator detector. These represent the
%key backgrounds at shallow and deep sites. In particular, we focus
%on \B~background suppression which is traditionally viewed as irreducible
%background to 0\nbb-decay searches. The technique compares
%event topology of the signal and background events using spherical
%harmonics analysis of the early light emitted in signal and
%background events. Selection of early photons using fast photo-detectors
%allows for separation of directional Cherenkov from isotropic
%scintillation light and identification of event topologies based
%on the spatial distribution of the early photons in the detector.

\end{abstract}

\date{\today}

\maketitle

%\newpage
\clearpage
\tableofcontents
%\newpage

\linenumbers\relax % Commence numbering lines
\clearpage

%\begin{comment}
\section{Introduction}

Over the past decade and a half, neutrino oscillation experiments have
been able to conclusively establish that neutrinos have mass
\cite{SNO2001,SNO2002,SuperK2002,kamland2003}. However, the nature of
that mass remains one of the most fundamental open questions in
particle physics. Is the neutrino unique among the Standard Model
fermions with a Majorana-type mass \cite{Majorana1937}, as is
predicted by most beyond the standard model (BSM) theories, or does it
have a Dirac-type mass like the rest of the fermions? A Majorana-type
mass would have far reaching implications, from explaining the
lightness of the neutrino and providing a bridge to higher energy
phenomena through the see-saw mechanism
\cite{GellMann1980,Yanagida1979} to being able to provide the required
lepton-number violation (LNV) and {\sf CP}-violation needed for
leptogenesis to explain the baryon asymmetry of the universe
\cite{Fukugita1986,Luty1992}. Conversely, a Dirac-type neutrino mass
could point to an underlying symmetry of the Universe.  Presently, the
most promising technique for answering these questions is the search
for Neutrinoless Double-Beta (0\nbb) decay \cite{Furry1939}. In this
decay, a nucleus undergoes a second order $\beta$-decay without
producing any neutrinos, $(Z,A)\rightarrow(Z+2,A)+2\beta^-$. 

Recently, this search has generated a significant amount of
experimental interest, with the largest on-going experiments searching
for {0\nbb} decay of $^{76}$Ge \cite{GERDA2013}, $^{130}$Te
\cite{CUORE2015,CUORE2016} and $^{136}$Xe
\cite{EXO2014,KamLANDZen2013}. At present, 0{\nbb} decay has never
been convincingly observed, but present limits indicate that the
half-lives are longer than $10^{23}-10^{25}\,\mathrm{yr}$ in the
isotopes studied. The standard mechanism of 0{\nbb} decay is
parameterized by the \emph{effective Majorana mass}, defined as
\mbox{$m_{\beta\beta}\equiv\left|\sum_i U^2_{ei}m_i\right|$}, where
$U_{ei}$ are the elements of the PMNS matrix and $m_i$ are the
neutrino masses. Current half-life limits translate to a limit on
\mbox{$m_{\beta\beta}\lesssim 150-700\,\mathrm{meV}$}. The majority of
the spread on this bound comes from theoretical uncertainty in the
nuclear modeling (see \cite{Vogel2012} for a review). The next
generation of 0{\nbb} decay experiments seek to be sensitive enough to
detect or rule out 0{\nbb} decay down to \mbox{$m_{\beta\beta}\lesssim
  10$~meV}. This will require a detector to instrument roughly a ton
of active isotope, maintain a good energy resolution, and achieve a
near zero background in the region of interest (ROI) over the course
of the experiment \cite{Cremonesi2013} [Second CITATION?].

Over the past few years, liquid scintillator-based detectors have
proven to be a competitive technology in this search
[CITATION?]. Their primary advantage is their ease of scalability to
larger instrumented masses, which involves disolving larger amounts of
the isotope of interest into the liquid scintillator (LS). This
feature can allow for rapid scaling to 1~ton or more using the
detectors already in operation \cite{Biller2013}. In a large LS
detector, most backgrounds can be strongly suppressed through a
combination of filtration of the LS to remove internal contaminents,
self-shielding to minimize the effects of external contaminents, and
vetoing to reduce cosmic ray effects. The backgrounds relevant to
0{\nbb} decay which cannot be reduced through these means are the
2{\nbb} decay and electron scattering of $^8$B solar neutrinos.

Since 0{\nbb} decay produces no neutrinos, the full energy of the
decay is contained within the detector and the observed spectrum of
this decay is a peak around the decay $Q$-value. Most of this energy
is carried by the electrons which have typical kinetic energies of
$\sim1-2\,\mathrm{MeV}$ each. Two neutrino double beta (2{\nbb}) decay
\cite{GoeppertMayer1935} is the Standard Model allowed second order
$\beta$-decay channel where lepton number is conserved by the
production of two anti-neutrinos,
\mbox{$(Z,A)\rightarrow(Z,A+2)+2\beta+2\bar\nu_e$}. Since the kinetic
energy of the neutrinos is practically never detected, the energy
spectrum measured from 2{\nbb} decay is broadened from 0~MeV up to the
decay $Q$-value. Because it is intrinsic to the target isotope, the
high energy tail of the 2{\nbb} spectrum forms an irreducible
background to the 0{\nbb} signal. The only way to distinguish the two
is through a shape analysis of the resulting decay spectrum. This
requires a detector with a good energy resolution (see
Fig.~\ref{fig:SNOp_bkgs}). Present LS-based detectors achieve typical
energy resolutions of \mbox{$\sigma(E)\sim 5\%/\sqrt{E(\rm
    MeV)}$}. The next generation of detectors will seek to improve
upon this by increasing both the photo-covering of the detector and
light yield of the LS. \JOcom{Eventually this will fold back in the
  question of slowing down the scintillation signal and improving the
  Cherenkov signal at the cost of decreasing the total light yield.}
  
The spectrum of ES interactions of $^{8}$B solar neutrinos falls
slowly over the range $2-3\,\mathrm{MeV}$, creating a nearly flat
background across the ROI and reducing the sensitivity to 0{\nbb}. In
this energy region, these interactions produce only a single
$\sim$2.5~MeV electron, rather than two $\sim$1.2~MeV electrons as in
0{\nbb}. In a LS, this difference in event topology manifests as two
distinct distributions of Cherenkov photons, and thus creates a way to
tag and remove these $^{8}$B solar neutrino events. As we have shown
in previous works, photo-detectors with timing resolution of
$\sim$100~ps can resolve the prompt Cherenkov photons from the slower
scintillation signal \cite{Aberle2014}. The challenge is that for a
given event, we expect $\sim$100 Cherenkov photons with which to
reconstruct the event topology. \JOcom{It would be good to say
  something like: In a SNO/KamLAND sized detector, we expect $\sim$50
  $^{8}$B events, so our rejection needs to be at least this good. But
  perhaps we say that later?}

In this paper, we propose to use a spherical harmonic decomposition
to analyze the distribution of early photo-electrons (PE) to
discriminate between $^{8}$B solar neutrinos and 0{\nbb} decay
events. In Section~\ref{sec:detector_description}, we describe the
detector model we will use throughout this paper. In
Section~\ref{sec:spherical_harmonics}, we introduce the spherical
harmonic decomposition, and discuss the performance of this analysis
in Section~\ref{sec:performance_and_challenges}.





\section{Detector Model}
\label{sec:detector_description}

In order to study the topology of $\vbb$ decay and background events in a liquid scintillator detector, a Geant4\cite{geant4one,geant4two} simulation has been constructed. This is the same simulation used in our preceding paper~\cite{Aberle2014}. Therefore, we limit out discussion of the simulation to a summary of the the most relevant simulation parameters.

The simulation uses Geant4~version 4.9.6.  We use the default liquid scintillator optical model, in which optical photons are assigned the group velocity in the wavelength region of normal dispersion.

The detector geometry is a sphere of 6.5~m radius filled with
scintillator. The default scintillator composition has been chosen to match a KamLAND-like
scintillator\cite{kamland2003}: 80\% n-dodecane, 20\% pseudocumene and 1.52~g/l PPO. The
scintillator properties implemented in the simulation include 
\begin{itemize}
\item the atomic composition and density ($\rho$ = 0.78~g/ml), 
\item the wavelength-dependent attenuation length\cite{tajimaMaster} and refractive index\cite{OlegThesis}, 
\item the scintillation emission spectrum\cite{tajimaMaster}, 
\item emission rise time ($\tau_r$ = 1.0~ns) and emission decay time constants ($\tau_{d1}$ = 6.9~ns and $\tau_{d2}$ = 8.8~ns with relative weights of 0.87 and 0.13)\cite{tajimaThesis}, 
\item scintillator light yield (9030 photons/MeV), and 
\item the Birks constant ($kB$ $\approx$ 0.1~mm/MeV)\cite{ChrisThesis}.  
\end{itemize}
The attenuation length at 400~nm, which is the position of the peak standard bialkali photocathode efficiency, is 25~m. The attenuation length drops precipitously from 6.5~m to 0.65~m between 370~nm and 360~nm. We use this drop to define the cutoff wavelength at 370~nm. This is a standard scintillator. However, we do deviate from the baseline KamLAND case in that the re-emission of absorbed photons in the scintillator bulk volume and optical scattering, specifically Rayleigh scattering, has not yet been included. A test simulation shows that the effect of optical scattering is negligible~\cite{Aberle2014}.

The inner sphere surface is used as the photodetector. It is treated
as fully absorbing (no reflections), with a photodetector coverage of
100\%. As in the case of optical scattering, reflections at the sphere are a small effect that would create a small tail at longer times. The default is the QE of a bialkali photocathode (Hamamatsu
R7081 PMT)\cite{Hamamatsu_R7081}. The QE values as a function of wavelength come from the Double Chooz\cite{dctwo}
Monte Carlo simulation. We note that the KamLAND 17-inch PMTs use the
same photocathode type with similar quantum efficiency. We are neglecting any threshold effects in the photodetector readout electronics.


Four effects primarily contribute to the timing of the scintillator detector
system: the travel time of the particle, the time constants of the scintillation process, chromatic dispersion, and the timing of the photodetector.

In the energy range important for $0\nu\beta\beta$, a 1.4~MeV electron travels a total path length of 0.8~cm, has a distance from the origin of 0.6~cm in 0.030$\pm$0.004~ns  and takes 0.028$\pm$0.004~ns to drop below Cherenkov threshold. We note that due to scattering of the electron, the final direction of the electron before it stops does not correspond to the initial direction; however the scattering angle is small while the majority of Cherenkov light is produced. The Cherenkov light thus still encodes the direction of the primary electron. The scattering physics is handled by Geant4's ``Multiple Scattering" process which is valid down to 1~keV, where atomic shell structure becomes important\cite{geant4scatt}.


The scintillator-specific rise and decay times are the second effect that determines the timing in a scintillator detector. The first step in the scintillation process is the transfer of energy from the solvent to the solute. The time constant of this
energy transfer accounts for a rise time in scintillation light
emission. Because past neutrino experiments were not highly sensitive to the
effect of the scintillation rise time, there is a lack of accurate measurements of this property. We assume a rise time of 1.0~ns -- but more
detailed studies are needed in the future. The two time constants used
to describe the falling edge of the scintillator emission time
distribution (quoted above) are values specific to the KamLAND scintillator.

Chromatic dispersion is the third effect that determines the timing in a scintillator detector. Due to the wavelength-dependence of the refractive index the speed of
light in the scintillator increases
with increasing photon wavelengths for normal dispersion, with red
light traveling faster than blue light.

Photoelectrons coming from Cherenkov light are on average
created about 0.5~ns earlier than PEs from scintillation light. The
RMS values from PE time distributions for Cherenkov and scintillation
light are both about 0.5~ns. Note that these numbers include the
effect of the finite electron travel time.

The fourth effect determining the timing in a scintillator detector is the timing of the photodetectors. The measurement of the arrival times of single photoelectrons is
affected by the transit-time spread (TTS) of the photodetectors, a
number which can be different by orders of magnitude depending on the
detector type. We use a TTS of 0.1~ns ($\sigma$), which can be achieved with large area picosecond photodetectors
(LAPPDs)\cite{Adams:2013nva,RSI_paper,PSEC4_paper,anode_paper} and possibly hybrid photodetectors
(HPDs)\cite{hpdThesis}; even significantly lower TTS numbers are
realistic with the LAPPD\cite{RSI_paper,PSEC4_paper,anode_paper}.

The primary quantities provided by the Geant4~simulation are the photoelectron hit
positions and the detection times after the TTS resolution has been
applied. These quantities are then used for event topology reconstruction.

Figures~\ref{fig:ArrivalTimeDist} and~\ref{fig:NPhotDist} show the output of the detector simulation discussed in this section. Left panel in Fig.~\ref{fig:ArrivalTimeDist} compares PE arrival time between Cherenkov and scintillation light for 1000 simulated $\Te$ $\vbb$-decay events. The right panel in Fig.~\ref{fig:ArrivalTimeDist} compares the Cherenkov PE arrival times between $\Te$ $\vbb$-decay and $\B$ events. $\B$ events produce a slightly higher number of the Cherenkov photons because they have only one electron carrying the same kinetic energy as opposed to the two electrons in the case of $\vbb$-decay events. Distributions of the scintillation PEs' arrival time are indistinguishable between $^{130}$Te 0{\nbb} decay and $^8$B due to identical total energy in the event, $Q(^{130}{\rm Te})=2.526$~MeV.

\begin{figure*}[ht]
  \centering
  \includegraphics[width=0.45\textwidth]{hT_Te130.pdf}
  \includegraphics[width=0.45\textwidth]{hTche_Te130_B8.pdf}
  \caption{\emph{Left:} Photo-electron (PE) arrival times after
    application of the photo-detector transit time spread (TTS) of
    100~ps for the simulation of 1000 0{\nbb} decay events of
    $^{130}$Te at the center of the detector. PEs from Cherenkov light
    (\emph{dashed red line}) and scintillation light (\emph{solid blue
      line}) are compared. The black vertical line illustrates a time
    cut at 33.5 ns. \emph{Right:} Comparison between Cherenkov PEs
    arrival time for $^{130}$Te {0\nbb} decay (\emph{solid line}) and
    $^{8}$B (\emph{dotted line}) events. {\bf Distributions of the
      scintillation PEs arrival time are indistinguishable between
      $^{130}$Te 0{\nbb} decay and $^8$B due to identical total energy
      in the event, $Q(^{130}{\rm Te})=2.526$~MeV.} }
\label{fig:ArrivalTimeDist}
\end{figure*}

As shown in Fig.~\ref{fig:ArrivalTimeDist}, a time cut of 33.5~ns on the PE arrival time selects a sample of early PEs that includes the majority of Cherenkov photons. Scintillation PEs also are selected with this time cut. Figure~\ref{fig:NPhotDist} shows the total number of scintillation and Cherenkov PE per event for $\vbb$ signal and $\B$ background events. 
The $\B$ events do have a higher number of Cherenkov PEs on average compared to $\vbb$ events because of the single electron in $\B$ events has a higher energy than the two electrons from $\vbb$ decay.   This difference, though, is not significant enough to be used alone as a reliable discriminant between $\vbb$-decay and $\B$ events.  However, it may provide an extra handle on signal-background separation in combination (e.g. by using multivariative techniques) with other event parameters.

\begin{figure*}[ht]
  \centering
  \includegraphics[width=0.45\textwidth]{hMomNPhot_Te130.pdf}
  \includegraphics[width=0.45\textwidth]{hMomNPhot_1el_2p529MeV.pdf}
  \caption{Number of Cherenkov (\emph{dashed red line}), scintillation
    (\emph{dotted blue line}), and total (\emph{solid black line}) PEs
    for the simulation of 1000 $^{130}$Te 0{\nbb} decay (left panel)
    and $^8$B (\emph{right panel}) events.}
\label{fig:NPhotDist}
\end{figure*}


\section{Kinematics and Timing of Signal and Background events}
\label{sec:kinematics_and_timing}

\subsection{0\nbb-decay signal and 2\nbb-decay background}

In both 0\nbb-decay signal and 2\nbb-decay background events near the decay energy spectrum endpoint, the kinematics of the
electron pair is very similar. Large fraction of events have a nearly back-to-back topology with a close to 
equal energy split between electrons. To simulate 0\nbb- and 2\nbb-decay events we use a Monte Carlo generator based on phase 
factors from Ref.~\cite{Jenni}. Similarity in kinematics of 0\nbb- and 2\nbb-decay events is demonstrated in Fig.~\ref{fig:Kinematics}.

The electron angular correlations for 0\nbb-decay are noticeably different from 2\nbb-decay due to a contribution from the
neutrino wave-functions even at vanishingly small energies of the neutrinos~\cite{Jenni}. However, any practical use of this difference 
in separating 0\nbb-decay from 2\nbb-decay would require extremely large number of candidate events. Given the half-time of 2\nbb-decay 
and upper limits on the half-time of 0\nbb-decay, electron angular correlations will not bring a decisive separation power in controlling 
2\nbb-decay background in currently planned 0\nbb-decay experiments. Excellent energy resolution at the Q-value remains the key parameter
in 2\nbb~background suppression.

While we do not exclude that the angular correlations as an input to a multivariative technique may improve sensitivity of 0\nbb-decay 
searches, in this paper we assume that there is no difference in the event topology between 0\nbb-~and 2\nbb-decay events. Any conclusions 
about 0\nbb-decay events also hold for 2\nbb-decay when the total energy of the electrons in 2\nbb-decay events is close to the Q-value.


\begin{figure*}[ht]
  \centering
  \includegraphics[width=0.49\textwidth]{hCos_Te130.pdf}
  \includegraphics[width=0.49\textwidth]{hE1toQ_Te130.pdf}
  \caption{Comparison between kinematics of 0{\nbb} (\emph{dashed red
      lines}) and 2{\nbb} decays (\emph{solid black lines}) for events
    with the total kinetic energy of the electrons above 90\% of the
    Q-value. \emph{Left:} Cosine of the angle between two
    electrons. \emph{Right:} Fraction of energy carried by one of the
    two electrons. Vertical bars at each bin of the histograms indicate
    statistical uncertainty for that bin.}
  \label{fig:Kinematics}
\end{figure*}


%Should we recalculate for 130Te.
Examining the kinematics for one of the 0\nbb~electrons with equal energy split, a 1.26~MeV electron travels a total path length of 0.X~cm, 
has a distance from the origin of 0.X~cm in 0.0X$\pm$0.00X~ns  and takes 0.0X$\pm$0.00X~ns to drop below Cherenkov threshold. 
We note that due to scattering of the electron, the final direction of the electron before it stops does not correspond to the initial 
direction; however the scattering angle is small at the time the majority of Cherenkov light is produced.

Figure~\ref{fig:ArrivalTimeDist} shows the output of the detector simulation for 1000 simulated \Te~ 0\nbb-decay 
events. The left panel in Fig.~\ref{fig:ArrivalTimeDist} compares PE arrival time between Cherenkov and scintillation light  
and the right panel in Fig.~\ref{fig:ArrivalTimeDist} zooms in on the Cherenkov photon distribution which is key to direction and 
topographical reconstruction.


\begin{figure*}[ht]
  \centering
  \includegraphics[width=0.45\textwidth]{hT_Te130.pdf}
  \includegraphics[width=0.45\textwidth]{hTche_Te130_B8.pdf}
  \caption{\emph{Left:} Photo-electron (PE) arrival times after
    application of the photo-detector transit time spread (TTS) of 100~ps for the default simulation 
    of \Te~0\nbb-decay produced at the center of the detector. 
    Scintillation PE arrival time distribution is compared for \nbb-decay (dashed blue line) and
    \Cten~events (dotted green line). The corresponding distribution for \B~events is not shown
    because it is indistinguishable from the distribution for \nbb-decay. Cherenkov PE arrival
    times are shown for \nbb-decay (\emph{solid red line}) to demonstrate their contribution to the early PE sample.
    The vertical line at 33.5~ns indicates the time cut for the selection of the early PE sample.
    The shape of scintillation PE arrival times for \B~events
    \emph{Right:} Comparison between Cherenkov PEs arrival time for \Te~0\nbb-decay (\emph{solid line}), 
    \B~(\emph{dashed line}).}
\label{fig:ArrivalTimeDist}
\end{figure*}


Selection of PEs with relatively small arrival time allows the selection of a sample of PEs with a high fraction of directional Cherenkov light.
This allows for event topology reconstruction. In particular, signal-like events with exactly two electrons can be separated from events 
with only one electron such as from \B~solar neutrino interactions.

As shown in Fig.~\ref{fig:ArrivalTimeDist}, for events produced at the center of the detector, a time cut of 33.5~ns on the PE arrival 
time selects a sample of early PEs that includes the majority of directional Cherenkov photons. Scintillation PEs also are 
selected with this time cut. Figure~\ref{fig:NPhotDist} shows the total number of scintillation and Cherenkov PE per event in the early 
PE sample for signal and background events.



\begin{figure*}[ht]
  \centering
  \includegraphics[width=0.45\textwidth]{hMomNPhot_Te130.pdf}
  \includegraphics[width=0.45\textwidth]{hMomNPhot_1el_2p529MeV.pdf}
%  \includegraphics[width=0.33\textwidth]{hMomNPhot_C10.pdf}
  \caption{Early PE sample composition: number of Cherenkov (\emph{dashed red line}), scintillation
    (\emph{dotted blue line}), and total (\emph{solid black line}) PEs per event
    for the simulation of 1000 $^{130}$Te 0\nbb-decay events (\emph{left panel}),
    1000 $^8$B events (\emph{middle panel}).}%, and 4152 \C~events (\emph{right panel}).}
\label{fig:NPhotDist}
\end{figure*}



\subsection{\B~background}

For a detector similar to our model, \B~background is significant due to large total mass of the liquid scintillator in
the active region of the detector.
Electrons from elastic scattering of \B~solar neutrinos have nearly a flat energy spectrum around the 
Q-value~\cite{SNOp-B8-bkg}. We simulate \B~background as a single monochromatic electron with energy of 2.53~MeV 
(Q-value of \Te). A 2.53~MeV electron travels a total path length of 0.X~cm, has a distance from the origin of 0.X~cm in 
0.0X$\pm$0.00X~ns  and takes 0.0X$\pm$0.00X~ns to drop below Cherenkov threshold.

The scintillation PE timing distribution is unchanged compared to 0\nbb-decay since the electron's path length at these energies 
is too short to affect the effective vertex of the scintillation light. The number of Cherenkov photons is increased 
(see Fig.~\ref{fig:ArrivalTimeDist} and~\ref{fig:NPhotDist}) due to the increased electron kinetic energy, but this alone is not 
sufficient to distinguish \B~events from 0\nbb-decay. However, it may provide an extra handle on signal-background separation in 
combination (e.g. by using multivariative techniques) with other event parameters.


\section{Event Topology and Spherical Harmonics Analysis}
\label{sec:topology_and_harmonics}

\subsection{Topology of $\vbb$-decay and $\B$ Events}
\label{subsec:topology}
%Signature of the $\vbb$-decay is two electrons with total kinetic energy equal to the isotope Q-value (e.g., 2.529~MeV for $\Te$). Therefore all $\vbb$-decay searches are designed to look for an excess of events in the energy spectrum around Q-value over the predicted number of background events. Backgrounds such as $\B$ solar neutrinos have distinct event topology that can be used to improve experimental sensitivity to the the $\vbb$-decay signal.

Electrons in the energy range around Q-value of all isotopes considered for $\vbb$-decay searches are above Cherenkov threshold in liquid scintillators. Each electron above the threshold will produce a fuzzy ring of Cherenkov light at the detector surface. The fuzziness of the ring depends on electron scattering. In most cases Cherenkov rings from low energy electrons degrade to randomly shaped clusters of Cherenkov photons around direction of the electron track. 

Large fraction of $\vbb$-decay events will have two Cherenkov clusters~\footnote{Only one Cherenkov cluster is produced when either the angle between the two $\vbb$-decay electrons is too small or when the energy splits between the electrons in such a way that one electron falls below the Cherenkov threshold.} as opposed to one cluster from $\B$ events. Therefore separation of $\vbb$-decay signal from $\B$ background depends on ability to identify topology of the Cherenkov light on the detector sphere on top of uniformly distributed scintillation light. We show that analysis of spherical harmonics of the early photons allows to achieve noticeable separation between $\vbb$-decay and $\B$ events.



%The simplest case for spherical harmonics analysis are events with the vertex located exactly in the center of the detector. For such event Cherenkov and scintillation light can be separated by applying a time cut on the photon arrival time as demonstrated in~\cite{Directionality}. To introduce the technique of spherical harmonics analysis we will follow the same strategy as in~\cite{Directionality} and use central events with a slightly different cut on the photon arrival time of 33.5~ns.

In order to illustrate differences between different event topologies we introduce three event topologies: two electrons produced back-to-back at 180$^{\circ}$ angle, two electrons at 90$^{\circ}$ angle, and a single electron. The two former are representative topologies of $\vbb$-decay signal events and the latter represents $\B$ background events. Figure~\ref{fig:Display_top_5MeV} shows Cherenkov photon distributions of 5~MeV electrons for each of the three topology. Overlay of 100 events with no QE applied is shown in order to make Cherenkov rings visible.


\begin{figure*}[h]
  \centering
  \includegraphics[width=0.3\textwidth]{hDisplay_topology180_5MeV.JPG}
  \includegraphics[width=0.3\textwidth]{hDisplay_topology90_5MeV.JPG}
  \includegraphics[width=0.3\textwidth]{hDisplay_1el_5MeV.JPG}
  \caption{Cherenkov photons distributions on the detector sphere for
    the three representative event topologies: two back-to-back
    electrons (\emph{left}), two electrons at 90$^{\circ}$ angle
    (\emph{middle}), and a single electron (\emph{center}).  All
    electrons are 5~MeV and originate at the center of the
    detector. 100 events overlayed for better visibility of the
    Cherenkov rings. 100\% QE is assumed. \JOcom{These can not be
      included as PDFs, but my conversion shrunk the size.}}
  \label{fig:Display_top_5MeV}
\end{figure*}

In practice Cherenkov rings from low energy electrons are not clearly
visible. Cherenkov clusters that form at lower energies are shown in Fig.~\ref{fig:Display_top_2p5MeV}. All Cherenkov photons produced in a single event are shown for the three event topologies with total kinetic energy of 2.529~MeV that corresponds to Q-value of $\Te$. One can try to guess the event topology by comparing different segments of the detector sphere.

\begin{figure*}[h]
  \centering
  \includegraphics[width=0.3\textwidth]{hDisplay_topology180_2p529MeVTot}
  \includegraphics[width=0.3\textwidth]{hDisplay_topology90_2p529MeVTot}
  \includegraphics[width=0.3\textwidth]{hDisplay_1el_2p529MeV}
  \caption{Cherenkov photons distributions on the detector sphere for
    the three representative event topologies: two back-to-back 1.26~MeV
    electrons (\emph{left}), two 1.26~MeV electrons at 90$^{\circ}$
    angle (\emph{middle}), and a single 2.529~MeV electron
    (\emph{center}).  All electrons originate at the center of the
    detector. One randomly selected event is chosen for each
    category. Default QE is applied.}
  \label{fig:Display_top_2p5MeV}
\end{figure*}

More realistic examples of $\Te$ $\vbb$-decay and $\B$ events simulated at the center of the detector are shown in Fig.~\ref{fig:Display_Te130}. Early PEs from Cherenkov and scintillation light are shown. Default simulation QE is applied. Time cut of 33.5~ns on the photon arrival time is used to select early PEs. Uniformly distributed scintillation light make it more difficult to guess the event topology. Nevertheless we show that there is still sufficient difference in the spatial distribution of the early PEs to separate two track and single track events.

\begin{figure*}[h]
  \centering
  \includegraphics[width=0.45\textwidth]{hDisplay_Te130_evt124_e1257_e1270_cos-0908}
  \includegraphics[width=0.45\textwidth]{hDisplay_Te130_evt131_e1264_e1263_cos-0029}
  \includegraphics[width=0.45\textwidth]{hDisplay_Te130_evt352_e1186_e1340_cos0888}
  \caption{Examples of PEs position on the detector sphere after time
    cut of 33.5ns. PEs from Cherenkov (\emph{red}) and scintillation
    light (\emph{blue}) are compared. \emph{Top left:} $^{130}$Te
    0{\nbb} decay back-to-back electrons: $E_1$=1.257~MeV,
    $E_2$=1.270~MeV, cos($\theta$)=-0.908. \emph{Top right:}
    $^{130}$Te 0{\nbb} decay electrons at $\sim$90$^{\circ}$:
    $E_1$=1.264~MeV, $E_2$=1.263~MeV,
    cos($\theta$)=-0.029. \emph{Bottom left:} $^{130}$Te 0{\nbb} decay
    electrons at $\sim$0$^{\circ}$: $E_1$=1.186~MeV, $E_2$=1.340~MeV,
    cos($\theta$)=0.888. \emph{Bottom right:} 2.529~MeV single
    electron. Events are simulated at the center of the
    detector. Default QE is applied.}
\label{fig:Display_Te130}
\end{figure*}


%$\vbb$-decay events become indistinguishable from single track topology when the angle between two electrons is small
For quantitative description of the difference in the event topology we analyze spherical harmonics of the photon distributions on the detector sphere. We construct rotation invariant variables and compare them between signal and background events. As it is shown in the bottom part of Fig.~\ref{fig:Display_Te130} $\vbb$-decay become indistinguishable from single track topology when the angle between two electrons is small (two degenerate tracks). Event topologies of $\vbb$-decay and $\B$ events are also very similar when only one electron from $\vbb$-decay is above the Cherenkov threshold. Therefore spherical harmonics analysis is most efficient for events with large angular separation between the two electrons and when both electrons are above Cherenkov threshold. 

In this paper we focus on topological difference between two tracks and single track events and do not make any attempt to use absolute directional information to suppress single track events  where direction of the track is consistent with the direction of solar neutrinos. Once a single track topology is established one can use a centroid method (see Ref.~\cite{Directionality}) to reconstruct directionality of the track (or two degenerate tracks) and suppress events that are aligned with the direction of $\B$ solar neutrinos.

\subsection{Description of Spherical Harmonics Analysis}
A function $f(\theta,\phi)$ can be decomposed to a sum of spherical harmonics:

\begin{eqnarray}
\label{eq1}
f(\theta,\phi) = \sum_{l=0}^{\infty} \sum_{m=-l}^{l} f_{lm} Y_{lm}(\theta,\phi),
\end{eqnarray}

where $Y_{lm}$ are Laplace's spherical harmonics defined in real-value basis using Legendre polynomials $P_l$:

\begin{eqnarray}
\label{eq2}
Y_{lm} = LONGformulaHERE,
\end{eqnarray}

 where  coefficients $f_{lm}$ are defined as
 
\begin{eqnarray}
\label{eq3}
f_{lm} = LONGformulaHERE.
\end{eqnarray}

Equation~\ref{eq4} defines power spectrum of $f(\theta,\phi)$ in spherical harmonics representation, $s_l$, where $l$ is a multiple moment. The power spectrum $s_l$ is invariant under rotation. It is unique to each of the functions $f_i(\theta,\phi)$, $i=$1,2,3..., that can not be transformed into each other by rotation.

\begin{eqnarray}
\label{eq4}
s_l = \sum_{m=-l}^{m=l} |f_{lm}|^2
\end{eqnarray}

One can consider PEs distribution for each of $\vbb$-decay signal or background event as a function $f_i(\theta,\phi)$. Events with similar power spectrum would correspond to PE distributions on the detector sphere that can be closely aligned by a rotation. Such PE distributions belong to events with similar topology.

Topology of $\vbb$-decay signal or background in a spherical detector determines the distribution of the PE's on the detector sphere and therefore a set of $s_l$'s can serve as a quantitative figure of merit for different event topologies. Rotation invariance of $s_l$'s ensures that this figure of merit does not depend on the orientation of the event with respect to the chosen coordinate frame.


Sum of $s_l$'s over all multiple moments equals to L2 norm of the function $f(\theta,\phi)$:

\begin{eqnarray}
\label{eq5}
\sum_{l=0}^{\infty} s_l = \int_{\Omega} |f(\theta,\phi)|^2 d\Omega.
\end{eqnarray}

Therefore normalized power spectrum 

\begin{eqnarray}
\label{eq5}
S_l = \frac{s_l}{\sum_{l=0}^{\infty} s_l} =  \frac{s_l}{\int_{\Omega} |f(\theta,\phi)|^2 d\Omega}
\end{eqnarray}

can be used to compare shapes of various functions $f(\theta,\phi)$ with different normalization. The total number of PEs detected on the detector sphere fluctuates from event to event, therefore in all of the following we use normalized power $S_l$.

Figure~\ref{fig:Moments} compares normalized power spectrum for the three representative event topologies that have been already shown in Fig.~\ref{fig:Display_top_5MeV}. We note that most of the information contains in the power spectrum with $l<$6. In most cases we found that there is no need to calculate $S_l$ for $l>$3 to achieve maximal separation between $\vbb$-decay and $\B$ events, because fluctuations in the PE distribution produce a lot of noise in the power spectrum for higher orders of multiple moments.


\begin{figure*}[h]
  \centering
  \includegraphics[width=0.95\textwidth]{Multiple_moment.JPG}
  \caption{Average $S_l$ values for two electrons at 180 degree
    (\emph{color1}) and 90 degree (\emph{color2}) 1.5~MeV each and a
    single electron (\emph{color3}) with the energy of 3~MeV. Error
    bars are RMS values of each corresponding individual $S_l$
    distribution (each consists of 1000 events simulated at the center
    of the detector) indicating typical event-by-event variation.}
\label{fig:Moments}
\end{figure*}


\subsection{Spherical Harmonics Analysis and Off-center Events}

In order to compare spherical harmonics for events with vertices located off-center anywhere inside the detector volume a coordinate transformation for each photon hit is needed. The necessary transformation applied for each PE within an event is illustrated in Fig.~\ref{fig:SphH_transform}. Solid circle with radius R schematically shows actual detector boundaries. Dotted circle shows a new sphere with the same radius R, which now has the event vertex in its center. The radius vector of each PE is stretched or shorten to its intersection with this new sphere using transformation $\vec{r}^{,}_{PE} = \frac{\vec{a}}{|\vec{a}|} \cdot R$. Where $\vec{r}^{,}_{PE}$ is a new radius vector of a PE and $\vec{a}=\vec{r}_{PE} - \vec{r}_{vtx}$ with $\vec{r}_{PE}$ and $\vec{r}_{vtx}$ being radius vectors of the PE and the vertex in the original coordinates respectively.

\begin{figure*}[h]
  \centering
  \includegraphics[width=0.95\textwidth]{SphH_transform_sketch.JPG}
  \caption{Coordinate transformation applied to events that are
    off-center. Solid circle schematically shows actual detector
    boundaries. Dotted circle shows a new sphere of radius R$=$6.5~m
    with the event vertex position in the center. The radius vector of
    each photon hit is stretched or shorten until intersection with
    this new sphere using transformation $\vec{r}^{,}_{hit} =
    \frac{\vec{a}}{|\vec{a}|} \cdot R$. Where $\vec{r}^{,}_{hit}$ is a
    new radius vector of the photon hit, $R$ is detector sphere radius,
    and $\vec{a}=\vec{r}_{hit} - \vec{r}_{vtx}$ with $\vec{r}_{hit}$
    and $\vec{r}_{vtx}$ being radius vectors of the photon hit and
    vertex position in original coordinates and correspondingly.}
  \label{fig:SphH_transform}
\end{figure*}


\subsection{Implementation of the spherical harmonics analysis}
{\bf A few words on the implementation. Calculation of $S_l$'s requires numerical integration that needs to be explained.}

To illustrate spherical harmonics analysis technique we compare distributions of $S_0$, $S_1$, $S_2$, and $S_3$ for the three representative event topologies described in Sec.~\ref{subsec:topology}. Almost all the information about event topology is carried by Cherenkov light. Therefore we first show spherical harmonics for back-to-back,  90$^{\circ}$ and single track topologies based on Cherenkov PEs only (see Fig.~\ref{fig:SL_topologies_CHE}).

Two top panels of Fig.~\ref{fig:SL_topologies_CHE} show 2-dimensional distributions, S0 vs S1 and S2 vs S3, to demonstrate that all four $S_l$'s provide separation between event topologies. No QE is applied in simulation of these events. We also introduce a 1-dimensional variable, S01 (bottom panel of Fig.~\ref{fig:SL_topologies_CHE}), that has the best separation power for majority of event topologies considered in this paper. $S_{01}$ is defined as a projection of S$_1$ vs S$_2$ distribution onto a linear fit of this 2-D distribution.

\begin{figure*}[h]
  \centering
  \includegraphics[width=0.49\textwidth]{ALL/hS0vsS1_topologies_CHELight_VtxSmear0cm_VtxShiftX0cm_33p5ns_center.pdf}
  \includegraphics[width=0.49\textwidth]{ALL/hS2vsS3_topologies_CHELight_VtxSmear0cm_VtxShiftX0cm_33p5ns_center.pdf}
  \includegraphics[width=0.9\textwidth]{ALL/hS01_topologies_CHELight_VtxSmear0cm_VtxShiftX0cm_33p5ns_center.pdf}
  \caption{Spherical harmonics for three event topologies: two
    back-to-back 1.26~MeV electrons (\emph{black squares and black
      dotted line}), two 1.26~MeV electrons at 90$^{\circ}$ angle
    (\emph{blue triangles and blue dashed line}), and a single
    2.529~MeV electron representing $^{8}$B background (\emph{red
      crosses and red solid line}). Simulation of 1000 events
    originated at the center of the sphere. Perfect separation between
    Cherenkov and scintillation light is implemented in this
    simulation by using only Cherenkov photons. \emph{Top left:} $S_0$
    versus $S_1$ scatter plot. Black dotted line is a linear fit of
    the 90$^{\circ}$ topology and $^{8}$B events. Variable $S_{01}$ is
    defined as a projection of 2D distribution onto this linear
    fit. \emph{Top right:} $S_2$ versus $S_3$ scatter
    plot. \emph{Bottom:} $S_{01}$ distributions for the three
    topologies. These distributions are normalized to unit area for
    shape comparison.}
  \label{fig:SL_topologies_CHE}
\end{figure*}


The effects due to presence of scintillation light and applying default QE are shown in Fig.~\ref{fig:SL_topologies_all}. Spherical harmonics of the same three representative event topologies are now calculated using early light (photons with arrival time less than 33.5~ns) that contains both directional Cherenkov light and uniform scintillation light. Default QE is also applied. Higher order multiple moments, S2 and S3, no longer provide noticeable separation between different event topologies.


\begin{figure*}[h]
  \centering
  \includegraphics[width=0.49\textwidth]{hS0vsS1_topologies_allLight_VtxSmear0cm_VtxShiftX0cm_33p5ns_center.pdf}
  \includegraphics[width=0.49\textwidth]{hS2vsS3_topologies_allLight_VtxSmear0cm_VtxShiftX0cm_33p5ns_center.pdf}
  \includegraphics[width=0.9\textwidth]{hS01_topologies_allLight_VtxSmear0cm_VtxShiftX0cm_33p5ns_center.pdf}
  \caption{Spherical harmonics for three event topologies: two
    back-to-back 1.26~MeV electrons (\emph{black squares and black
      dotted line}), two 1.26~MeV electrons at 90$^{\circ}$ angle
    (\emph{blue triangles and blue dashed line}), and a single
    2.529~MeV electron representing $^{8}$B background (\emph{red
      crosses and red solid line}). Simulation of 1000 events
    originated at the center of the sphere. Separation between
    Cherenkov and scintillation light is implemented 33.5~ns cut on
    the photon arrival time. Perfect vertex reconstruction - true
    vertex position is used. \emph{Top left:} $S_0$ versus $S_1$
    scatter plot. Black dotted line is a linear fit of the
    90$^{\circ}$ topology and $^{8}$B events. Variable $S_{01}$ is
    defined as a projection of 2D distribution onto this linear
    fit. \emph{Top right:} $S_2$ versus $S_3$ scatter
    plot. \emph{Bottom:} $S_{01}$ distributions for the three
    topologies. These distributions are normalized to unit area for
    shape comparison}
\label{fig:SL_topologies_all}
\end{figure*}


\section{Performance and Experimental Challenges}
\label{sec:performance_and_challenges}

\subsection{Performance of the spherical harmonics analysis on 0{\nbb} decay and $^{8}$B events.}

Comparison of $S_0$ and $S_1$ distributions between 0{\nbb} decay and
$^{8}$B events is shown in Fig.~\ref{fig:S_vs_energy}. There is a
noticeable separation between the signal and background. We also note
that in the energy range of interest $S_l$'s do not have strong
dependence on the energy deposited in the detector, which makes them
reliable discriminators at the end point of the 0{\nbb} decay energy
spectrum. The information about the event topology is complimentary to
the energy measurements.

\begin{figure*}[h]
\centering
\includegraphics[width=0.49\textwidth]{hS0.pdf}
\includegraphics[width=0.49\textwidth]{hS1.pdf}
\caption{$S_0$ (\emph{left}) and $S_1$ (\emph{right}) distributions
  for events with two different event topologies and total kinetic
  energy. $^{130}$Te, $^{82}$Se 0{\nbb} decay, 2.529 MeV and 2.995 MeV
  events are compared. The simulation is done for events with the
  vertex in the center of the detector. $^{8}$B events are implemented
  as 2.529~MeV or 2.995~MeV electrons with initial direction along
  $x$-axis. Perfect vertex reconstruction - true vertex position is
  used. Time cut of 33.5~ns on the photon arrival time is applied.}
\label{fig:S_vs_energy}
\end{figure*}

Figure~\ref{fig:SL_Te_33p5ns_center} shows separation between
$^{130}$Te signal and $^{8}$B background events simulated at the
center of the detector. True values of vertex position and time is
used. Time cut of 33.5~ns on the photon arrival time is applied to
separate Cherenkov and scintillation light. Most of the discrimination
between signal and background comes from $S_0$ and $S_1$. In the
following $S_2$ and $S_3$ are not used to separate $^{130}$Te and
$^{8}$B events\footnote{$S_2$ and $S_3$ are helpful for separation of
  $^{130}$Te signal from $^{10}$C background. See Appendix.}. The
scatter plot of $S_2$ vs $S_3$ is shown here for completeness.

In order to optimize separation between $^{130}$Te signal and $^{8}$B
background a linear combination of $S_0$ and $S_1$, $S_{01}$, is
used. A linear fit, $S_0$ = $A \times S_1 + B$, of 2-dimensional $S_0$
vs $S_1$ scatter plot is performed as shown in
Fig.~\ref{fig:SL_Te_33p5ns_center}. Then this 2-dimensional
distribution is projected onto the fitted line. {\bf A little bit of
  math here to quantitatively describe $S_{01}$ via $S_0$ and $S_1$:}
A new coordinate frame is obtained by rotation of the original
$S_0$-$S_1$ frame at angle $\theta$ obtained from the fit:
$tan(\theta)$=$A$. A transformation, $S_{01} = S_1 \cdot cos(\theta) +
S_0 \cdot sin(\theta)$, defines the $S_{01}$ variable.

Bottom plot in Fig.~\ref{fig:SL_Te_33p5ns_center} shows performance of
the $S_{01}$ variable to separate $^{130}$Te signal and $^{8}$B
background. A fit to this distribution can be done to optimize the
discrimination power in a particular experimental settings. Here we
refrain from quantitative estimates on the improvements in sensitivity
to 0{\nbb} decay search using this method of spherical harmonics as a
reliable estimate would require a dedicated analysis taking into
account all the details of a particular experiment.

\begin{figure*}[h]
  \centering
  \includegraphics[width=0.49\textwidth]{hS0vsS1_Te130_1el_allLight_VtxSmear0cm_VtxShiftX0cm_33p5ns_center.pdf}
  \includegraphics[width=0.49\textwidth]{hS2vsS3_Te130_1el_allLight_VtxSmear0cm_VtxShiftX0cm_33p5ns_center.pdf}
  \includegraphics[width=0.9\textwidth]{hS01_allLight_VtxSmear0cm_VtxShiftX0cm_33p5ns_center.pdf}
  \caption{Spherical harmonics comparison between $^{130}$Te 0{\nbb}
    decay signal ($Q=2.529$~MeV) (\emph{red}) and $^{8}$B solar
    neutrinos background (\emph{blue}) for 1000 simulated events
    originated at the center of the sphere. $^{8}$B events are
    implemented as 2.529~MeV electrons with initial direction along
    $x$-axis. Perfect vertex reconstruction - true vertex position is
    used. Time cut of 33.5~ns on the photon arrival time is
    applied. \emph{Top left:} $S_0$ versus $S_1$ scatter plot. Black
    dotted line is a linear fit of these 2D histograms. Variable
    $S_{01}$ is defined as a projection of 2D distribution onto this
    linear fit. \emph{Top right:} $S_2$ versus $S_3$ scatter
    plot. \emph{Bottom:} $S_{01}$ distribution for the signal and
    background.}
\label{fig:SL_Te_33p5ns_center}
\end{figure*}


\subsection{Experimental challenges}

So far only events at the center of the detector have been
considered. In this section we discuss performance of the spherical
harmonics analysis for events distributed within the fiducial volume
of the detector taking into account finite resolution on vertex
position reconstruction.

The absolute cut (e.g., 33.5~ns) on the photon arrival time for central events relies on the fact that within the uncertainty on electron track length all photons travel equal distance before they reach the surface of the detector. PEs with early measured time correspond mostly to Cherenkov photons because of the delay in the scintillation process and longer wavelength of the Cherenkov light. Early 

When the vertex is not at the center, a uniform absolute time cut on the photon
arrival time is no longer effective in the selection of Cherenkov
photons. In the case of off-center vertex, there could be a situation when even significantly delayed scintillation photons reach the side of the detector that is
closer to the vertex much earlier than Cherenkov photons traveling to
the opposite side of the detector. Therefore, the time cut has to be
position dependent and take into account the total distance traveled
by each individual photon.

We found that a relative time cut defined as $\Delta t=t^{phot}_{measured} -
t^{phot}_{predicted}<$1~ns selects photons with sufficient fraction of
Cherenkov photons. However two factors reduce Cherenkov/scintillation light separation when this relative time cut is applied to select early PEs compared to the absolute time cut.

Predicted time, $ t^{phot}_{predicted}=l/v^{phot}$, depends on total distance, $l$, traveled by the photon and velocity of the photon, $v^{phot}$. One reduction in the light separation comes from chromatic dispersion. Since the wavelength information is not available for any PE, we use an average index of refraction of n=1.53 and define photon velocity as $v^{phot} = c/n$. This uncertainty on the photon velocity due to chromatic dispersion reduces the separation between scintillation and Cherenkov light. Additional reduction in the light separation is caused by uncertainty on the total distance traveled by the photon caused by finite vertex resolution.

When the whole fiducial volume of the detector is considered only relative time cut can be used. Therefore effectiveness of the spherical harmonics analysis in separating of 0{\nbb} decay and $^{8}$B events is reduced. To demonstrate performance of the spherical harmonics analysis in a  more realistic experimental settings we define fiducial volume within our detector as $R<3$~m and  simulate $\Te$ $\vbb$ signal and $\B$ background events with verticies uniformly distributed  within that fiducial volume.

Figure~\ref{fig:SL_Te_SmearX0cm_momDT1ns_rndVtx_3p0m} compare spherical harmonics for the signal and background events within the fiducial volume. Perfect vertex reconstruction is used so that only effect due to chromatic dispersion is shown in Fig.~\ref{fig:SL_Te_SmearX0cm_momDT1ns_rndVtx_3p0m} (to be compared with central events shown in Fig.~\ref{fig:SL_Te_33p5ns_center}). Next generation LS detectors may be able to recover theses losses due to chromatic dispersion by choosing liquid scintillators with a more narrow
emission spectrum (e.g. see Ref~\cite{LS_narrow_emission}).


\begin{figure*}[h]
  \centering
  \includegraphics[width=0.49\textwidth]{hS0vsS1_Te130_1el_allLight_VtxSmear0cm_VtxShiftX0cm_momDT1p0ns_rndVtx_3p0mSphere.pdf}
  \includegraphics[width=0.49\textwidth]{hS01_allLight_VtxSmear0cm_VtxShiftX0cm_momDT1p0ns_rndVtx_3p0mSphere.pdf}
  \caption{Spherical harmonics comparison between $^{130}$Te 0{\nbb}
    decay signal ($Q=2.529$~MeV) (\emph{red}) and $^{8}$B solar
    neutrinos background (\emph{blue}) for 1000 simulated
    events.Verticies are uniformly distributed within the fiducial
    volume, $R<3$~m. $^8$Be events are implemented as 2.529~MeV
    electrons with the initial momentum direction uniformly
    distributed within 4$\pi$ solid angle. Perfect vertex
    reconstruction - true vertex position is used. \emph{Left:} $S_0$
    versus $S_1$ scatter plot. Black dotted line is a linear fit of
    these 2D histograms. Variable $S_{01}$ is defined as a projection
    of 2D distribution onto this linear fit. \emph{Right:} $S_{01}$}
  \label{fig:SL_Te_SmearX0cm_momDT1ns_rndVtx_3p0m}
\end{figure*}


Imprecise knowledge of the vertex position due to finite resolution is
the second factor affecting performance of the spherical harmonics
analysis. The vertex resolution not only reduces Cherenkov/scintillation light separation but it also affects uniformity of the scintillation light distribution in the early PE sample.

Small deviations in vertex reconstruction cause large effect
on $S_0$ and $S_1$ for single electron event topology.
For the verticies shifted along the direction of the electron track the relative time cut $\Delta t$
makes uniform scintillation light distribution less uniform. The
$\Delta t$ cut selects more forward emitted photons in the case when
the reconstructed vertex is shifted to the direction opposite to the
electron momentum (enhancing forward region populated by Cherenkov
photons - more asymmetric photon distribution causing higher values of
$S_1$). It selects more backward emitted photons in the case when the
reconstructed vertex is shifted in the direction along the electron
momentum (counter balancing forward region populated by Cherenkov
photons - more symmetric photon distribution causing smaller values of
$S_1$).

Figure~\ref{fig:SL_Te_SmearX3cm_momDT1ns_rndVtx_3p0m} shows performance of the spherical harmonics analysis for events with verticies smeared with 3~cm resolution and uniformly distributed within the fiducial volume. Since the goal of this paper is to introduce spherical harmonic analysis we do not perform full vertex reconstruction and only apply smearing to the simulated vertex position using Gaussian distribution with sigma of 3~cm on $x$, $y$, and $z$ coordinates. 

\begin{figure*}[h]
  \centering
  \includegraphics[width=0.49\textwidth]{hS0vsS1_Te130_1el_allLight_VtxSmear3cm_VtxShiftX0cm_momDT1p0ns_rndVtx_3p0mSphere.pdf}
  \includegraphics[width=0.49\textwidth]{hS01_allLight_VtxSmear3cm_VtxShiftX0cm_momDT1p0ns_rndVtx_3p0mSphere.pdf}
  \caption{Spherical harmonics comparison between $^{130}$Te 0{\nbb}
    decay signal ($Q=2.529$~MeV) (\emph{red}) and $^{8}$B solar
    neutrinos background (\emph{blue}) for 1000 simulated
    events.Verticies are uniformly distributed within the fiducial
    volume, R$<$3~m. $^8$Be events are implemented as 2.529~MeV
    electrons with the initial momentum direction uniformly
    distributed within 4$\pi$ solid angle. Vetrex is smeared with 3~cm
    resolution. \emph{Left:} $S_0$ versus $S_1$ scatter plot. Black
    dotted line is a linear fit of these 2D histograms. Variable
    $S_{01}$ is defined as a projection of 2D distribution onto this
    linear fit. \emph{Right:} $S_{01}$}
\label{fig:SL_Te_SmearX3cm_momDT1ns_rndVtx_3p0m}
\end{figure*}


{\bf Solution to this problem would be a better selection criteria of
  early light. It has to preserve high admixture of the Cherenkov
  photons, but needs to select scintillation photons in a more uniform
  manner. Working on it, but may not be simple so I don't want to
  include it in this paper.}

Good vertex resolution is essential for spherical harmonics analysis. However, as one can see in Fig.~\ref{fig:SL_Te_SmearX3cm_momDT1ns_rndVtx_3p0m}, even 3~cm vertex resolution reduces the discrimination power of the spherical harmonics analysis.

Such strong dependence on the vertex resolution can be
addressed by choosing a different liquid scintillator mixture with a
more delayed emission of the scintillation light with respect to the Cherenkov light. With larger delay in the scintillation light, larger uncertainty on the photon predicted time is allowed to maintain high fraction of Cherenkov light in the early PE sample selected with relative time cut $\Delta t$. In addition, if the fraction of scintillation light is small compared to Cherenkov light, the distortions in the uniformity of the scintillation light distribution due to mis-reconstructed vertex do not significantly affect $S_0$ and $S_1$ variables.

Figure~\ref{fig:SL_Te_momDT1ns_sci0p5ns_rndVtx_3p0m} shows
spherical harmonics analysis for the simulation where the
scintillation component is delayed by additional 0.5ns compared to our default simulation. Events are simulated uniformly within the fiducial volume of the detector. Vertex resolution of 3~cm is assumed. Noticeable separation between $\vbb$ and $\B$ events is achieved.

The discrimination power of the spherical harmonics analysis improves with vertex resolution and more delay in the emission of the scintillation light. Moreover the dependence on vertex reconstruction reduces with delay in the scintillation light. 

\begin{figure*}[h]
  \centering
  \includegraphics[width=0.49\textwidth]{hS0vsS1_Te130_1el_allLight_VtxSmear3cm_VtxShiftX0cm_momDT1p0ns_sci0p5ns_rndVtx_3p0mSphere.pdf}
  \includegraphics[width=0.49\textwidth]{hS01_allLight_VtxSmear3cm_VtxShiftX0cm_momDT1p0ns_sci0p5ns_rndVtx_3p0mSphere.pdf}
  \caption{Spherical harmonics comparison between $^{130}$Te 0{\nbb}
    decay signal ($Q=2.529$~MeV) (\emph{red}) and $^{8}$B solar
    neutrinos background (\emph{blue}) for 1000 simulated
    events. Verticies are uniformly distributed within the fiducial
    volume, $R<3$~m. $^8$Be events are implemented as 2.529~MeV
    electrons with the initial momentum direction uniformly
    distributed within 4$\pi$ solid angle. Vetrex is smeared with 3~cm
    resolution. {\bf Scintillation light is delayed by additional
      0.5~ns.} \emph{Left:} $S_0$ versus $S_1$ scatter plot. Black dotted
    line is a linear fit of these 2D histograms. Variable $S_{01}$ is
    defined as a projection of 2D distribution onto this linear
    fit. \emph{Right:} $S_{01}$}
\label{fig:SL_Te_SmearX3cm_momDT1ns_sci0p5ns_rndVtx_3p0m}
\end{figure*}


%\clearpage
\section{Conclusions}
\label{sec:conclusions}
We consider the use of large-area photodetectors with good time and
space resolution in kiloton scale liquid scintillator detectors to
suppress background coming from $^{8}$B solar neutrino
interactions. Using a default model detector with parameters derived
from present practice we show that a sample of detected photons
enriched in Cherenkov light by a cut on time-of-arrival contains
directional information can be used to separate 0{\nbb}~ decay from
$^{8}$B solar neutrino interactions. The separation is based on a
spherical harmonics analysis of the event topologies of the
two electrons in signal events and the single electron in the
background. The performance of the technique is constrained by
chromatic dispersion, vertex reconstruction, and the time profile of
the emission of scintillation light. The development of a scintillator
with a rise time constant of at least 5~ns would allow a
Cherenkov-scintillation light separation with a background rejection
factor for \B~ solar neutrinos of xx and an efficiency for 0\nbb~ signal
of xx\%.



\section*{Acknowledgements}
This work was supported by DOE grant number A, and NSF grant number B.

We thank Jenni Kotilla of Yale for providing data with phase factors for generating 0\nbb-decay events.
We are greateful to Christoph Aberle, formely at UCLA, for initial development of Geant-4 
detector model used at this paper. We thank Carla Pilcher of 
the University of Chicago for discussions on gammas interactions in liquid scintillators.
We thank Davide Franco of CNRS and Paolo Crivelli of ETH, Zurich for sharing their experience with ortho-positronium 
simulation in Geant-4. 


%Begin Comment here
%\begin{comment}
% bugger the bibliography for now.. HJF
%\end{comment}

\clearpage

\appendix
\renewcommand*{\thesection}{\Alph{section}}
\section{Appendix A}
\label{Appendix_A}

\subsection{Defining the Power Spectrum}

Let the function $f(\theta,\phi)$ represent the distribution of the
photo-electrons (PE) on the detector surface. The function
$f(\theta,\phi)$ can be decomposed into a sum of spherical harmonics:

\begin{eqnarray}
\label{eq1}
f(\theta,\phi) = \sum_{\ell=0}^{\infty} \sum_{m=-\ell}^{\ell} f_{\ell m} Y_{\ell m}(\theta,\phi),
\end{eqnarray}

where $Y_{\ell m}$ are Laplace's spherical harmonics defined in a
real-value basis using Legendre polynomials $P_{\ell}$~\cite{legendre_polynomials}:

\begin{eqnarray}
\label{eq2}
Y_{\ell m} = \left\{
  \begin{array}{@{}ll@{}}
    \sqrt{2}N_{\ell m}P_{\ell}^m(\cos\theta)\cos~m\phi, & \text{if}\ m>0 \\
    N_{\ell m} = \sqrt{\frac{(2\ell+1)}{4\pi} \frac{(\ell-m)!}{(\ell+m)!}}, & \text{if}\ m=0 \\
    \sqrt{2}N_{\ell |m|}P_{\ell}^{|m|}(\cos\theta)\sin~|m|\phi, & \text{if}\ m<0
  \end{array}\right.
\end{eqnarray}

where the coefficients $f_{\ell m}$ are defined as
 
\begin{eqnarray}
\label{eq3}
f_{\ell m} = \int_{0}^{2\pi} d\phi \int_0^{\pi} d\theta \sin\theta f(\theta,\phi) Y_{\ell m}(\theta,\phi).
\end{eqnarray}

Equation~\ref{eq4} defines the power spectrum of $f(\theta,\phi)$ in
the spherical harmonics representation, $s_{\ell}$, where $l$ is a multipole
moment. The power spectrum, $s_{\ell}$, is invariant under rotation. 
%It is
%unique to each of the functions $f_i(\theta,\phi)$, $i=$1,2,3...,
%which cannot be transformed into each other by rotation.

\begin{eqnarray}
\label{eq4}
s_{\ell} = \sum_{m=-\ell}^{m=\ell} |f_{\ell m}|^2
\end{eqnarray}

The event topology in a spherical detector determines the distribution
of the PE's on the detector sphere, and, therefore, a set of
$s_{\ell}$'s. These values can serve as a quantitative figure of merit for
different event topologies. The rotation invariance of the $s_{\ell}$'s ensures
that this figure of merit does not depend on the orientation of the
event with respect to the chosen coordinate frame.

The sum of $s_{\ell}$'s over all multipole moments equals to the $L^2$ norm of the
function $f(\theta,\phi)$:

\begin{eqnarray}
\label{eq5}
\sum_{\ell=0}^{\infty} s_{\ell} = \int_{\Omega} |f(\theta,\phi)|^2 d\Omega.
\end{eqnarray}

The normalized power spectrum is thus:

\begin{eqnarray}
\label{eq6}
\mathcal{S}_{\ell} = \frac{s_{\ell}}{\sum_{\ell=0}^{\infty} s_{\ell}} =  \frac{s_{\ell}}{\int_{\Omega} |f(\theta,\phi)|^2 d\Omega},
\end{eqnarray}

and can be used to compare the shapes of various functions
$f(\theta,\phi)$ with different normalizations. As the total number of
PEs detected on the detector sphere fluctuates from event to event we
use the normalized power $\mathcal{S}_{\ell}$.


\subsection{Spherical Harmonics Analysis and Off-center Events}

In general, events with the same event topology result in the same
the power spectrum $S_{\ell}$ only if events originate in the center 
of the detector. In
order to compare the spherical harmonics for events with verticies away
from the center, a coordinate transformation for each photon hit is
needed. The necessary transformation applied for each PE within an
event is illustrated in Fig.~\ref{fig:SphH_transform}.  The solid
circle in Fig.~\ref{fig:SphH_transform}~has a radius R and shows the
actual detector boundaries. The dotted circle shows a new sphere with
the same radius R, which now has the event vertex in its center. The
radius vector of each PE is stretched or shortened to its intersection
with this new sphere using the transformation, $\vec{r}^{,}_{PE} =
\frac{\vec{a}}{|\vec{a}|} \cdot R$, where $\vec{r}^{,}_{PE}$ is a new
radius vector of a PE and $\vec{a}=\vec{r}_{PE} - \vec{r}_{vtx}$ with
$\vec{r}_{PE}$ and $\vec{r}_{vtx}$ being radius vectors of the PE and
the vertex in the original coordinates, respectively.

\begin{figure*}[h]
  \centering
%  \includegraphics[width=0.95\textwidth]{SphH_transform_sketch.JPG}
  \includegraphics[width=0.4\textwidth]{SphH_transform.pdf}
  \caption{The coordinate transformation which is applied to events that are
    off-center. The solid circle schematically shows the actual detector
    boundaries. The dotted circle shows a new sphere of radius R$=$6.5~m
    with the event vertex position in the center. The radius vector of
    each photon hit is stretched or shortened until the intersection with
    this new sphere using the transformation $\vec{r}^{,}_{hit} =
    \frac{\vec{a}}{|\vec{a}|} \cdot R$, where $\vec{r}^{,}_{hit}$ is a
    new radius vector of the photon hit, $R$ is detector sphere radius,
    and $\vec{a}=\vec{r}_{hit} - \vec{r}_{vtx}$ with $\vec{r}_{hit}$
    and $\vec{r}_{vtx}$ being the radius vectors of the photon hit and
    vertex position in original coordinates, respectively.}
  \label{fig:SphH_transform}
\end{figure*}


%\subsection{Implementation of the spherical harmonics analysis}

%The numerical calculation of the power spectrum is implemented as follows.
%For each event, a 2-D histogram of  the distribution of PEs on the
%detector surface in $\theta$ vs $\phi$ is created. We then treat this
%histogram as a function $f(\theta,\phi)$, where the value of the
%function for any pair of $\theta$ and $\phi$ is equal to the number of
%PE's in the histogram bin corresponding to that pair.

%The coefficients $f_{\ell m}$ from Eq.~\ref{eq3} are calculated using a
%Monte Carlo integration technique. The values of the $S_{\ell}$ moments are
%calculated using Eqs.\ref{eq4} - \ref{eq6}. 
%{\bf Also need to provide
%reference to the libraries for Legendre polynomials.}




\clearpage

\bibliography{bibliography}

%\begin{thebibliography}{99}
%\input{biblio_intro}
%\input{biblio_detector}
%\input{biblio_eventtiming}
%\input{biblio_sphericalharmonics}
%\input{biblio_performance}
%\input{biblio_appendixA}
%\input{biblio_}
%\end{thebibliography}


\end{document}


%%%%%%%%%%%%%%%%%%%%%%%%%%%%%%%%%%%%%%%%%%%%%%%%%%%%%%%%%%%%%%%%%%%%%%%%%%%%%%%
%\bibliography{bibtmp}
%\appendix
% The following line is a hack for elsevier to get the appendix and
% the table of contents to play well together.
%\renewcommand*{\thesection}{\Alph{section}}
%\section{Timing of photons coming from $^{10}$C background}

Typical energy deposition by $^{10}$C events is shown in
Fig.~\ref{fig:Edep_C10}. Assuming $\sim$15\% energy resolution, events in 
the energy range of 2.1-2.9~MeV would contribute to the backgroung count in 
the ROI.


\begin{figure}[h]
  \centering
  \includegraphics[width=0.95\textwidth]{hEdep_C10.pdf}
  \caption{Energy deposition in $^{10}$C events.}
  \label{fig:Edep_C10}
\end{figure}

Since 98\% of $^{10}$C decays through an excited state of $^{10}$B(718), 
which has a half-life time of $\sim$1~ns, the majority of $^{10}$C events have 
a prompt positron accompanied by a delayed 0.718~MeV gamma. The positron energy
has to be 0.79~MeV for an event to have energy deposition equal to Q-value of
$\Te$ $\vbb$-decay.

The positron from $\Cten$ on average travels 4~mm before it stops and anihilates
producing two 0.511~MeV gammas. Those gammas then interact in the scintillator via 
Compton scattering and photo-electric effect along while they loose their energy
over $\sim X_0$ distance. Therefore light emmited in $\Cten$ events originates from
several clusters that are spread over $\sim X_0$. Significant fraction of the 
early PE would be due to primary positron cluster. Because the positron has smaller 
kinetic energy than kinetic energy of electrons from $\vbb$-decay the amount of early 
PE is smaller for $\Cten$ events. Delayed 0.718~MeV gamma in some of $\Cten$ events 
also result in a delay in the photon arrival time with respect to $\vbb$-decay events.

Figure~\ref{fig:Arrival_time_C10_overlaid} compares PE arrival times between 
$\vbb$-decay and $\Cten$ events. Prompt $\Cten$ is a simplified simulation where 
a positron is simultaneously produced with 0.718~MeV gamma. The difference between
prompt $\Cten$ and $\vbb$-decay is caused by presence of a positron and this 
difference is typical event by even basis. An additional difference due to delayed
gamma shown in Fig.~\ref{fig:Arrival_time_C10_overlaid} is an average over 1000 events.

\begin{figure}[h]
  \centering
  \includegraphics[width=0.95\textwidth]{hT_Te130vsC10_overlaid_v2.pdf}
  \caption{Photo-electron (PE) arrival times after application of the
    photo-detector transit time spread (TTS) of 100~ps for the
    simulation of 1000 0{\nbb} decay events of $^{130}$Te (\emph{solid
      lines}) and $^{10}$C (\emph{dotted lines}) events at the center
    of the detector. Cherenkov and scintillation components are are normalized for 
    for shape comparison.}
\label{fig:Arrival_time_C10_overlaid}
\end{figure}

To distinguish between $\vbb$-decay and $\Cten$ we count total number of PEs in the 
early light sample. For central events where we assume perfect knowledge of the 
primary vertex location the early light sample is defined as t$<$33.5~ns. For a more 
realistic scenario where the vertex is uniformly distributed within the fiducial volume
the early light sample is defined as $\Delta t=t^{phot}_{measured} - 
t^{phot}_{predicted}<$1~ns.


Number of Cherenkov and scintillation PEs in early light samples for $\vbb$-decay and 
$\Cten$ central events is shown in Fig.~\ref{fig:NPhotDist_C10}. Here a perfect 
reconstrution of the primary vertex is assumed. Figure~\ref{fig:NPhot_compare_central} shows
separation between $\vbb$-decay and $\Cten$ events by counting total number of PEs. 

\begin{figure*}[ht]
  \centering
  \includegraphics[width=0.45\textwidth]{hMomNPhot_Te130.pdf}
  \includegraphics[width=0.45\textwidth]{hMomNPhot_C10.pdf}
  \caption{Early photons. Number of Cherenkov (\emph{dashed red line}), 
    scintillation
    (\emph{dotted blue line}), and total (\emph{solid black line}) PEs
    for the simulation of 1000 $^{130}$Te 0{\nbb} decay (left panel)
    and of 648 $^{10}$C (\emph{right panel}) events (1000 $^{10}$C events was 
    generated, but selected only those that has total energy deposition in the 
    detector in the range between 2.1 and 2.9~MeV).}
\label{fig:NPhotDist_C10}
\end{figure*}



\begin{figure*}[ht]
  \centering
  \includegraphics[width=0.95\textwidth]{hMomNPhot_Te130vsC10_VtxSmear0cm_VtxShiftX0cm_33p5ns_center.pdf}
  \caption{Comparison of total number of early photons between $^{130}$Te 0{\nbb} decay 
    and $^{10}$C events with energy deposition in the range between 2.1 and 2.9~MeV. 
    Events originated at the center of the sphere.
    Perfect vertex reconstruction - true vertex position is used. Time cut of 
    33.5~ns on the photon arrival time is applied.}
\label{fig:NPhot_compare_central}
\end{figure*}


Figure~\ref{fig:NPhot_compare_rndVtx_noSmear} compares total number of PEs for events uniformly 
distributed within the fiducial volume. Perfect knowledge of the vertex is assumed.

\begin{figure*}[ht]
  \centering
  \includegraphics[width=0.45\textwidth]{hMomDT_Te130vsC10_VtxSmear0cm_VtxShiftX0cm_momDT1p0ns_rndVtx_3p0mSphere.pdf}
  \includegraphics[width=0.45\textwidth]{hMomNPhot_Te130vsC10_VtxSmear0cm_VtxShiftX0cm_momDT1p0ns_rndVtx_3p0mSphere.pdf}
  \caption{(Left) Difference between measured PE arrival time and arrival time prediction based on 
	vertex location (T$^{predicted} = |r_{hit} - r_{vtx}|/v_{phot}$, where $v_phot = c/1.53$).
        $\vbb$-decay (black solid line) and $\Cten$ events (magenta dashed line) are compared. 
	Vertical line at 1~ns indicates cut for early light selection. 
        (Right) Total number of PEs in the early light sample. 
        $^{10}$C events with energy deposition in the range between 2.1 and 2.9~MeV are
	selected. Verticies are uniformly distributed within the fiducial volume, $R<3$~m.
        {\bf Perfect vertex reconstruction - true vertex position is used.}}
\label{fig:NPhot_compare_rndVtx_noSmear}
\end{figure*}


Figure~\ref{fig:NPhot_compare_rndVtx_Smear3cm} compares total number of PEs for events uniformly
distributed within the fiducial volume and reconstructed vertex smeared with 3~cm resolution.

\begin{figure*}[ht]
  \centering
  \includegraphics[width=0.45\textwidth]{hMomDT_Te130vsC10_VtxSmear3cm_VtxShiftX0cm_momDT1p0ns_rndVtx_3p0mSphere.pdf}
  \includegraphics[width=0.45\textwidth]{hMomNPhot_Te130vsC10_VtxSmear3cm_VtxShiftX0cm_momDT1p0ns_rndVtx_3p0mSphere.pdf}
  \caption{(Left) Difference between measured PE arrival time and arrival time prediction based on
        vertex location (T$^{predicted} = |r_{hit} - r_{vtx}|/v_{phot}$, where $v_phot = c/1.53$).
        $\vbb$-decay (black solid line) and $\Cten$ events (magenta dashed line) are compared.
        Vertical line at 1~ns indicates cut for early light selection.
        (Right) Total number of PEs in the early light sample.
        $^{10}$C events with energy deposition in the range between 2.1 and 2.9~MeV are
        selected. Verticies are uniformly distributed within the fiducial volume, $R<3$~m.
        {\bf Vetrex is smeared with 3~cm resolution.}}
\label{fig:NPhot_compare_rndVtx_Smear3cm}
\end{figure*}


Figure~\ref{fig:NPhot_compare_rndVtx_Smear10cm} compares total number of PEs for events uniformly
distributed within the fiducial volume and reconstructed vertex smeared with 10~cm resolution.

\begin{figure*}[ht]
  \centering
  \includegraphics[width=0.45\textwidth]{hMomDT_Te130vsC10_VtxSmear10cm_VtxShiftX0cm_momDT1p0ns_rndVtx_3p0mSphere.pdf}
  \includegraphics[width=0.45\textwidth]{hMomNPhot_Te130vsC10_VtxSmear10cm_VtxShiftX0cm_momDT1p0ns_rndVtx_3p0mSphere.pdf}
  \caption{(Left) Difference between measured PE arrival time and arrival time prediction based on
        vertex location (T$^{predicted} = |r_{hit} - r_{vtx}|/v_{phot}$, where $v_phot = c/1.53$).
        $\vbb$-decay (black solid line) and $\Cten$ events (magenta dashed line) are compared.
        Vertical line at 1~ns indicates cut for early light selection.
        (Right) Total number of PEs in the early light sample.
        $^{10}$C events with energy deposition in the range between 2.1 and 2.9~MeV are
        selected. Verticies are uniformly distributed within the fiducial volume, $R<3$~m.
        {\bf Vetrex is smeared with 10~cm resolution.}}
\label{fig:NPhot_compare_rndVtx_Smear10cm}
\end{figure*}



Figure~\ref{fig:NPhot_compare_rndVtx_Smear30cm} compares total number of PEs for events uniformly
distributed within the fiducial volume and reconstructed vertex smeared with 30~cm resolution.

\begin{figure*}[ht]
  \centering
  \includegraphics[width=0.45\textwidth]{hMomDT_Te130vsC10_VtxSmear30cm_VtxShiftX0cm_momDT1p0ns_rndVtx_3p0mSphere.pdf}
  \includegraphics[width=0.45\textwidth]{hMomNPhot_Te130vsC10_VtxSmear30cm_VtxShiftX0cm_momDT1p0ns_rndVtx_3p0mSphere.pdf}
  \caption{(Left) Difference between measured PE arrival time and arrival time prediction based on
        vertex location (T$^{predicted} = |r_{hit} - r_{vtx}|/v_{phot}$, where $v_phot = c/1.53$).
        $\vbb$-decay (black solid line) and $\Cten$ events (magenta dashed line) are compared.
        Vertical line at 1~ns indicates cut for early light selection.
        (Right) Total number of PEs in the early light sample.
        $^{10}$C events with energy deposition in the range between 2.1 and 2.9~MeV are
        selected. Verticies are uniformly distributed within the fiducial volume, $R<3$~m.
        {\bf Vetrex is smeared with 30~cm resolution.}}
\label{fig:NPhot_compare_rndVtx_Smear30cm}
\end{figure*}


Figure~\ref{fig:NPhot_compare_rndVtx_Smear50cm} compares total number of PEs for events uniformly
distributed within the fiducial volume and reconstructed vertex smeared with 50~cm resolution.

\begin{figure*}[ht]
  \centering
  \includegraphics[width=0.45\textwidth]{hMomDT_Te130vsC10_VtxSmear50cm_VtxShiftX0cm_momDT1p0ns_rndVtx_3p0mSphere.pdf}
  \includegraphics[width=0.45\textwidth]{hMomNPhot_Te130vsC10_VtxSmear50cm_VtxShiftX0cm_momDT1p0ns_rndVtx_3p0mSphere.pdf}
  \caption{(Left) Difference between measured PE arrival time and arrival time prediction based on
        vertex location (T$^{predicted} = |r_{hit} - r_{vtx}|/v_{phot}$, where $v_phot = c/1.53$).
        $\vbb$-decay (black solid line) and $\Cten$ events (magenta dashed line) are compared.
        Vertical line at 1~ns indicates cut for early light selection.
        (Right) Total number of PEs in the early light sample.
        $^{10}$C events with energy deposition in the range between 2.1 and 2.9~MeV are
        selected. Verticies are uniformly distributed within the fiducial volume, $R<3$~m.
        {\bf Vetrex is smeared with 50~cm resolution.}}
\label{fig:NPhot_compare_rndVtx_Smear50cm}
\end{figure*}



% !!!!!!!!!!!!	Commented text begins	!!!!!!!!!!!!!!!!!!!!!
\begin{comment}
\newpage

\section{0{\nbb} decay vs $^{10}$C background}

Other common backgrounds to 0{\nbb} decay search include radioactive
decays of nuclei that are excited by cosmic muons and produced through the decays of Th and U
naturally present in the materials. In liquid scintillator detectors,
most of events from Th and U decays occur in the materials of
the scintillator enclosure. Typically, they enter the fiducial volume
as 2.6~MeV gammas. These gammas pass into the fiducial volume either because they showered too late or have
mis-reconstructed vertex. Both effects depend on details of a
particular experiment and in this paper we make no attempt
to introduce a topology reconstruction for the backgrounds coming from
Th and U lines. Cosmic induced backgrounds, to the contrary, are more
generic and originate inside the fiducial volume. In this section we
discuss event topology of $^{10}$C events that are most relevant in the
energy of 2-3~MeV.

Typical energy deposition by $^{10}$C events is shown in
Fig.~\ref{fig:Edep_C10}. We propose to use spherical harmonics
analysis to separate 0{\nbb} decay events from $^{10}$C events that
within energy resolution overlap with the 0{\nbb} decay Q-value.

\begin{figure}[h]
  \centering
  \includegraphics[width=0.95\textwidth]{hEdep_C10.pdf}
  \caption{Energy deposition in $^{10}$C events.}
  \label{fig:Edep_C10}
\end{figure}

\begin{figure}[h]
  \centering
  \includegraphics[width=0.45\textwidth]{hT_C10.pdf}
  \includegraphics[width=0.45\textwidth]{hTche_C10.pdf}
  \caption{Photo-electron (PE) arrival times after application of the
    photo-detector transit time spread (TTS) of 100~ps for the
    simulation of 1000 0{\nbb} decay events of $^{130}$Te (\emph{solid
      lines}) and $^{10}$C (\emph{dotted lines}) events at the center
    of the detector. All distributions are normalized for shape
    comparison. {\bf Absolute number of PEs per event depends on the
      total energy deposited in the
      detector. Figure~\ref{fig:Edep_C10} shows energy deposited in
      the detector in $^{10}$C events.} \emph{Left:} Scintillation PEs
    arrival time. The black vertical line illustrates a time cut at
    33.5 ns. \emph{Right:} Cherenkov PEs arrival time.}
\label{fig:Arrival_time_C10}
\end{figure}

We note that 98\% of $^{10}$C decays through the excited state of
$^{10}$B(718), which has a half-life time of $\sim$1~ns. Therefore, the
majority of $^{10}$C events have a prompt positron accompanied by a
delayed 0.718~MeV gamma. This delayed gamma affects the PE arrival time
distribution. Figure~\ref{fig:Arrival_time_C10} compares the shape of the
PE arrival time distribution between $^{130}$Te 0{\nbb} decays and
$^{10}$C events. The time profile of the scintillation photons can be used
to separate signal from $^{10}$C events.

\end{comment}

%	!!!!!!!!!!!!!!!!	Commented text ends 	!!!!!!!!!!!!!!!!!!

Comparison of $S_0$ and $S_1$ distributions between 0{\nbb} decay and
$^{10}$C events is shown in Fig.~\ref{fig:S_vs_energy_C10}.

\begin{figure*}[h]
\centering
\includegraphics[width=0.49\textwidth]{hS0_C10.pdf}
\includegraphics[width=0.49\textwidth]{hS1_C10.pdf}
\caption{$S_0$ (\emph{left}) and $S_1$ (\emph{right}) distributions
  for events with different event topologies. $^{130}$Te, $^{82}$Se 0{\nbb} 
  decays compared with $^{8}$B and $^{10}$C events. The simulation is done 
  for events with the vertex in the center of the detector. $^{8}$B events 
  are implemented as 2.529~MeV or 2.995~MeV electrons with initial direction 
  along $x$-axis. $^{10}$C events are selected in the energy range between 2.1 
  and 2.9~MeV. Perfect vertex reconstruction - true vertex position is used. 
  Time cut of 33.5~ns on the photon arrival time is applied.}
\label{fig:S_vs_energy}
\end{figure*}


\begin{figure}[h]
  \centering
  \includegraphics[width=0.95\textwidth]{hSLPlots_C10_allLight_VtxSmear0cm_VtxShiftX0cm_33p5ns_center.pdf}
  \caption{Spherical harmonics comparison between $^{130}$Te 0{\nbb}
    decay signal ($Q=2.529$~MeV) (\emph{red}) and $^{10}$C solar
    neutrinos background (blue) for 1000 simulated events originated
    at the center of the sphere. $^{10}$C with energy deposition
    between 2.1~MeV and 2.9~MeV are considered. Perfect vertex
    reconstruction - true vertex position is used. Time cut of 33.5~ns
    on the photon arrival time is applied. \emph{Top left:} S$_0$
    versus S$_1$ scatter plot. \emph{Top right:} S$_2$ versus S$_3$
    scatter plot. \emph{Bottom left:} Distribution of the
    S$^{C10}_{01}$ variable calculated for signal (\emph{red}) and
    background (\emph{green}). \emph{Bottom right:} Distribution of
    the S$^{C10}_{23}$ variable calculated for signal (\emph{red}) and
    background (\emph{green}).}
  \label{fig:SL_C10_33p5ns_center}
\end{figure}


Comparison of spherical harmonics is shown in
Fig.~\ref{fig:SL_C10_33p5ns_center}. $^{10}$C events are generated at
the center of the detector. True vertex position is used to apply a
33.5~ns time cut to select photons for the spherical harmonics
analysis. The separation is seen in S0 vs S1 and S2 vs S3 scatter
plots. We project both scatter plots to a line that gives maximum
separation (two bottom panels in Fig.~\ref{fig:SL_C10_33p5ns_center}).  
There is enough separation between the distributions to suggest that this analysis can be used to distinguish between 0{\nbb} and $^{10}$C events.

%\section{0{\nbb} decay vs backgrounds from Th and U series}



%\end{document}

