\section{Introduction}

The electron, muon, and tau neutrinos are unique among the standard
model fermions in being electrically neutral and orders-of-magnitude
less massive than their standard model charged
partners~\cite{PDG_mass}.  These two properties motivate the
possibility that these neutrinos are `Majorana' rather than `Dirac'
particles, i.e. different from their respective charged partner
leptons by being their own
anti-particle~\cite{Majorana1937,PDG_mass}. In 1939 W. Furry
pointed out that a Majorana nature of the electron neutrino would
allow neutrinoless double-beta decay, in which a nucleus undergoes a
second order $\beta$-decay without producing any neutrinos,
$(Z,A)\rightarrow(Z+2,A)+2\beta^-$~\cite{Furry1939}.  This is in
contrast to the Goeppert-Mayer two-neutrino double beta (2{\nbb})
decay, the second order standard model (SM) \bmd-decay channel in which
lepton number is conserved by the production of two anti-neutrinos,
\mbox{$(Z,A)\rightarrow(Z,A+2)+2\beta+2\bar\nu_e$}~\cite{GoeppertMayer1935}.

The standard mechanism of 0\nbb-decay is parametrized by the
effective Majorana mass, defined as
\mbox{$m_{\beta\beta}\equiv\left|\sum_i U^2_{ei}m_i\right|$}, where
$U_{ei}$ are the elements of the PMNS matrix and $m_i$ are the
neutrino masses~\cite{PDG_mass}. Current half-life limit
translate to a limit on \mbox{$m_{\beta\beta}\lesssim
61-165\,\mathrm{meV}$}~\cite{KamLANDZen2016}.  The next generation of 0\nbb-decay
experiments~\cite{NSACreport} seek to be sensitive enough to
detect or rule out 0\nbb-decay down to \mbox{$m_{\beta\beta}\lesssim
10$~meV}. This will require a detector to instrument roughly a ton of
active isotope with good energy resolution and a near zero background.

Liquid scintillator-based detectors have proven to be a competitive
technology~\cite{KamLANDZen2013} and offer the advantage of
scalability to larger instrumented masses by dissolving larger amounts
of the isotope of interest into the liquid scintillator (LS).  This
may allow scaling to 1~ton or more of isotope using detectors already
in operation \cite{Biller2013}.  In a large LS detector, most
backgrounds can be strongly suppressed through a combination of
filtration of the LS to remove internal contaminants, self-shielding
to minimize the effects of external contaminants, and vetoes to reduce
muon spallation backgrounds. The dominant backgrounds are
the standard model 2\nbb-decay and electron scattering of 
neutrinos from $^8$B decays in the sun.

 In a previous work~\cite{Aberle2014} we have shown that large-area
photo-detectors with timing resolution of $\sim$100~ps can be used to
resolve prompt Cherenkov photons from the slower scintillation signal
in a large LS detector and that the resulting distributions can be fit for
the directions and origin of $\sim$MeV electrons. Here we present a
study of applying this technique to the topological separation of
0\nbb-decay signal and \B~ background using a spherical harmonic
decomposition to analyze the distribution of early (and hence weighted
toward Cherenkov photons) photoelectrons (PEs) as a topological
discriminant.

The organization of the paper is as follows. 
Section~\ref{sec:detector_description} describes
the detector model. Details on event
kinematics and PE timing for signal and background are given in
Section~\ref{sec:kinematics_and_timing}. In
Section~\ref{sec:topology_and_harmonics}, we introduce the spherical
harmonic decomposition and discuss the performance of this analysis
in Section~\ref{sec:performance}. The conclusions are summarized in Section~\ref{sec:conclusions}.




%Since 0{\nbb} decay produces no neutrinos, the full energy of the
%decay is contained within the detector and the observed spectrum of
%this decay is a peak around the decay $Q$-value. Most of this energy
%is carried by the electrons which have typical kinetic energies of
%$\sim1-2\,\mathrm{MeV}$ each. Two neutrino double beta (2{\nbb}) decay
%\cite{GoeppertMayer1935} is the Standard Model allowed second order
%$\beta$-decay channel where lepton number is conserved by the
%production of two anti-neutrinos,
%\mbox{$(Z,A)\rightarrow(Z,A+2)+2\beta+2\bar\nu_e$}. Since the kinetic
%energy of the neutrinos is practically never detected, the energy
%spectrum measured from 2{\nbb} decay is broadened from 0~MeV up to the
%decay $Q$-value. Because it is intrinsic to the target isotope, the
%high energy tail of the 2{\nbb} spectrum forms an irreducible
%background to the 0{\nbb} signal. The only way to distinguish the two
%is through a shape analysis of the resulting decay spectrum. This
%requires a detector with a good energy resolution (see
%Fig.~\ref{fig:SNOp_bkgs}). Present LS-based detectors achieve typical
%energy resolutions of \mbox{$\sigma(E)\sim 5\%/\sqrt{E(\rm
%    MeV)}$}. The next generation of detectors will seek to improve
%upon this by increasing both the photo-covering of the detector and
%light yield of the LS. \JOcom{Eventually this will fold back in the
%  question of slowing down the scintillation signal and improving the
%  Cherenkov signal at the cost of decreasing the total light yield.}
  
%The spectrum of ES interactions of $^{8}$B solar neutrinos falls
%slowly over the range $2-3\,\mathrm{MeV}$, creating a nearly flat
%background across the ROI and reducing the sensitivity to 0{\nbb}. In
%this energy region, these interactions produce only a single
%$\sim$2.5~MeV electron, rather than two $\sim$1.2~MeV electrons as in
%0{\nbb}. In a LS, this difference in event topology manifests as two
%distinct distributions of Cherenkov photons, and thus creates a way to
%tag and remove these $^{8}$B solar neutrino events. As we have shown
%in previous works, photo-detectors with timing resolution of
%$\sim$100~ps can resolve the prompt Cherenkov photons from the slower
%scintillation signal \cite{Aberle2014}. The challenge is that for a
%given event, we expect $\sim$100 Cherenkov photons with which to
%reconstruct the event topology. \JOcom{It would be good to say
%something like: In a SNO/KamLAND sized detector, we expect $\sim$50
%  $^{8}$B events, so our rejection needs to be at least this good. But
%  perhaps we say that later?}








