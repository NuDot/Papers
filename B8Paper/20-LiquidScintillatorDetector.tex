\section{Detector Model}
\label{sec:detector_description}

%% Comment from Taritree (9/20): Though it is probably self-evident, you have not explicitly told the reader why she is reading about a detector model.  One could put in a line somewhere here or in the last paragraph in the previous section that this model is used to generate data to demonstrate/study this technique.

We use the Geant4-based simulation of Ref.~\cite{Aberle2014}
to model a sphere of 6.5~m radius filled with liquid scintillator. We
consequently limit the discussion of the simulation to a summary of
the most relevant parameters.

The scintillator composition has been chosen to match a KamLAND-like
scintillator\cite{kamland2003}. The composition is 80\% n-dodecane,
20\% pseudocumene and 1.52~g/l PPO with a density of $\rho$ =
0.78~g/ml).  We use the Geant4~default liquid scintillator optical
model, in which optical photons are assigned the group velocity in the
wavelength region of normal dispersion. The attenuation
length\cite{tajimaMaster}, scintillation emission
spectrum\cite{tajimaMaster}, and refractive index\cite{OlegThesis}
include wavelength-dependence. The scintillator light yield is assumed
to be 9030 photons/MeV) with Birks quenching ($kB$ $\approx$
0.1~mm/MeV)\cite{ChrisThesis}. However, we deviate from the
baseline KamLAND case in that the re-emission of absorbed photons in
the scintillator bulk volume and optical scattering, specifically
Rayleigh scattering, have not yet been included. A test simulation
shows that the effect of optical scattering is
negligible~\cite{Aberle2014}.

%This is important but I think I put it back below.
%The attenuation length at 400~nm, which is the position of the peak standard bialkali photocathode efficiency, is 25~m. The attenuation length drops precipitously from 6.5~m to 0.65~m between 370~nm and 360~nm. We use this drop to define the cutoff wavelength at 370~nm. This is a standard scintillator. However, we do deviate from the baseline KamLAND case in that the re-emission of absorbed photons in the scintillator bulk volume and optical scattering, specifically Rayleigh scattering, has not yet been included. A test simulation shows that the effect of optical scattering is negligible~\cite{Aberle2014}.

The technique of using Cherenkov light for topological \B~ background rejection
depends on the inherent time constants that (on average) slow
scintillation light relative to the Cherenkov light for wavelengths
longer than the scintillator absorption cutoff (between
360-370~nm~\cite{scint_cutoff}). The first step in the scintillation
process is the transfer of energy deposited by the primary
particles from the scintillator's solvent to the solute. The time
constant of this energy transfer accounts for a rise time in
scintillation light emission. Because past neutrino experiments were
not highly sensitive to the effect of the scintillation rise time,
there is a lack of accurate measurements of this property. We assume a
rise time of 1.0~ns from a re-analysis of the data in
Ref.~\cite{ChristophThesis} but more detailed studies are needed. 

The decay time constants are determined by the vibrational energy
levels of the solute and are measured to be $\tau_{d1}$ = 6.9~ns and
$\tau_{d2}$ = 8.8~ns with relative weights of 0.87 and 0.13 for the
KamLAND scintillator~\cite{tajimaThesis}. In a detector of this size,
chromatic dispersion, wherein red light traveling faster than blue due to the
wavelength-dependent index of refraction, enhances the separation but
is not the dominant effect.

The inner sphere surface is used as the photodetector. It is treated
as fully absorbing with no reflections and with 100\% photocathode
coverage~\cite{Juno_coverage}. As in the case of optical scattering,
reflections at the sphere are a small effect that would create a small
tail at longer times and, hence, does not affect the identification of the
early Cherenkov light. The assumed quantum efficiency (QE) is that
of a typical bialkali photocathode (Hamamatsu R7081
PMT~\cite{Hamamatsu_R7081}, see also Ref.~\cite{dctwo}). 
%% this is a natural place to state the assumed QE
We note that the KamLAND 17-inch PMTs use the same photocathode type with similar
quantum efficiency; photocathodes with higher efficiencies are now
starting to become better understood theoretically and may become
commercially available~\cite{Photonis, Smedley, Cultrera}.  In order to
neglect the effect of the transit-time-spread (TTS) of the
photodetectors, we use a TTS of 0.1~ns ($\sigma$), which can be
achieved with large area picosecond photodetectors
(LAPPDs)~\cite{anode_paper,PSEC4_paper,RSI_paper,Vienna2013,Ceramic_paper1,HV_paper,Timing_paper,Incom_paper}.
We neglect the (small) threshold effects in the photodetector readout
electronics, spatial resolution of the photoelectron hit positions,
and contributions to time resolution other than the photodetector TTS.

