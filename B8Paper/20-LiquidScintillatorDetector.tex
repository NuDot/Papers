\section{Detector Model}
\label{sec:detector_description}

A simulation using Geant4~version 4.9.6  is used to model a sphere of 6.5~m radius filled with
liquid scintillator\cite{geant4one,geant4two}.This is the same simulation used in our preceding paper~\cite{Aberle2014}. Therefore, we limit out discussion of the simulation to a summary of the the most relevant simulation parameters.  The scintillator composition has been chosen to match a KamLAND-like scintillator\cite{kamland2003}. The composition is 80\% n-dodecane, 20\% pseudocumene and 1.52~g/l PPO with a density of $\rho$ = 0.78~g/ml).

We use the Geant4~default liquid scintillator optical model, in which optical photons are assigned the group velocity in the wavelength region of normal dispersion. Wavelength dependent attenuation length\cite{tajimaMaster}, scintillation emission spectrum\cite{tajimaMaster}, and refractive index\cite{OlegThesis} are used. The scintillator light yield is assumed to be 9030 photons/MeV) with Birks quenching  ($kB$ $\approx$ 0.1~mm/MeV)\cite{ChrisThesis}. However, we do deviate from the baseline KamLAND case in that the re-emission of absorbed photons in the scintillator bulk volume and optical scattering, specifically Rayleigh scattering, has not yet been included. A test simulation shows that the effect of optical scattering is negligible~\cite{Aberle2014}.

%This is important but I think I put it back below.
%The attenuation length at 400~nm, which is the position of the peak standard bialkali photocathode efficiency, is 25~m. The attenuation length drops precipitously from 6.5~m to 0.65~m between 370~nm and 360~nm. We use this drop to define the cutoff wavelength at 370~nm. This is a standard scintillator. However, we do deviate from the baseline KamLAND case in that the re-emission of absorbed photons in the scintillator bulk volume and optical scattering, specifically Rayleigh scattering, has not yet been included. A test simulation shows that the effect of optical scattering is negligible~\cite{Aberle2014}.

In this study, we use the fact that scintillators have inherent time constants that slow this light relative to the Cherenkov light with wavelengths too long to be absorbed by the scintillator, between 360-370~nm for this scintillator. Therefore, the key inputs to the simulation are the time constants inherent to the scintillator cocktail that will determine the timing of the scintillation light relative to the unperterbed Cherenkov light. The first step in the scintillation process is the transfer of the energy deposited by the primary particles from the scintillator's solvent to the solute. The time constant of this energy transfer accounts for a rise time in scintillation light emission. Because past neutrino experiments were not highly sensitive to the effect of the scintillation rise time, there is a lack of accurate measurements of this property. We assume a rise time of 1.0~ns from a re-analysis of the data in Ref.~\cite{ChristophThesis} but more detailed studies are needed. The decay time constants are determined by the vibrational energy levels of the solute and are measured to be $\tau_{d1}$ = 6.9~ns and $\tau_{d2}$ = 8.8~ns with relative weights of 0.87 and 0.13 for the KamLAND scintillator\cite{tajimaThesis}. In a detector of this size, chromatic dispersion, red light travelling faster than blue due to the wave-length dependent index of refraction, enhances the separation but is not the dominant effect.

The inner sphere surface is used as the photodetector. It is treated as fully absorbing, no reflections, with 100\% photocathode coverage. As in the case of optical scattering, reflections at the sphere are a small effect that would create a small tail at longer times. The quantum efficiency (QE) of a bialkali photocathode (Hamamatsu R7081 PMT)\cite{Hamamatsu_R7081}, see also Ref.~\cite{dctwo}. We note that the KamLAND 17-inch PMTs use the same photocathode type with similar quantum efficiency. In order to neglect the effect of the transit-time-spread (TTS) of the photodetectors, we use a TTS of 0.1~ns ($\sigma$), which can be achieved with large area picosecond photodetectors
(LAPPDs)\cite{anode_paper,PSEC4_paper,RSI_paper,Vienna2013,Ceramic_paper1,HV_paper,Timing_paper,Incom_paper}. We neglect any threshold effects in the photodetector readout electronics and the photoelectron hit positions and the detection times after the TTS resolution has been applied are used for event reconstruction.
