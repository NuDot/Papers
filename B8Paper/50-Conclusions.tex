\section{Conclusions}
\label{sec:conclusions}
We consider the use of large-area photodetectors with good time and
space resolution in kiloton scale liquid scintillator detectors to
suppress background coming from $^{8}$B solar neutrino
interactions. Using a default model detector with parameters derived
from present practice, we show that a sample of detected photons
enriched in Cherenkov light by a cut on time-of-arrival contains
directional information that can be used to separate 0\nbb-decay from
$^{8}$B solar neutrino interactions. The separation is based on a
spherical harmonics analysis of the event topologies of the
two electrons in signal events and the single electron in the
background. The performance of the technique is constrained by
chromatic dispersion, vertex reconstruction, and the time profile of
the emission of scintillation light. The development of a scintillator
with a rise time constant of at least 5~ns would allow a
Cherenkov-scintillation light separation with a background rejection
factor for \B~ solar neutrinos of 3 and an efficiency for 0\nbb-decay signal
of 70\%.


