\section{Kinematics and Timing of Signal and Background events}
\label{sec:kinematics_and_timing}

\subsection{0\nbb-decay signal and 2\nbb-decay background}

In both 0\nbb-decay signal and 2\nbb-decay background events near the decay energy spectrum endpoint, the kinematics of the
electron pair is very similar. Large fraction of events have a nearly back-to-back topology with a close to 
equal energy split between electrons. To simulate 0\nbb- and 2\nbb-decay events we use a Monte Carlo generator based on phase 
factors from Ref.~\cite{Jenni}. Similarity in kinematics of 0\nbb- and 2\nbb-decay events is demonstrated in Fig.~\ref{fig:Kinematics}.

The electron angular correlations for 0\nbb-decay are noticeably different from 2\nbb-decay due to a contribution from the
neutrino wave-functions even at vanishingly small energies of the neutrinos~\cite{Jenni}. However, any practical use of this difference 
in separating 0\nbb-decay from 2\nbb-decay would require extremely large number of candidate events. Given the half-time of 2\nbb-decay 
and upper limits on the half-time of 0\nbb-decay, electron angular correlations will not bring a decisive separation power in controlling 
2\nbb-decay background in currently planned 0\nbb-decay experiments. Excellent energy resolution at the Q-value remains the key parameter
in 2\nbb~background suppression.

While we do not exclude that the angular correlations as an input to a multivariative technique may improve sensitivity of 0\nbb-decay 
searches, in this paper we assume that there is no difference in the event topology between 0\nbb-~and 2\nbb-decay events. Any conclusions 
about 0\nbb-decay events also hold for 2\nbb-decay when the total energy of the electrons in 2\nbb-decay events is close to the Q-value.


\begin{figure*}[ht]
  \centering
  \includegraphics[width=0.49\textwidth]{hCos_Te130.pdf}
  \includegraphics[width=0.49\textwidth]{hE1toQ_Te130.pdf}
  \caption{Comparison between kinematics of 0{\nbb} (\emph{dashed red
      lines}) and 2{\nbb} decays (\emph{solid black lines}) for events
    with the total kinetic energy of the electrons above 90\% of the
    Q-value. \emph{Left:} Cosine of the angle between two
    electrons. \emph{Right:} Fraction of energy carried by one of the
    two electrons. Vertical bars at each bin of the histograms indicate
    statistical uncertainty for that bin.}
  \label{fig:Kinematics}
\end{figure*}


%Should we recalculate for 130Te.
Examining the kinematics for one of the 0\nbb~electrons with equal energy split, a 1.26~MeV electron travels a total path length of 0.X~cm, 
has a distance from the origin of 0.X~cm in 0.0X$\pm$0.00X~ns  and takes 0.0X$\pm$0.00X~ns to drop below Cherenkov threshold. 
We note that due to scattering of the electron, the final direction of the electron before it stops does not correspond to the initial 
direction; however the scattering angle is small at the time the majority of Cherenkov light is produced.

Figure~\ref{fig:ArrivalTimeDist} shows the output of the detector simulation for 1000 simulated \Te~ 0\nbb-decay 
events. The left panel in Fig.~\ref{fig:ArrivalTimeDist} compares PE arrival time between Cherenkov and scintillation light  
and the right panel in Fig.~\ref{fig:ArrivalTimeDist} zooms in on the Cherenkov photon distribution which is key to direction and 
topographical reconstruction.


\begin{figure*}[ht]
  \centering
  \includegraphics[width=0.45\textwidth]{hT_Te130.pdf}
  \includegraphics[width=0.45\textwidth]{hTche_Te130_B8.pdf}
  \caption{\emph{Left:} Photo-electron (PE) arrival times after
    application of the photo-detector transit time spread (TTS) of 100~ps for the default simulation 
    of \Te~0\nbb-decay produced at the center of the detector. 
    Scintillation PE arrival time distribution is compared for \nbb-decay (dashed blue line) and
    \Cten~events (dotted green line). The corresponding distribution for \B~events is not shown
    because it is indistinguishable from the distribution for \nbb-decay. Cherenkov PE arrival
    times are shown for \nbb-decay (\emph{solid red line}) to demonstrate their contribution to the early PE sample.
    The vertical line at 33.5~ns indicates the time cut for the selection of the early PE sample.
    The shape of scintillation PE arrival times for \B~events
    \emph{Right:} Comparison between Cherenkov PEs arrival time for \Te~0\nbb-decay (\emph{solid line}), 
    \B~(\emph{dashed line}).}
\label{fig:ArrivalTimeDist}
\end{figure*}


Selection of PEs with relatively small arrival time allows the selection of a sample of PEs with a high fraction of directional Cherenkov light.
This allows for event topology reconstruction. In particular, signal-like events with exactly two electrons can be separated from events 
with only one electron such as from \B~solar neutrino interactions.

As shown in Fig.~\ref{fig:ArrivalTimeDist}, for events produced at the center of the detector, a time cut of 33.5~ns on the PE arrival 
time selects a sample of early PEs that includes the majority of directional Cherenkov photons. Scintillation PEs also are 
selected with this time cut. Figure~\ref{fig:NPhotDist} shows the total number of scintillation and Cherenkov PE per event in the early 
PE sample for signal and background events.



\begin{figure*}[ht]
  \centering
  \includegraphics[width=0.45\textwidth]{hMomNPhot_Te130.pdf}
  \includegraphics[width=0.45\textwidth]{hMomNPhot_1el_2p529MeV.pdf}
%  \includegraphics[width=0.33\textwidth]{hMomNPhot_C10.pdf}
  \caption{Early PE sample composition: number of Cherenkov (\emph{dashed red line}), scintillation
    (\emph{dotted blue line}), and total (\emph{solid black line}) PEs per event
    for the simulation of 1000 $^{130}$Te 0\nbb-decay events (\emph{left panel}),
    1000 $^8$B events (\emph{middle panel}).}%, and 4152 \C~events (\emph{right panel}).}
\label{fig:NPhotDist}
\end{figure*}



\subsection{\B~background}

For a detector similar to our model, \B~background is significant due to large total mass of the liquid scintillator in
the active region of the detector.
Electrons from elastic scattering of \B~solar neutrinos have nearly a flat energy spectrum around the 
Q-value~\cite{SNOp-B8-bkg}. We simulate \B~background as a single monochromatic electron with energy of 2.53~MeV 
(Q-value of \Te). A 2.53~MeV electron travels a total path length of 0.X~cm, has a distance from the origin of 0.X~cm in 
0.0X$\pm$0.00X~ns  and takes 0.0X$\pm$0.00X~ns to drop below Cherenkov threshold.

The scintillation PE timing distribution is unchanged compared to 0\nbb-decay since the electron's path length at these energies 
is too short to affect the effective vertex of the scintillation light. The number of Cherenkov photons is increased 
(see Fig.~\ref{fig:ArrivalTimeDist} and~\ref{fig:NPhotDist}) due to the increased electron kinetic energy, but this alone is not 
sufficient to distinguish \B~events from 0\nbb-decay. However, it may provide an extra handle on signal-background separation in 
combination (e.g. by using multivariative techniques) with other event parameters.

