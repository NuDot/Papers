\section{Introduction}

Over the past decade and a half, neutrino oscillation experiments have
been able to conclusively establish that neutrinos have mass
\cite{SNO2001,SNO2002,SuperK2002,kamland2003}. However, the nature of
that mass remains one of the most fundamental open questions in
particle physics. Is the neutrino unique among the Standard Model
fermions with a Majorana-type mass \cite{Majorana1937}, as is
predicted by most beyond the standard model (BSM) theories, or does it
have a Dirac-type mass like the rest of the fermions? A Majorana-type
mass would have far reaching implications, from explaining the
lightness of the neutrino and providing a bridge to higher energy
phenomena through the see-saw mechanism
\cite{GellMann1980,Yanagida1979} to being able to provide the required
lepton-number violation (LNV) and {\sf CP}-violation needed for
leptogenesis to explain the baryon asymmetry of the universe
\cite{Fukugita1986,Luty1992}. Conversely, a Dirac-type neutrino mass
could point to an underlying symmetry of the Universe.  Presently, the
most promising technique for answering these questions is the search
for Neutrinoless Double-Beta (0\nbb) decay \cite{Furry1939}. In this
decay, a nucleus undergoes a second order $\beta$-decay without
producing any neutrinos, $(Z,A)\rightarrow(Z+2,A)+2\beta^-$. 

Recently, this search has generated a significant amount of
experimental interest, with the largest on-going experiments searching
for {0\nbb} decay of $^{76}$Ge \cite{GERDA2013}, $^{130}$Te
\cite{CUORE2015,CUORE2016} and $^{136}$Xe
\cite{EXO2014,KamLANDZen2013}. At present, 0{\nbb} decay has never
been convincingly observed, but present limits indicate that the
half-lives are longer than $10^{23}-10^{25}\,\mathrm{yr}$ in the
isotopes studied. The standard mechanism of 0{\nbb} decay is
parameterized by the \emph{effective Majorana mass}, defined as
\mbox{$m_{\beta\beta}\equiv\left|\sum_i U^2_{ei}m_i\right|$}, where
$U_{ei}$ are the elements of the PMNS matrix and $m_i$ are the
neutrino masses. Current half-life limits translate to a limit on
\mbox{$m_{\beta\beta}\lesssim 150-700\,\mathrm{meV}$}. The majority of
the spread on this bound comes from theoretical uncertainty in the
nuclear modeling (see \cite{Vogel2012} for a review). The next
generation of 0{\nbb} decay experiments seek to be sensitive enough to
detect or rule out 0{\nbb} decay down to \mbox{$m_{\beta\beta}\lesssim
  10$~meV}. This will require a detector to instrument roughly a ton
of active isotope, maintain a good energy resolution, and achieve a
near zero background in the region of interest (ROI) over the course
of the experiment \cite{Cremonesi2013} [Second CITATION?].

Over the past few years, liquid scintillator-based detectors have
proven to be a competitive technology in this search
[CITATION?]. Their primary advantage is their ease of scalability to
larger instrumented masses, which involves disolving larger amounts of
the isotope of interest into the liquid scintillator (LS). This
feature can allow for rapid scaling to 1~ton or more using the
detectors already in operation \cite{Biller2013}. In a large LS
detector, most backgrounds can be strongly suppressed through a
combination of filtration of the LS to remove internal contaminents,
self-shielding to minimize the effects of external contaminents, and
vetoing to reduce cosmic ray effects. The backgrounds relevant to
0{\nbb} decay which cannot be reduced through these means are the
2{\nbb} decay and electron scattering of $^8$B solar neutrinos.

Since 0{\nbb} decay produces no neutrinos, the full energy of the
decay is contained within the detector and the observed spectrum of
this decay is a peak around the decay $Q$-value. Most of this energy
is carried by the electrons which have typical kinetic energies of
$\sim1-2\,\mathrm{MeV}$ each. Two neutrino double beta (2{\nbb}) decay
\cite{GoeppertMayer1935} is the Standard Model allowed second order
$\beta$-decay channel where lepton number is conserved by the
production of two anti-neutrinos,
\mbox{$(Z,A)\rightarrow(Z,A+2)+2\beta+2\bar\nu_e$}. Since the kinetic
energy of the neutrinos is practically never detected, the energy
spectrum measured from 2{\nbb} decay is broadened from 0~MeV up to the
decay $Q$-value. Because it is intrinsic to the target isotope, the
high energy tail of the 2{\nbb} spectrum forms an irreducible
background to the 0{\nbb} signal. The only way to distinguish the two
is through a shape analysis of the resulting decay spectrum. This
requires a detector with a good energy resolution (see
Fig.~\ref{fig:SNOp_bkgs}). Present LS-based detectors achieve typical
energy resolutions of \mbox{$\sigma(E)\sim 5\%/\sqrt{E(\rm
    MeV)}$}. The next generation of detectors will seek to improve
upon this by increasing both the photo-covering of the detector and
light yield of the LS. \JOcom{Eventually this will fold back in the
  question of slowing down the scintillation signal and improving the
  Cherenkov signal at the cost of decreasing the total light yield.}
  
The spectrum of ES interactions of $^{8}$B solar neutrinos falls
slowly over the range $2-3\,\mathrm{MeV}$, creating a nearly flat
background across the ROI and reducing the sensitivity to 0{\nbb}. In
this energy region, these interactions produce only a single
$\sim$2.5~MeV electron, rather than two $\sim$1.2~MeV electrons as in
0{\nbb}. In a LS, this difference in event topology manifests as two
distinct distributions of Cherenkov photons, and thus creates a way to
tag and remove these $^{8}$B solar neutrino events. As we have shown
in previous works, photo-detectors with timing resolution of
$\sim$100~ps can resolve the prompt Cherenkov photons from the slower
scintillation signal \cite{Aberle2014}. The challenge is that for a
given event, we expect $\sim$100 Cherenkov photons with which to
reconstruct the event topology. \JOcom{It would be good to say
  something like: In a SNO/KamLAND sized detector, we expect $\sim$50
  $^{8}$B events, so our rejection needs to be at least this good. But
  perhaps we say that later?}

In this paper, we propose to use a spherical harmonic decomposition
to analyze the distribution of early photo-electrons (PE) to
discriminate between $^{8}$B solar neutrinos and 0{\nbb} decay
events. In Section~\ref{sec:detector_description}, we describe the
detector model we will use throughout this paper. In
Section~\ref{sec:spherical_harmonics}, we introduce the spherical
harmonic decomposition, and discuss the performance of this analysis
in Section~\ref{sec:performance_and_challenges}.




