\section{Introduction}

Over the past decade, neutrino oscillation experiments have
conclusively established that neutrinos have mass. However, one of the
most fundamental and still open questions in particle physics is the
nature of that mass. Is the neutrino unique in the Standard Model with
a Majorana type mass, as is predicted by most theoretical arguments,
or does it have a Dirac type mass, like the rest of the fermions in
the Standard Model? A Majorana mass would have far reaching
implications, from explaining the lightness of the neutrino and
providing a bridge to higher energy phenomena through the see-saw
mechanism [CITATION] to providing the required lepton-number violation
(LNV) and {\sf CP}-violation needed for leptogenesis to explain the
baryon asymmetry of the universe \cite{Luty1992}. Conversely, a Dirac
neutrino could point to an underlying symmetry of the Universe.
Presently, the most promising technique for answering these questions
is the search for Neutrinoless Double-Beta (0\nbb) decay
[CITATION]. In this decay, a nucleus undergoes a second order
$\beta$-decay without producing any neutrinos,
$(Z,A)\rightarrow(Z+2,A)+2\beta^-$. The resulting decay products are
ejected with a total kinetic energy equal to the decay $Q$-value, and
most of this is carried by the electrons which have typical kinetic
energies of $\sim$1--2~MeV.

The standard mechanism of 0{\nbb} decay is parameterized by the
\emph{effective Majorana mass},
\mbox{$m_{\beta\beta}\equiv\left|\sum_i U^2_{ei}m_i\right|$}, where
$U_{ei}$ are the elements of the PMNS matrix and $m_i$ are the
neutrino masses [CITATION].  Recently, this search has generated a
significant amount of experimental interest, with current experiments
searching for 0\nbb decay of $^{76}$Ge \cite{GERDA2013}, $^{130}$Te
\cite{CUORE2015} and $^{136}$Xe \cite{EXO2014,KamLANDZen2013}. At
present, 0{\nbb} decay has never been convincingly observed, but
limits place the half-lives at longer than $10^{23}$--$10^{25}$~yr in
the isotopes studied. These limits translate to a limit on
\mbox{$m_{\beta\beta}\lesssim 150$--$700$~meV}, with the majority of
the spread on this limit coming from uncertainty in the nuclear
modeling. The next generation of 0{\nbb} decay experiments seek to be
sensitive enough to detect or rule out 0{\nbb} decay down to
\mbox{$m_{\beta\beta}<10$~meV}. This will require, among other things,
$\sim$a ton of active isotope, a good energy resolution, and a near
zero background in the region of interest (ROI) [CITATION?].

Over the past few years, liquid scintillator-based detectors have
proven to be a competitive technology in this search
[CITATION?]. Their primary advantage is in their scalability to larger
active masses, which entails disolving larger amounts of active
isotope into the liquid scintillator (LS). This feature allows for
rapid scaling to the 1~ton or more of active isotope. In a large LS
detector, where backgrounds from the outside of the detector can be
efficiently self-shielded or tagged and vetoed, the two backgrounds
relavent to 0{\nbb} decay are 2{\nbb} and electron scattering (ES) interactions of $^{8}$B solar neutrinos.

Two neutrino double beta (2{\nbb}) decay is the Standard Model allowed
second order $\beta$-decay channel where lepton number is conserved,
\mbox{$(Z,A)\rightarrow(Z,A+2)+2\beta+2\bar\nu_e$}. Since the kinetic
energy of the neutrinos is practically never detected, the resulting
2{\nbb} spectrum is broadened from 0~MeV up to the decay
$Q$-value. The high energy tail of this spectrum forms a background
for the 0{\nbb} signal. Since it is intrinsic to the target isotope
and the decay kinematics are nearly identical to 0{\nbb} decay,
2{\nbb} can only be separated from 0{\nbb} with a detector with
powerful enough resolution (see Fig.~\ref{fig:SNOp_bkgs}). Present
LS-based detectors achieve typical energy resolutions of
\mbox{$\sigma(E)\sim 5\%/\sqrt{E(\rm MeV)}$}. The next generation of
detectors will improve up this... \JOcom{Something about
  photo-coverage (etc). Eventually this will fold back in the question
  of slowing down the scintillation signal and improving the Cherenkov
  signal at the cost of decreasing the total light yield.}
  
  
The spectrum of ES interactions of $^{8}$B solar neutrinos falls
slowly over the range 2--3~MeV, creating a nearly flat background
across the 0{\nbb} ROI. However, these interactions produce only a
single $\sim$2.5~MeV electron, rather than two $\sim$1.2~MeV electrons
as in 0{\nbb}. In a LS, this difference in event topology manifests as
two distinct distributions of Cherenkov photons, and thus creates a
way to tag and remove these $^{8}$B solar neutrino events. As we have
shown in previous works, photo-detectors with timing resolution of
$\sim$100~ps can resolve the prompt Cherenkov photons from the slower
scintillation signal \cite{Aberle2014}. The challenge is that for a
given event, we expect XXX Cherenkov photons with which to reconstruct
the event topology.

In this paper, we proposed to use a spherical harmonic decomposition
to analyze the distribution of early photo-electrons (PE) to
discriminate between $^{8}$B solar neutrinos and 0{\nbb} decay
events. In Section~\ref{sec:detector_description}, we describe the
detector model we will use throughout this paper. In
Section~\ref{sec:spherical_harmonics}, we introduce the spherical
harmonic decomposition, and discuss the performance of this analysis
in Section~\ref{sec:performance_and_challenges}.




