\section{Detector Model}
\label{sec:detector_description}
{\bf Copy info from~\cite{Aberle2014}:}

In order to study the effects relevant to directional reconstruction
in liquid scintillators, a Geant4\cite{geant4one,geant4two} simulation
has been constructed. The simulation uses Geant4~version 4.9.6 with the default liquid scintillator
optical model, in which optical photons are
assigned the group velocity in the wavelength region of normal
dispersion.

The detector geometry is a sphere of 6.5~m radius filled with
scintillator. Figure \ref{detector_view} shows the geometry and the
Cherenkov light from an example $^{116}$Cd $0\nu\beta\beta$ event. The
default scintillator properties have been chosen to match a KamLAND-like
scintillator\cite{kamland2003}: 80\% n-dodecane, 20\% pseudocumene and 1.52~g/l PPO. The
scintillator properties implemented in the simulation include the
atomic composition and density ($\rho$ = 0.78~g/ml), the
wavelength-dependent attenuation length\cite{tajimaMaster} and
refractive index\cite{OlegThesis}, the scintillation emission
spectrum\cite{tajimaMaster}, emission rise time ($\tau_r$ = 1.0~ns)
and emission decay time constants ($\tau_{d1}$ = 6.9~ns and
$\tau_{d2}$ = 8.8~ns with relative weights of 0.87 and 
0.13)\cite{tajimaThesis}, scintillator light yield (9030 photons/MeV),
and the Birks constant ($kB$ $\approx$ 0.1~mm/MeV)\cite{ChrisThesis}.  This is a standard scintillator. The attenuation length at 400~nm, the position of the peak standard bialkali photocathode efficiency, is 25~m. The attenuation length drops precipitously between 370~nm and 360~nm from 6.5~m to 0.65~m. We use this drop to define the cutoff wavelength at 370~nm. Variations from the baseline KamLAND case are discussed below. 

Re-emission of absorbed photons in the scintillator
bulk volume and optical scattering, specifically Rayleigh scattering, have not yet been included by default. A test simulation shows that the effect of optical scattering is negligible~\cite{Aberle2014}.

The inner sphere surface is used as the photodetector. It is treated
as fully absorbing (no reflections), with a photodetector coverage of
100\%. As in the case of optical scattering, reflections at the sphere are a small effect that would create a small tail at longer times. Two important photodetector properties have been varied: 1)
the transit-time spread (TTS, default $\sigma$ = 0.1~ns) and 2) the
wavelength-dependent quantum efficiency (QE) for photoelectron
production. The default is the QE of a bialkali photocathode (Hamamatsu
R7081 PMT)\cite{Hamamatsu_R7081}. The QE values as a function of wavelength come from the Double Chooz\cite{dctwo}
Monte Carlo simulation. We note that the KamLAND 17-inch PMTs use the
same photocathode type with similar quantum efficiency. We are neglecting any threshold effects in the photodetector readout electronics.


Four effects primarily contribute to the timing of the scintillator detector
system: the travel time of the particle, the time constants of the scintillation process, chromatic dispersion, and the timing of the photodetector.

In the energy range important for $0\nu\beta\beta$, a 1.4~MeV electron travels a total path length of 0.8~cm, has a distance from the origin of 0.6~cm in 0.030$\pm$0.004~ns  and takes 0.028$\pm$0.004~ns to drop below Cherenkov threshold. We note that due to scattering the final direction of the electron before it stops does not correspond to the initial direction; however the scattering angle is small while the majority of Cherenkov light is produced. The Cherenkov light thus encodes the direction of the primary electron. The scattering physics is handled by Geant4's ``Multiple Scattering" process which is valid down to 1~keV, where atomic shell structure becomes important\cite{geant4scatt}.


The scintillator-specific rise and decay times are the second effect that determines the timing in a scintillator detector. The first step in the scintillation process is the transfer of energy from the solvent to the solute. The time constant of this
energy transfer accounts for a rise time in scintillation light
emission. Past neutrino experiments were not highly sensitive to the
effect of the scintillation rise time, which is the reason why there
is a lack of accurate numbers. We assume a rise time of 1.0~ns; more
detailed studies are needed in the future. The two time constants used
to describe the falling edge of the scintillator emission time
distribution (quoted above) are values specific to the KamLAND
scintillator.

Chromatic dispersion is the third effect that determines the timing in a scintillator detector. Due to the wavelength-dependence of the refractive index the speed of
light in the scintillator (see Equation (\ref{eqGroup})) increases
with increasing photon wavelengths for normal dispersion, with red
light traveling faster than blue light.

Photoelectrons coming from Cherenkov light are on average
created about 0.5~ns earlier than PEs from scintillation light. The
RMS values from PE time distributions for Cherenkov and scintillation
light are both about 0.5~ns. Note that these numbers include the
effect of the finite electron travel time.

The fourth effect determining the timing in a scintillator detector is the timing of the photodetectors. The measurement of the arrival times of single photoelectrons is
affected by the transit-time spread (TTS) of the photodetectors, a
number which can be different by orders of magnitude depending on the
detector type. The default TTS of 0.1~ns ($\sigma$) can be achieved with large area picosecond photodetectors
(LAPPDs)\cite{Adams:2013nva,RSI_paper,PSEC4_paper,anode_paper} and possibly hybrid photodetectors
(HPDs)\cite{hpdThesis}; even significantly lower TTS numbers are
realistic with the LAPPD\cite{RSI_paper,PSEC4_paper,anode_paper}.

The primary
quantities provided by the Geant4~simulation are the photoelectron hit
positions and the detection times after the TTS resolution has been
applied. In section~\ref{reconstruction_sec} these quantities are then
used for event reconstruction.




{\bf Figures~\ref{fig:ArrivalTimeDist} and~\ref{fig:NPhotDist} show simulation
output relevant for further discussion.}

\begin{figure*}[ht]
  \centering
  \includegraphics[width=0.45\textwidth]{hT_Te130.pdf}
  \includegraphics[width=0.45\textwidth]{hTche_Te130_B8.pdf}
  \caption{\emph{Left:} Photo-electron (PE) arrival times after
    application of the photo-detector transit time spread (TTS) of
    100~ps for the simulation of 1000 0{\nbb} decay events of
    $^{130}$Te at the center of the detector. PEs from Cherenkov light
    (\emph{dashed red line}) and scintillation light (\emph{solid blue
      line}) are compared. The black vertical line illustrates a time
    cut at 33.5 ns. \emph{Right:} Comparison between Cherenkov PEs
    arrival time for $^{130}$Te {0\nbb} decay (\emph{solid line}) and
    $^{8}$B (\emph{dotted line}) events. {\bf Distributions of the
      scintillation PEs arrival time are indistinguishable between
      $^{130}$Te 0{\nbb} decay and $^8$B due to identical total energy
      in the event, $Q(^{130}{\rm Te})=2.526$~MeV.} }
\label{fig:ArrivalTimeDist}
\end{figure*}


\begin{figure*}[ht]
  \centering
  \includegraphics[width=0.45\textwidth]{hMomNPhot_Te130.pdf}
  \includegraphics[width=0.45\textwidth]{hMomNPhot_1el_2p529MeV.pdf}
  \caption{Number of Cherenkov (\emph{dashed red line}), scintillation
    (\emph{dotted blue line}), and total (\emph{solid black line}) PEs
    for the simulation of 1000 $^{130}$Te 0{\nbb} decay (left panel)
    and $^8$B (\emph{right panel}) events.}
\label{fig:NPhotDist}
\end{figure*}

