\section{Timing of photons coming from $^{10}$C background}

Typical energy deposition by $^{10}$C events is shown in
Fig.~\ref{fig:Edep_C10}. Assuming $\sim$15\% energy resolution, events in 
the energy range of 2.1-2.9~MeV would contribute to the backgroung count in 
the ROI.


\begin{figure}[h]
  \centering
  \includegraphics[width=0.95\textwidth]{hEdep_C10.pdf}
  \caption{Energy deposition in $^{10}$C events.}
  \label{fig:Edep_C10}
\end{figure}

Since 98\% of $^{10}$C decays through an excited state of $^{10}$B(718), 
which has a half-life time of $\sim$1~ns, the majority of $^{10}$C events have 
a prompt positron accompanied by a delayed 0.718~MeV gamma. The positron energy
has to be 0.79~MeV for an event to have energy deposition equal to Q-value of
$\Te$ $\vbb$-decay.

The positron from $\Cten$ on average travels 4~mm before it stops and anihilates
producing two 0.511~MeV gammas. Those gammas then interact in the scintillator via 
Compton scattering and photo-electric effect along while they loose their energy
over $\sim X_0$ distance. Therefore light emmited in $\Cten$ events originates from
several clusters that are spread over $\sim X_0$. Significant fraction of the 
early PE would be due to primary positron cluster. Because the positron has smaller 
kinetic energy than kinetic energy of electrons from $\vbb$-decay the amount of early 
PE is smaller for $\Cten$ events. Delayed 0.718~MeV gamma in some of $\Cten$ events 
also result in a delay in the photon arrival time with respect to $\vbb$-decay events.

Figure~\ref{fig:Arrival_time_C10_overlaid} compares PE arrival times between 
$\vbb$-decay and $\Cten$ events. Prompt $\Cten$ is a simplified simulation where 
a positron is simultaneously produced with 0.718~MeV gamma. The difference between
prompt $\Cten$ and $\vbb$-decay is caused by presence of a positron and this 
difference is typical event by even basis. An additional difference due to delayed
gamma shown in Fig.~\ref{fig:Arrival_time_C10_overlaid} is an average over 1000 events.

\begin{figure}[h]
  \centering
  \includegraphics[width=0.95\textwidth]{hT_Te130vsC10_overlaid_v2.pdf}
  \caption{Photo-electron (PE) arrival times after application of the
    photo-detector transit time spread (TTS) of 100~ps for the
    simulation of 1000 0{\nbb} decay events of $^{130}$Te (\emph{solid
      lines}) and $^{10}$C (\emph{dotted lines}) events at the center
    of the detector. Cherenkov and scintillation components are are normalized for 
    for shape comparison.}
\label{fig:Arrival_time_C10_overlaid}
\end{figure}

To distinguish between $\vbb$-decay and $\Cten$ we count total number of PEs in the 
early light sample. For central events where we assume perfect knowledge of the 
primary vertex location the early light sample is defined as t$<$33.5~ns. For a more 
realistic scenario where the vertex is uniformly distributed within the fiducial volume
the early light sample is defined as $\Delta t=t^{phot}_{measured} - 
t^{phot}_{predicted}<$1~ns.


Number of Cherenkov and scintillation PEs in early light samples for $\vbb$-decay and 
$\Cten$ central events is shown in Fig.~\ref{fig:NPhotDist_C10}. Here a perfect 
reconstrution of the primary vertex is assumed. Figure~\ref{fig:NPhot_compare_central} shows
separation between $\vbb$-decay and $\Cten$ events by counting total number of PEs. 

\begin{figure*}[ht]
  \centering
  \includegraphics[width=0.45\textwidth]{hMomNPhot_Te130.pdf}
  \includegraphics[width=0.45\textwidth]{hMomNPhot_C10.pdf}
  \caption{Early photons. Number of Cherenkov (\emph{dashed red line}), 
    scintillation
    (\emph{dotted blue line}), and total (\emph{solid black line}) PEs
    for the simulation of 1000 $^{130}$Te 0{\nbb} decay (left panel)
    and of 648 $^{10}$C (\emph{right panel}) events (1000 $^{10}$C events was 
    generated, but selected only those that has total energy deposition in the 
    detector in the range between 2.1 and 2.9~MeV).}
\label{fig:NPhotDist_C10}
\end{figure*}



\begin{figure*}[ht]
  \centering
  \includegraphics[width=0.95\textwidth]{hMomNPhot_Te130vsC10_VtxSmear0cm_VtxShiftX0cm_33p5ns_center.pdf}
  \caption{Comparison of total number of early photons between $^{130}$Te 0{\nbb} decay 
    and $^{10}$C events with energy deposition in the range between 2.1 and 2.9~MeV. 
    Events originated at the center of the sphere.
    Perfect vertex reconstruction - true vertex position is used. Time cut of 
    33.5~ns on the photon arrival time is applied.}
\label{fig:NPhot_compare_central}
\end{figure*}


Figure~\ref{fig:NPhot_compare_rndVtx_noSmear} compares total number of PEs for events uniformly 
distributed within the fiducial volume. Perfect knowledge of the vertex is assumed.

\begin{figure*}[ht]
  \centering
  \includegraphics[width=0.45\textwidth]{hMomDT_Te130vsC10_VtxSmear0cm_VtxShiftX0cm_momDT1p0ns_rndVtx_3p0mSphere.pdf}
  \includegraphics[width=0.45\textwidth]{hMomNPhot_Te130vsC10_VtxSmear0cm_VtxShiftX0cm_momDT1p0ns_rndVtx_3p0mSphere.pdf}
  \caption{(Left) Difference between measured PE arrival time and arrival time prediction based on 
	vertex location (T$^{predicted} = |r_{hit} - r_{vtx}|/v_{phot}$, where $v_phot = c/1.53$).
        $\vbb$-decay (black solid line) and $\Cten$ events (magenta dashed line) are compared. 
	Vertical line at 1~ns indicates cut for early light selection. 
        (Right) Total number of PEs in the early light sample. 
        $^{10}$C events with energy deposition in the range between 2.1 and 2.9~MeV are
	selected. Verticies are uniformly distributed within the fiducial volume, $R<3$~m.
        {\bf Perfect vertex reconstruction - true vertex position is used.}}
\label{fig:NPhot_compare_rndVtx_noSmear}
\end{figure*}


Figure~\ref{fig:NPhot_compare_rndVtx_Smear3cm} compares total number of PEs for events uniformly
distributed within the fiducial volume and reconstructed vertex smeared with 3~cm resolution.

\begin{figure*}[ht]
  \centering
  \includegraphics[width=0.45\textwidth]{hMomDT_Te130vsC10_VtxSmear3cm_VtxShiftX0cm_momDT1p0ns_rndVtx_3p0mSphere.pdf}
  \includegraphics[width=0.45\textwidth]{hMomNPhot_Te130vsC10_VtxSmear3cm_VtxShiftX0cm_momDT1p0ns_rndVtx_3p0mSphere.pdf}
  \caption{(Left) Difference between measured PE arrival time and arrival time prediction based on
        vertex location (T$^{predicted} = |r_{hit} - r_{vtx}|/v_{phot}$, where $v_phot = c/1.53$).
        $\vbb$-decay (black solid line) and $\Cten$ events (magenta dashed line) are compared.
        Vertical line at 1~ns indicates cut for early light selection.
        (Right) Total number of PEs in the early light sample.
        $^{10}$C events with energy deposition in the range between 2.1 and 2.9~MeV are
        selected. Verticies are uniformly distributed within the fiducial volume, $R<3$~m.
        {\bf Vetrex is smeared with 3~cm resolution.}}
\label{fig:NPhot_compare_rndVtx_Smear3cm}
\end{figure*}


Figure~\ref{fig:NPhot_compare_rndVtx_Smear10cm} compares total number of PEs for events uniformly
distributed within the fiducial volume and reconstructed vertex smeared with 10~cm resolution.

\begin{figure*}[ht]
  \centering
  \includegraphics[width=0.45\textwidth]{hMomDT_Te130vsC10_VtxSmear10cm_VtxShiftX0cm_momDT1p0ns_rndVtx_3p0mSphere.pdf}
  \includegraphics[width=0.45\textwidth]{hMomNPhot_Te130vsC10_VtxSmear10cm_VtxShiftX0cm_momDT1p0ns_rndVtx_3p0mSphere.pdf}
  \caption{(Left) Difference between measured PE arrival time and arrival time prediction based on
        vertex location (T$^{predicted} = |r_{hit} - r_{vtx}|/v_{phot}$, where $v_phot = c/1.53$).
        $\vbb$-decay (black solid line) and $\Cten$ events (magenta dashed line) are compared.
        Vertical line at 1~ns indicates cut for early light selection.
        (Right) Total number of PEs in the early light sample.
        $^{10}$C events with energy deposition in the range between 2.1 and 2.9~MeV are
        selected. Verticies are uniformly distributed within the fiducial volume, $R<3$~m.
        {\bf Vetrex is smeared with 10~cm resolution.}}
\label{fig:NPhot_compare_rndVtx_Smear10cm}
\end{figure*}



Figure~\ref{fig:NPhot_compare_rndVtx_Smear30cm} compares total number of PEs for events uniformly
distributed within the fiducial volume and reconstructed vertex smeared with 30~cm resolution.

\begin{figure*}[ht]
  \centering
  \includegraphics[width=0.45\textwidth]{hMomDT_Te130vsC10_VtxSmear30cm_VtxShiftX0cm_momDT1p0ns_rndVtx_3p0mSphere.pdf}
  \includegraphics[width=0.45\textwidth]{hMomNPhot_Te130vsC10_VtxSmear30cm_VtxShiftX0cm_momDT1p0ns_rndVtx_3p0mSphere.pdf}
  \caption{(Left) Difference between measured PE arrival time and arrival time prediction based on
        vertex location (T$^{predicted} = |r_{hit} - r_{vtx}|/v_{phot}$, where $v_phot = c/1.53$).
        $\vbb$-decay (black solid line) and $\Cten$ events (magenta dashed line) are compared.
        Vertical line at 1~ns indicates cut for early light selection.
        (Right) Total number of PEs in the early light sample.
        $^{10}$C events with energy deposition in the range between 2.1 and 2.9~MeV are
        selected. Verticies are uniformly distributed within the fiducial volume, $R<3$~m.
        {\bf Vetrex is smeared with 30~cm resolution.}}
\label{fig:NPhot_compare_rndVtx_Smear30cm}
\end{figure*}


Figure~\ref{fig:NPhot_compare_rndVtx_Smear50cm} compares total number of PEs for events uniformly
distributed within the fiducial volume and reconstructed vertex smeared with 50~cm resolution.

\begin{figure*}[ht]
  \centering
  \includegraphics[width=0.45\textwidth]{hMomDT_Te130vsC10_VtxSmear50cm_VtxShiftX0cm_momDT1p0ns_rndVtx_3p0mSphere.pdf}
  \includegraphics[width=0.45\textwidth]{hMomNPhot_Te130vsC10_VtxSmear50cm_VtxShiftX0cm_momDT1p0ns_rndVtx_3p0mSphere.pdf}
  \caption{(Left) Difference between measured PE arrival time and arrival time prediction based on
        vertex location (T$^{predicted} = |r_{hit} - r_{vtx}|/v_{phot}$, where $v_phot = c/1.53$).
        $\vbb$-decay (black solid line) and $\Cten$ events (magenta dashed line) are compared.
        Vertical line at 1~ns indicates cut for early light selection.
        (Right) Total number of PEs in the early light sample.
        $^{10}$C events with energy deposition in the range between 2.1 and 2.9~MeV are
        selected. Verticies are uniformly distributed within the fiducial volume, $R<3$~m.
        {\bf Vetrex is smeared with 50~cm resolution.}}
\label{fig:NPhot_compare_rndVtx_Smear50cm}
\end{figure*}



% !!!!!!!!!!!!	Commented text begins	!!!!!!!!!!!!!!!!!!!!!
\begin{comment}
\newpage

\section{0{\nbb} decay vs $^{10}$C background}

Other common backgrounds to 0{\nbb} decay search include radioactive
decays of nuclei that are excited by cosmic muons and produced through the decays of Th and U
naturally present in the materials. In liquid scintillator detectors,
most of events from Th and U decays occur in the materials of
the scintillator enclosure. Typically, they enter the fiducial volume
as 2.6~MeV gammas. These gammas pass into the fiducial volume either because they showered too late or have
mis-reconstructed vertex. Both effects depend on details of a
particular experiment and in this paper we make no attempt
to introduce a topology reconstruction for the backgrounds coming from
Th and U lines. Cosmic induced backgrounds, to the contrary, are more
generic and originate inside the fiducial volume. In this section we
discuss event topology of $^{10}$C events that are most relevant in the
energy of 2-3~MeV.

Typical energy deposition by $^{10}$C events is shown in
Fig.~\ref{fig:Edep_C10}. We propose to use spherical harmonics
analysis to separate 0{\nbb} decay events from $^{10}$C events that
within energy resolution overlap with the 0{\nbb} decay Q-value.

\begin{figure}[h]
  \centering
  \includegraphics[width=0.95\textwidth]{hEdep_C10.pdf}
  \caption{Energy deposition in $^{10}$C events.}
  \label{fig:Edep_C10}
\end{figure}

\begin{figure}[h]
  \centering
  \includegraphics[width=0.45\textwidth]{hT_C10.pdf}
  \includegraphics[width=0.45\textwidth]{hTche_C10.pdf}
  \caption{Photo-electron (PE) arrival times after application of the
    photo-detector transit time spread (TTS) of 100~ps for the
    simulation of 1000 0{\nbb} decay events of $^{130}$Te (\emph{solid
      lines}) and $^{10}$C (\emph{dotted lines}) events at the center
    of the detector. All distributions are normalized for shape
    comparison. {\bf Absolute number of PEs per event depends on the
      total energy deposited in the
      detector. Figure~\ref{fig:Edep_C10} shows energy deposited in
      the detector in $^{10}$C events.} \emph{Left:} Scintillation PEs
    arrival time. The black vertical line illustrates a time cut at
    33.5 ns. \emph{Right:} Cherenkov PEs arrival time.}
\label{fig:Arrival_time_C10}
\end{figure}

We note that 98\% of $^{10}$C decays through the excited state of
$^{10}$B(718), which has a half-life time of $\sim$1~ns. Therefore, the
majority of $^{10}$C events have a prompt positron accompanied by a
delayed 0.718~MeV gamma. This delayed gamma affects the PE arrival time
distribution. Figure~\ref{fig:Arrival_time_C10} compares the shape of the
PE arrival time distribution between $^{130}$Te 0{\nbb} decays and
$^{10}$C events. The time profile of the scintillation photons can be used
to separate signal from $^{10}$C events.

\end{comment}

%	!!!!!!!!!!!!!!!!	Commented text ends 	!!!!!!!!!!!!!!!!!!

Comparison of $S_0$ and $S_1$ distributions between 0{\nbb} decay and
$^{10}$C events is shown in Fig.~\ref{fig:S_vs_energy_C10}.

\begin{figure*}[h]
\centering
\includegraphics[width=0.49\textwidth]{hS0_C10.pdf}
\includegraphics[width=0.49\textwidth]{hS1_C10.pdf}
\caption{$S_0$ (\emph{left}) and $S_1$ (\emph{right}) distributions
  for events with different event topologies. $^{130}$Te, $^{82}$Se 0{\nbb} 
  decays compared with $^{8}$B and $^{10}$C events. The simulation is done 
  for events with the vertex in the center of the detector. $^{8}$B events 
  are implemented as 2.529~MeV or 2.995~MeV electrons with initial direction 
  along $x$-axis. $^{10}$C events are selected in the energy range between 2.1 
  and 2.9~MeV. Perfect vertex reconstruction - true vertex position is used. 
  Time cut of 33.5~ns on the photon arrival time is applied.}
\label{fig:S_vs_energy}
\end{figure*}


\begin{figure}[h]
  \centering
  \includegraphics[width=0.95\textwidth]{hSLPlots_C10_allLight_VtxSmear0cm_VtxShiftX0cm_33p5ns_center.pdf}
  \caption{Spherical harmonics comparison between $^{130}$Te 0{\nbb}
    decay signal ($Q=2.529$~MeV) (\emph{red}) and $^{10}$C solar
    neutrinos background (blue) for 1000 simulated events originated
    at the center of the sphere. $^{10}$C with energy deposition
    between 2.1~MeV and 2.9~MeV are considered. Perfect vertex
    reconstruction - true vertex position is used. Time cut of 33.5~ns
    on the photon arrival time is applied. \emph{Top left:} S$_0$
    versus S$_1$ scatter plot. \emph{Top right:} S$_2$ versus S$_3$
    scatter plot. \emph{Bottom left:} Distribution of the
    S$^{C10}_{01}$ variable calculated for signal (\emph{red}) and
    background (\emph{green}). \emph{Bottom right:} Distribution of
    the S$^{C10}_{23}$ variable calculated for signal (\emph{red}) and
    background (\emph{green}).}
  \label{fig:SL_C10_33p5ns_center}
\end{figure}


Comparison of spherical harmonics is shown in
Fig.~\ref{fig:SL_C10_33p5ns_center}. $^{10}$C events are generated at
the center of the detector. True vertex position is used to apply a
33.5~ns time cut to select photons for the spherical harmonics
analysis. The separation is seen in S0 vs S1 and S2 vs S3 scatter
plots. We project both scatter plots to a line that gives maximum
separation (two bottom panels in Fig.~\ref{fig:SL_C10_33p5ns_center}).  
There is enough separation between the distributions to suggest that this analysis can be used to distinguish between 0{\nbb} and $^{10}$C events.

%\section{0{\nbb} decay vs backgrounds from Th and U series}
