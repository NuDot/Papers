We discuss suppression of background due to \C~decay in a large liquid scintillator detector
for neutrinoless double beta decay (0\nbb-decay). This background becomes significant at a
shallow detector site. While there are several handles to discriminate between 0\nbb-decay 
signal and \C~background we discuss an additional separation that comes from precise 
measurement of the photon arrival times in each candidate event. Using a simulation of 
a 6.5~m radius liquid scintillator detector with 100~ps resolution photo-detectors, we show
that the photons arrival time distribution in \C~events is shifted towards longer arrival times 
compared to 0\nbb-decay events. 

Since 98\% of \C~decays through an excited state of
$^{10}$B(718), which has an unusual half-life time of ∼1 ns, the majority of \C~events have 
a prompt positron accompanied by a delayed 0.718 MeV gamma.
Lower kinetic energy of the positron in \C~events compared 
to kinetic energy of the electrons in 0\nbb-decay events diminishes the amount of early 
light produced by the primary charged particles. Even when the $^{10}$B(718) de-excitation time 
is smaller than the photo-detector time resolution there is still a delay in the arrival time of
the photons associated with multiple interactions of the gamma(s).

We also note that the shift in the photons arrival time distribution is further increased in about 
50\% of \C~events in which an ortho-positronium is formed after the positron thermalized in the
liquid scintillator.

